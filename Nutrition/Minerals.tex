\documentclass[../main.tex]{subfiles}
\graphicspath{{\subfix{../images/}}}
\begin{document}

Minerals\index{mineral} are vital inorganic compounds, which can't be synthesized by the organism
and therefore have to be furnished with the food.
There's almost no process in the the human organism where there is not at least one mineral involved,
if it is a chemical, electrical or transport process.
Similar to the vitamins, every food contains its very own characteristic set of minerals and contributes it's part
to covering the daily need of minerals.

Different from vitamins, minerals are pretty inert to most food preparation methods (heating), but they can be leached out into
the water used for cooking if the foods are cooked too long.
If the cooking water isn't also eaten, they nevertheless can be missing from our food.

\vspace{5mm}
\noindent Minerals get classified into the the following two groups, according to the amount in which they  appear in the human body:

\begin{description}
  \item[Bulk elements] consist of more than 0.1\% of the body weight
\item[Trace elements] consist of less than 0.01\% of the body weight
\end{description}

\begin{table}[htb!]
  \centering
  \begin{tabular}{ll}
    \textbf{Bulk elements} & \textbf{Trace elements} \\
    \hline
    Sodium & Iron \\
    Potassium & Zinc \\
    Chloride & Iodine \\
    Calcium & Fluorine \\
    Phosphorus & Selenium \\
    Magnesium & Copper \\
                           & Manganese \\
                           & Chromium \\
    & Molybdenum \\
  \end{tabular}
  \caption[Most important minerals]{Some of the most important mineral nutrients}
\end{table}

Minerals have very diverse functions:
\begin{itemize}
\item there's barely a process in the human body, where there's not at least one mineral contributing
  \item some are components of hormones (ie. iodine in the thyroid hormone\index{hormone!thyroid})
\end{itemize}

\subsection{How much Minerals Does a Human Being Need?}

    The table minerals (table~\ref{tab:minerals}, p.~\pageref{tab:minerals}) shows the daily amount of minerals,
    that an average human should consume every day in order to prevent deficiency syndromes.
    The requirement of minerals can be significantly higher, depending on life situation
    (ie. children, adolescents, pregnancy and breast feeding, disease and last but not least stress).

    \textbf{Careful:} Given that overdosing with minerals is very dangerous (poisoning),
    you should seek counsel with a expert before subsisting.

\end{document}