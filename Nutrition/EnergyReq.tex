\documentclass[../main.tex]{subfiles}
\graphicspath{{\subfix{../images/}}}
\begin{document}

The total energy requirements\index{energy requirements} of humans per day can be calculated from the basic turnover and the activity turnover.

\vspace{2mm}

\begin{center}
\begin{fminipage}{8cm}
  Total turnover  = basic turnover + activity turnover \\
  (total energy requirements)
\end{fminipage}
\end{center}

\vspace{2mm}

\subsection{Basic Energy Turnover}

The basic energy turnover is the energy that our body needs while totally inactive, lying down, in order to
stabilise the body temperature and to maintain the vital functions of respiration, circulation and metabolism for 24 hours.
The basic turnover dpends on factors like for indstance age, gender, stature, weight and height.
Also the ratio between muscle mass and body fat plays a role.
Muscles require more energy than fat tissue.
Given that men typically have more muscle mass and less fat tissue than women, their basic turnover is in average ten percent higher.
So, by increasing the muscle mass, for instance by an increased amount of sport, the basic turnover increases in turn.
On the other side, stress, sickness and the intake of medication aslo influences the basic turnover.

\end{document}