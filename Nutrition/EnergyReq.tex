\documentclass[../main.tex]{subfiles}
\graphicspath{{\subfix{../images/}}}
\begin{document}

The total energy requirements\index{energy requirements} of humans per day can be calculated from the basic turnover and the activity turnover.

\vspace{2mm}

\begin{center}
\begin{fminipage}{7cm}
  Total turnover   (total energy requirements)\\
  = basic turnover (basal metabolic rate)\\
  + physical activity energy expenditure
\end{fminipage}
\end{center}

\vspace{2mm}

\subsection{Basic Energy Turnover (Basal Metabolic Rate)}

The basic calorie turnover\index{basal metabolic rate} is the energy that our body needs while being totally inactive, lying down, in order to
stabilize the body temperature and to maintain the vital functions of respiration, circulation and metabolism for 24 hours.
The basic turnover depends on factors like for instance age, gender, stature, weight and height.
Also the ratio between muscle mass and body fat plays a role.
Muscles require more energy than fat tissue.
Given that men typically have more muscle mass and less fat tissue than women, their basic turnover is in average ten percent higher.
So, by increasing the muscle mass, for instance by an increased amount of sport, the basic turnover increases in turn.
On the other side, stress, sickness and the intake of medication also influences the basic turnover.


\subsection{Calculation of the Basal Metabolic Rate}

The following formula, which takes into account some of the above mentioned factors has been published in the year 1918
by \textit{A. Harris} and \textit{F. G. Benedict}.
The so called ``Harris--Benedict--formula'' is (first in metric: cm and kg, then in the empirical system: lbs and in):

\noindent For men:\footnote{A revision from 1984 by {Roza} and {Shizgal} improved the accuracy and {Mifflin et al.}
 published 1990 an equation more predictive for modern lifestyles \cite{WikiCalorie}.}\\
Basal metabolic rate   [kcal/24 h] \\
= 66.47 + (13.7 * weight [kg]) + (5.0 * height [cm]) - (6.8 * age [yrs]) \\
= 66.47 + (6.2 * weight [lbs]) + (12.7 * height [in]) - (6.8 * age [yrs])

\vspace{3mm}

\noindent For women:\\
Basal metabolic rate  [kcal/24 h] \\
= 655 + (9.6 * weight [kg]) + (1.8 * height [cm]) - (4.7 * age [yrs]) \\
= 655 + (4.3 * weight [lbs]) + (4.7 * height [in]) - (4.7 * age [yrs])

\vspace{3mm}

\noindent Example calculation for a basic calorie turnover(basal metabolic rate):\\
A 35 years old man, 74 kg and 175 cm tall\\
Basic calorie turnover = 66.47 + (13.7 * 74) + (5 * 175) - (6.8 * 35) = 1712.3 kcal/24h

\vspace{3mm}

For the daily use, the following reference values are typically sufficient,
based on reference values for height and weight is enough (by DACH 2000):

\begin{table}[htb]
  \centering
  \begin{tabular}{c c c c c}
    \textbf{Age} & \multicolumn{2}{c}{\textbf{Women}} & \multicolumn{2}{c}{\textbf{Men}} \\
    years & kJ & kcal & kJ & kcal \\
    \hline
    15 -- 18 & 6113 & 1460 & 7620 & 1820 \\
    19 -- 24 & 5820 & 1390 & 7620 & 1820 \\
    25 -- 50 & 5610 & 1340 & 7285 & 1740 \\
    51 -- 64 & 5317 & 1270 & 6615 & 1580 \\
    above 64 & 4899 & 1170 & 5903 & 1410 \\
  \end{tabular}
  \caption{Basal metabolic rate}
\end{table}

\subsection{Physical Activity Energy Expenditure}

The physical activity energy expenditure\index{physical activity energy expenditure} corresponds to the energy needed, which a human needs additionally to the basal energy expenditure for physical and mental activity, for instance for the regulation of the body temperature at different surrounding temperatures or for pregnancy or breast feeding.
Physical activity (using the body's muscles requires more energy than mental activity (with your brain). With additional physical activity, for instance endurance sports, the physical activity energy expenditure can be significantly increased.
To calculate the physical activity energy expenditure the so called PAL (Physical activity level) is used as a measure for the physical activity level.

\subsection[Calculation of the Individual PAL Level]{Calculation of the Individual PAL Level (Factor for the Physical Activity Energy Expenditure)}

Given that humans typically don't maintain the same physical activity level over the course of 24 hours, the individual daily PAL\index{PAL level} level is calculated by splitting the day into three PAL blocks: for sleeping, work and free time lasting 8 hours each.
The average of these three PAL levels gives then the factor for the physical activity energy expenditure of the person:

\vspace{2mm}

\begin{center}
\begin{fminipage}{7cm}
$ Total PAL = \frac{PAL work + PAL free time + PAL sleep}{3}$ 
\end{fminipage}
\end{center}

\vspace{2mm}

\begin{table}[htb]
  \centering
  \begin{tabular}{cll}
    \textbf{PAL value} & \textbf{Activity} & \textbf{Example} \\
    \hline
    0.95 & Sleeping \\
    1.2 & exclusively sitting or recumbent & old, fragile people \\
    & living style \\
    1.4 -- 1.5 & exclusively sitting way of life & office clerks, fine \\
              & with little free time activities  & mechanics \\
    1.6 -- 1.7& mostly sitting, at times walking& lab workers, truck drivers, \\
              & and upright activity & students, assembly line workers\\
    1.8 -- 1.9 & mostly walking and upright work & house makers, retail, waiters,\\
              & & mechanics, craftsman\\
    2.0 -- 2.4 & physical exhausting work & construction and forest worker, \\
    & & farmer, professional sports\\
  \end{tabular}
  \vspace{1mm}
  
  \noindent With additional sport activity or exhausting free time activities of about 30-60~min, 4-5 times a week, the PAL value increases by 0.3 units
  \caption{PAL values for work activity levels and free time behaviors, according to DACH 2000}
\end{table}

Example calculation PAL:
For a craftsman, who spends his free time typically sitting and does 3-5 times a week sport during 30-60 min (jogging or weight lifting).

  \begin{tabular}{lll}
    \textbf{Activity} & \textbf{hrs} & \textbf{PAL--value} \\
    \hline
    Craftsman & 8 & 1.9 \\
    Free time activity & 8 & 1.4 \\
    Sleep & 8 & 0.95 \\
    \hline
                      & & 4.25/3 = 1.4 \\
    Regular sports activity & & + 0.3 \\
    \hline
    \textbf{Average PAL} & & \textbf{1.7}\\
  \end{tabular}
  \vspace{1mm}

  \subsection{Total Metabolic Rate (Total Energy Turnover)}

  The total metabolic rate, the total energy turnover is calculated by multiplying the basal metabolic rate with the corresponding PAL value.

  \begin{center}
\begin{fminipage}{9cm}
  Total metabolic rate = basal metabolic rate * PAL value
  \end{fminipage}
\end{center}


\vspace{2mm}
\noindent  Example calculation total metabolic rate:
  35 year old handicraft worker doing sports in his free time:

  \begin{tabular}{ll}
    Basal metabolic rate (see example) & 1712 kcal/24h \\
    PAL (see example) & 1.7 \\
    Total metabolic rate [kcal/24 hrs] & 1712*1.7 = 2910 kcal/24hrs
  \end{tabular}

\vspace{2mm}

\begin{table}[htb]
  \begin{center}
  \begin{tabular}{c|ccc|ccc}
    \textbf{Age} & \multicolumn{3}{c}{\textbf{Women}} & \multicolumn{3}{c}{\textbf{Men}} \\
    years & PAL 1.4 & PAL 1.6 & PAL 1.8 & PAL 1.4 & PAL 1.6 & PAL 1.8 \\
    \hline
    15 -- 18 & 2000 & 2300 & 2600 & 2500 & 2900 & 3300 \\
    19 -- 24 & 1900 & 2200 & 2500 & 2500 & 2900 & 3300 \\
    25 -- 50 & 1900 & 2100 & 2400 & 2400 & 2800 & 3100 \\
    51 -- 64 & 1800 & 2000 & 2300 & 2200 & 2500 & 2800 \\
    over 64 & 1600 & 1800 & 2100 & 2000 & 2300 & 2500 \\
  \end{tabular}
  \end{center}
%  \vspace{1mm}

  \begin{small}
  \noindent With additional sport activity or exhausting free time activities of about 30-60~min, 4-5 times a week, the PAL value increases by 0.3 units.

  \vspace{1mm}
  \noindent Pregnant women receive an additional 255 kcal/day, independent of their PAL value.

  \vspace{1mm}
  \noindent Women who are breast--feeding, receive an addition independently of their PAL value:

  \begin{tabular}{ll}
    635 kcal/day & up to the 5th month \\
    525 kcal/day & after the 5th month while fully breast--feeding.\\
    285 kcal/day & after the 5th month while partially breast--feeding.
  \end{tabular}
  \end{small}
  \caption{DACH reference values for the average energy consumption in kcal/day.}
\end{table}

  
\end{document}