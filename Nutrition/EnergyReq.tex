\documentclass[../main.tex]{subfiles}
\graphicspath{{\subfix{../images/}}}
\begin{document}

The total energy requirements\index{energy requirements} of humans per day can be calculated from the basic turnover and the activity turnover.

\vspace{2mm}

\begin{center}
\begin{fminipage}{8cm}
  Total turnover  = basic turnover + activity turnover \\
  (total energy requirements)
\end{fminipage}
\end{center}

\vspace{2mm}

\subsection{Basic Energy Turnover}

The basic calorie turnover is the energy that our body needs while totally inactive, lying down, in order to
stabilise the body temperature and to maintain the vital functions of respiration, circulation and metabolism for 24 hours.
The basic turnover dpends on factors like for indstance age, gender, stature, weight and height.
Also the ratio between muscle mass and body fat plays a role.
Muscles require more energy than fat tissue.
Given that men typically have more muscle mass and less fat tissue than women, their basic turnover is in average ten percent higher.
So, by increasing the muscle mass, for instance by an increased amount of sport, the basic turnover increases in turn.
On the other side, stress, sickness and the intake of medication aslo influences the basic turnover.


\subsection{Calculation of the Basic Calorie Turnover}

The following formula, which takes into account some of the above mentionned factors has been published in the year 1918
by \textit{A. Harris} and \textit{F. G. Benedict}.
The so called ``Harris--Benedict--formula'' is (first in metric: cm and kg, then in the empirical system: lbs and in):

\noindent For men:\footnote{A revision from 1984 by {Roza} and {Shizgal} improved the accuracy and {Mifflin et al.}
 published 1990 an equation more predictive for modern lifestyles \cite{WikiCalorie}.}\\
Basic calorie turnover [kcal/24 h] \\
= 66.47 + (13.7 * weight [kg]) + (5.0 * height [cm]) - (6.8 * age [yrs]) \\
= 66.47 + (6.2 * weight [lbs]) + (12.7 * height [in]) - (6.8 * age [yrs])

\vspace{3mm}

\noindent For women:\\
Basic calorie turnover [kcal/24 h] \\
= 655 + (9.6 * weight [kg]) + (1.8 * height [cm]) - (4.7 * age [yrs]) \\
= 655 + (4.3 * weight [lbs]) + (4.7 * height [in]) - (4.7 * age [yrs])

\vspace{3mm}

\noindent Example calculation for basic calorie turnover:\\
A 35 years old man, 74 kg and 175 cm tall\\
Basic calorie turnover = 66.47 + (13.7 * 74) + (5 * 175) - (6.8 * 35) = 1712.3 kcal/24h

\vspace{3mm}

For the daily use, the follwoing reference values are typically sufficient,
based on reference values for height and weight is enough (by DACH 2000):

\begin{table}[htb]
  \centering
  \begin{tabular}{c|c c|c c}
    \textbf{Age} & \textbf{Women} & & \textbf{Men} \\
    years & kJ & kcal & kJ & kcal \\
    \hline
    15 -- 18 & 6113 & 1460 & 7620 & 1820 \\
    19 -- 24 & 5820 & 1390 & 7620 & 1820 \\
    25 -- 50 & 5610 & 1340 & 7285 & 1740 \\
    51 -- 64 & 5317 & 1270 & 6615 & 1580 \\
    above 64 & 4899 & 1170 & 5903 & 1410 \\
  \end{tabular}
  \caption{Basic caloric turnover}
\end{table}

\end{document}