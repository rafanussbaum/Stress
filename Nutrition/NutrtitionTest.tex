\documentclass[../main.tex]{subfiles}
\graphicspath{{\subfix{../images/}}}
\begin{document}

Nutrition test of the Swiss Society of Nutrtion~\cite{SGE}. Analyze your nutrtional habits.

\vspace{5mm}

\textbf{By filling out this questionnaire below, you easily find out in which areas your nutritional habits leaves room for improvement
  and therefore a change is especially important.}

\vspace{5mm}

Maybe you think now: ``But I already know that for a long time!''
Let yourself be surprised, maybe is what you perceived as a big weakness in your nutrtion as not even that problematic!

On the other side you might have some ``blind spots in your perception'' (like most people),
nutritional habits which are engrained and ``guilty pleasures'' which you might not even have considered before.
That might mean, that a change doesn't even have to start, where you always thought it would have to.

\vspace{5mm}

\textbf{How do my eating habits look like?}

\vspace{5mm}

\begin{longtable}{rp{8cm}|l|l|l|l|p{4mm}|}
  & \textbf{This statement fits} &
      \STAB{\rotatebox[origin=c]{90}{\textbf{4 - totally}}} &
    \STAB{\rotatebox[origin=c]{90}{\textbf{3 - more or less}}} &
    \STAB{\rotatebox[origin=c]{90}{\textbf{2 - mostly not}}} &
    \STAB{\rotatebox[origin=c]{90}{\textbf{1 - not at all}}} & \\
    \toprule
    \endhead
    1 & I take sufficient time for my mels and eat slowly &
    \qedsymbol{} & \qedsymbol{} & \qedsymbol{} & \qedsymbol{} & \\  \cline{7-7}
        2 & I consciously only eat rarely fatty meats and sausages &
    \qedsymbol{} & \qedsymbol{} & \qedsymbol{} & \qedsymbol{} & \\  \cline{7-7}

\end{longtable}

\end{document}
