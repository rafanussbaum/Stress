\documentclass[../main.tex]{subfiles}
\graphicspath{{\subfix{../images/}}}
\begin{document}

In order to function, the human body requires energy.
This energy is gained from different component of the nutrition.
Especially the carbohydrates, which get cleaved into glucose during the digestion deliver a lot of energy.
The glucose travels in the blood stream to the cell of the body, where they are transported into the cell with the help of insulin.
The insulin is synthesized in the beta cells of the Lagerhans' islands of the pancreas and after a meal,
they are excreted into the blood stream in an increased amount.
By doing so, the blood sugar level is kept constant.
If the body doesn't produce enough insulin or if the effect of the insulin is reduced, this regulation of the blood sugar level is disturbed.
This has as an effect, that the blood sugar level is permanently too high.
This excessive sugar can't be used by the body and gets excreted over the urine\index{symptom!urine, sweet}.

If the blood sugar level is too high over an extended periods of time, the classical symptoms of diabetes show up:
an excessive thirst\index{symptom!thirst, excessive}, having to urinate often\index{symptom!urinate, often},
tiredness\index{symptom!tiredness} and weakness\index{symptom!weakness}.
If these symptoms are overlooked and an eventual diabetes isn't treated, then severe damages to the health can ensue.
Among others eye problems\index{symptom!eye problems}, disturbances of the kidney function\index{symptom!kidney, function impaired},
cardiac infarction\index{symptom!cardiac infarction}, stroke\index{symptom!stroke} or the  dreaded diabetic foot\index{symptom!foot, diabetic}.


The term diabetes, respectively the full term diabetes mellitus, encompasses different form of diabetes.
The two most common forms are diabetes type 1\index{diabetes!type 1} (also called juvenile diabetes)
and diabetes type 2\index{diabetes!type 2}.
Diabetes type 2 is with about 90\% the most common type of diabetes.
Diabetes type 2 used to be called adult--onset diabetes, because it used to normally only be present above an age of 60.
This changed drastically in the last few years and the people affected by it are being younger and younger.
Out of this reason, the term adult--onset diabetes or late--onset diabetes aren't used anymore.

As opposed to diabetes type 1, the pancreas is producing insulin, but either it's in insufficient amounts or the insulin isn't effective enough.

Diabetes type 2 --- as opposed to diabetes type 1 ---- counts as a disease of affluence.
The main causes are being overweight and a lack of movement.

Different scientific studies show, that with an active life style the risk of getting diabetes type 2 can get reduced by more than 50\%.

An active lifestyle means:
\begin{itemize}
\item enough movement/exercise
\item a healthy diet
\item reducing the overweight and keep it in check
\end{itemize}

\end{document}
