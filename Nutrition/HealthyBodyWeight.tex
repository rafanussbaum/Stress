\documentclass[../main.tex]{subfiles}
\graphicspath{{\subfix{../images/}}}
\begin{document}

It is generally known, that too much or not enough body weight can have consequences on our health.
A weight below the ideal weight indicates an undersupply of vital nutrients, which expresses itself in deficiency syndromes,
like tiredness, in a decrese of power, susceptability to infections and so on.
Overweight indicates an oversupply with energy rich nutrients and/or a lack of movement.
Severe diseases of the heart--circulation system, metabolism problems like diabetes type 2, up to the metabolic syndrome are the consequences.

But what's a healthy body weight? At which point is it considered overweight or underweight?
From a medical viewpoint, there's no concensus, how much the ``normal body weight'' of a human being should be.
For  a long time, the research of a French surgeon Broca (1824-1880) was the basis to calculate the normal, ideal and overweight:

\noindent
\begin{tabular}{lll}
  Normal weight & = Height in cm &-- 100, ex: 175 cm -- 100 = 75 kg \\
  Ideal weight for men & = Normal weight &-- 10\% \\
  Ideal weight for women & = Normal weight &-- 5\% \\
  Overweight & = Normal weight &+ 10\% \\
\end{tabular}

The big disadvantage of the Broca formula is the fact, that it is only applicable to people with a height between 1.60 m and 1.80 m (5ft3 -- 5ft11).

\subsection{Body Mass Index or BMI}

Nowadays, the body mass index (BMI)\index{bmi}\index{body mass index BMI} is the standard to jusge the body weight.
The necessary metrics of body weight, and height are easy to evaluate exactly.

The formula to calculate the BMI with an example:

\vspace{2mm}
\begin{center}
  $BMI = \frac{Body weight [kg]}{Height [m]^2} = 703 * \frac{Body weight [lbs]}{Height [in]}
  = \frac{75 kg}{(1.75 m)^2} = 24.5$
\end{center}

With the help of the calculated BMI value and the following table, it can be determined if there's under--, normal-- or overweight and the corresponding health risk.

\begin{table}[htb!]
  \centering
  \begin{tabular}{lll}
    \textbf{BMI [kg/m$^2$]} & \textbf{Classification} & \textbf{Health risk} \\
    \hline
    $<$ 18.5 & Underweight & low \\
    18.5 -- 24.9 & Normal weight & normal \\
    25.0 --29.9 & Overweight & increased \\
    30.0 -- 34.9 & Adiposity class 1 & strongly increased \\
    35.0 -- 39.9 & Adiposity class 2 & very strongly increased \\
    $\geq$ 40 & Adiposity class 3 & extremely  increased \\
  \end{tabular}
\end{table}

\subsection[Waist--hip--ratio (WHR)]{Waist--hip--ratio (WHR) or how is the fat distributed}
In order to accurately gauge the health risks in connection with overweight, the distribution of the excess fat pads also plays a role.
This can be determined by measuring the circumference of the body.
From the quotient of waist circumference to hip circumference, the so called waist--hip--ratio can be determined.


In order to be mesured, the patient has to be standing up:
\begin{itemize}
\item the circumference of the waist, between the upper edge of the hip and the first rip.
  \item the circumference of the hip at the largest place
\end{itemize}
\end{document}