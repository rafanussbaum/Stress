\documentclass[../main.tex]{subfiles}
\graphicspath{{\subfix{../images/}}}
\begin{document}

Low--carb (low on carbohydrates) is a term used for different diets, which reduces the daily amount of carbohydrates consumed.
Some of these diets are also propagated as long term, permanent diet, therefore low--carb is increasingly seen as a way of  eating.
The different Low--Carb diets, which are used primarily to loose weight, are very different.
For instance the Atkins diet allows almost no carbohydrates at all, where as other avoid the intake of foods with high glycemic index.
Scientific studies proved the effectiveness of diets reduced in carbohydrates as means to loose weight.

The principle of diets poor in carbohydrates isn't new.
Already 1850 was it getting known, by the English William Banting.
He lost \SI{23}{\kilo\gram} (\SI{50}{\lbs}) in one year by adhering to a diet poor in carbohydrates, but rich in meat.

Proponents of the Low--Carb way of living go from the assumptions, that a high ratio of carbohydrates in the diet favors the so called
lifestyle diseases (overweight, obesity, cardiovascular diseases, diabetes type 2).
Another assumption is that the high blood sugar level is high enough by the intake of carbohydrates and that therefore
the body doesn't have to deplete its fat deposits.
Low--carb diets try to reduce the insulin production as much as possible to stimulate the body to gain energy from ketosis
(fat based metabolism).

The most important low--carb diet methods:
\begin{description}
\item[Atkins diet] barely any carbohydrates, but very rich in fats and proteins.
\item[Montignac diet] Carbohydrates are allowed, depending on their GI value, max. 30\% fats.
\item[Logi method] Considers glycemic index and load of carbohydrates, rich in proteins.
\end{description}

\end{document}
