\documentclass[../main.tex]{subfiles}
\graphicspath{{\subfix{../images/}}}
\begin{document}

Your eating habits can decide during stress,\index{stress!nutrition} if the state normalizes or if the stress spiral keeps turning.
Many people instinctively react the wrong way when they are overtaxed.
They uncontrollably consume foods, which they connect with positive emotions and from which they hope to get a relief.
It's not the appetite, but the grasping at straw lead us to consume sweets, potato chips, an immense amounts of caffeine and alcohol.
In a short term they might alleviate the symptoms a bit.
But on the long term will this instinct controlled diet bears big health risks.
By adapting your eating habits and targeting certain foods, you can prevent stress and also lessen the symptoms of existing stress.

\subsection{Careful with Sugar, Alcohol and Caffeine}

Sugar from chocolate and soft drinks gets very quickly in the blood stream and in big quantities.
As we saw, will the pancreas excrete too much insulin, which in turn lower the levels of blood sugar.
This has as an effect that the level of blood sugar spikes, which releases stress hormones.
The stress hormones mobilize sugar from the liver.
During stress, the blood sugar level reacts very sensitively to fluctuations.
These metabolic processes need vitamins and minerals, which aren't contained in refined sugar.
Alcohol\index{food!alcohol} decreases the blood sugar level and therefore decreases the productivity.
Drinking due to stress also bears a huge potential for addictions.
There's nothing to be said against a glass of red wine\index{food!wine, red} with dinner after a day of work.
Consumed in responsible amounts can it indeed help to relax.
Neither is a cup of coffee a bad thing in principle, but if possible not more than that.
People who exaggerate with coffee\index{food!coffee} get more easily affected by stress.
An overdose of caffeine disturbs the labile hormone balance int he body.
As seen in the chapter~\ref{SpecialFoods} ``Special Foods'', page~\pageref{SpecialFoods}, is green tea
an excellent substitute for coffee.

\subsection{Stress Killer Magnesium}
 
A magnesium\index{mineral!magnesium} deficiency means more stress and it can be the cause of many other diseases,
like for instance osteoporosis\index{symptom!osteoporosis}.

The fact that in modern times many people can't cope with the stress often has to too with a
considerable deficiency of magnesium.
Magnesium almost acts like grease in a machine in our body and regulates the intake of calcium\index{mineral!calcium} into the cells.

Calcium is almost like fuel for muscles and glands.
Especially in times of big strain is it a priority of the stress hormones to provide sufficient calcium to the target organs muscles and glands.
To the muscles, because it is urgently needed (fight or flight) and to the glands,
because they have to produce hormones.

\end{document}
