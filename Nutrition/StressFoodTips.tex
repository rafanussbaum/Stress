\documentclass[../main.tex]{subfiles}
\graphicspath{{\subfix{../images/}}}
\begin{document}

Your eating habits can decide during stress,\index{stress!nutrition} if the state normalizes or if the stress spiral keeps turning.
Many people instinctively react the wrong way when they are overtaxed.
They uncontrollably consume foods, which they connect with positive emotions and from which they hope to get a relief.
It's not the appetite, but the grasping at straw lead us to consume sweets, potato chips, an immense amounts of caffeine and alcohol.
In a short term they might alleviate the symptoms a bit.
But on the long term will this instinct controlled diet bears big health risks.
By adapting your eating habits and targeting certain foods, you can prevent stress and also lessen the symptoms of existing stress.

\subsection{Careful with Sugar, Alcohol and Caffeine}

Sugar from chocolate and soft drinks gets very quickly in the blood stream and in big quantities.
As we saw, will the pancreas excrete too much insulin, which in turn lower the levels of blood sugar.
This has as an effect that the level of blood sugar spikes, which releases stress hormones.
The stress hormones mobilize sugar from the liver.
During stress, the blood sugar level reacts very sensitively to fluctuations.
These metabolic processes need vitamins and minerals, which aren't contained in refined sugar.
Alcohol\index{food!alcohol} decreases the blood sugar level and therefore decreases the productivity.
Drinking due to stress also bears a huge potential for addictions.
There's nothing to be said against a glass of red wine\index{food!wine, red} with dinner after a day of work.
Consumed in responsible amounts can it indeed help to relax.
Neither is a cup of coffee a bad thing in principle, but if possible not more than that.
People who exaggerate with coffee\index{food!coffee} get more easily affected by stress.
An overdose of caffeine disturbs the labile hormone balance int he body.
As seen in the chapter~\ref{SpecialFoods} ``Special Foods'', page~\pageref{SpecialFoods}, is green tea
an excellent substitute for coffee.

\subsection{Stress Killer Magnesium}
 
A magnesium\index{mineral!magnesium} deficiency means more stress and it can be the cause of many other diseases,
like for instance osteoporosis\index{symptom!osteoporosis}.

The fact that in modern times many people can't cope with the stress often has to too with a
considerable deficiency of magnesium.
Magnesium almost acts like grease in a machine in our body and regulates the intake of calcium\index{mineral!calcium} into the cells.

Calcium is almost like fuel for muscles and glands.
Especially in times of big strain is it a priority of the stress hormones to provide sufficient calcium to the target organs muscles and glands.
To the muscles, because it is urgently needed (fight or flight) and to the glands,
because they have to produce hormones.

Magnesium regulates the intake of calcium into glands and muscles.
The stress hormones tend to ``exaggerate'' a bit at times and too much calcium gets into the glands and muscles,
which means that they get activated.
That means for instance for the glands to produce a lot of hormones, including stress hormones.
When now magnesium is missing will the stress hormones pump too much calcium into muscles and glands.
This leads to a heightened activity of the glands\index{symptom!gland hyperactivity},
also of the glands producing stress hormones and ``hard and tense'' muscles\index{symptom!muscles, tense and hard}.

Osteoporosis\index{symptom} also can be a consequence of a magnesium deficiency.
When magnesium is lacking, too much calcium is send to the muscles.
Where does the body get that calcium from?
It has to take it from the bones, which can be a cause of osteoporosis.

\begin{table}{htb}
  \caption{Some foods and their magnesium content}
  \centering
  \begin{tabular}{lccc}
    \toprule
    \textbf{Content per \SI{100}{\gram} (\SI[parse-numbers=false]{\text{\ensuremath{3\sfrac{1}{2}}}}{\oz})} & Magnesium & & Magnesium \\
    & \unit{\milli\gram} & Kcal & mg/Kcal \\
    \midrule
    Apples & 5 & 58 & 0.09 \\
    Bananas & 31 & 85 & 0.36 \\
    Bread (Rye) & 40 & 254 & 0.16 \\
    Carrots & 21 & 40 & 0.53 \\
    \rowcolor{lightgray} Cashew nuts & 480 & 598 & 0.80 \\
    Chicken & 37 & 128 & 0.29 \\
    Chocolate (dark) & 133 & 505 & 0.26 \\
    Codfish & 28 & 78 & 0.36 \\
    Eggs & 13 & 162 & 0.08 \\
    Emmental Cheese & 55 &    398 & 0.14 \\
    \rowcolor{lightgray} Lettuce (butter head) & 10 & 14 & 0.71 \\
    Milk, whole & 13 & 64 & 0.20 \\
    Potatoes & 27 & 76 & 0.36 \\
    \rowcolor{lightgray} Pumpkin Seeds & 535 & 400 & 1.33 \\
    \rowcolor{lightgray} Spinach & 62 & 26 & 2.38 \\
    Yogurt & 14 & 71 & 0.20 \\
  \end{tabular}
\end{table}

Magnesium also helps to get as much energy out of sugar as possible.
In stress situations uses the body a lot of energy.
The stress hormones get a lot of sugar from the liver into the muscles.
The sugar reserves in the liver get depleted quickly.
This sudden decrease in turn leads to excretion of stress hormones, what leads in turn to more stress.
If we have a  sufficient amount of magnesium the body goes easy on the liver, which means that the stress
hormones don't have to get overly active.

\subsection{Chocolate is an Anti Stress Measure}

Stressed people often will instinctively grab sweets, as we've seen before.
By doing so they often self medicate unconsciously with a very efficient anti stress remedy, chocolate\index{food!chocolate}.
Especially if you let a few pieces of chocolate --- the darker the better --- melt in your mouth.
Chocolate is the one popular food, which contains that contains sugar and magnesium in high quantities.
A bar of dark chocolate (\SI{100}{\gram} or \SI[parse-numbers=false]{\text{\ensuremath{3\sfrac{1}{2}}}}{\oz}) can contain up to
\SI{300}{\milli\gram} of magnesium, brown chocolate less.
The rule of thumb is the cocoa content of the chocolate, the more cocoa the more magnesium does the chocolate contain.
The daily dosage (about \SI{500}{\milli\gram}) is then almost already covered.
And as seen above, magnesium can help to regulate the stress hormones and keep them in check.

But why exactly chocolate?
Other foods also have a lot of magnesium, for instance nuts, especially cashew.
Chocolate has magnesium and sugar.
In the meanwhile you know, that stress leads to a sudden decrease of the blood sugar level, because of
too much sugar being transported from the liver to the muscles.
The sugar in the chocolate helps to deplete the liver less.
The magnesium from the chocolate helps that the energy of the sugar can be used in an optimal way
and on the other hand that less calcium is transported to the muscles and glands.
Therefore the glands are producing less, including stress hormones.
Otherwise, stress would lead to additional stress!

When you savor a few pieces of chocolate (as dark as possible) in an acute stress situation,
you can count on the effect of calming the stress after about 15 minutes.
This effect can be increased by simultaneously getting some movement with a brisk walk (maybe in a break).\cite{PortaStress}
\end{document}
