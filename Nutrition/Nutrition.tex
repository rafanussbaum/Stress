\documentclass[../main.tex]{subfiles}
\graphicspath{{\subfix{../images/}}}
\begin{document}

\section{Nutrition and Health}

The institute of health promotion Switzerland\footnote{We are taking here Switzerland as an example for the health data.
  The original material has been compiled in Switzerland by Kilian Schmidt.
  Another point is in it's function as a representative western, first world country
  which is pretty healthy, and not so much affected by the first world problems as the USA.} leads on a regular base representative surveys
in the general population of Switzerland to the topic ``health and movement'',
in order to find out how sensitized the general population is with respect to these topics.

The most important outcomes are:
\begin{itemize}
\item A big part of the Swiss society seems to know a lot about the topics nutrition, movement and weight.
\item Two thirds of the people surveyed estimate that they can judge for themselves, what is healthy and what is unhealthy nutrition.
\item Half of the surveyed population says that they are informed about movement for the sake of health benefits.
\item In average 80\% say that they can estimate themselves, if their weight is healthy.
\item Only one in three sees a need to act in their own case.
\end{itemize}


\begin{center}
  \begin{fminipage}{4.5cm}
    \textbf{Our most important mistakes with food} \\
    We eat too:
    \begin{itemize}
      \setlength\itemsep{0em}
    \item much
    \item often
    \item fast
    \item sour
    \item sweet
    \item much proteins
    \end{itemize}
    \vspace{2mm}
    \ldots and we don't drink enough
 \end{fminipage}
 \end{center}



  \subsection[Nevertheless, the People get More Obese and Sick]{Nevertheless, the People get More and More Obese and Sick.}

  \begin{itemize}
  \item The health survey of the Swiss Federal Institute for Statistics in the years 92, 97 and 2002
    showed a rapid increase in overweight (BMI 25--30) and obesity (adiposity) (BMI $>$ 30).
  \item In 2002 were 29.4\% of the Swiss adults overweight and 7.7\% obese.
  \item 2008, already every 5th child was overweight and every 20th child was obese.
  \item In the year 2000, 158,000 Swiss citizens (2.5\% of the population) suffered from Diabetes type 2. This number doubles about every ten years.
  \item Every person living in Switzerland eats every year about 106 lbs (48 kg) sugar.\index{sugar}
  \end{itemize}

\subsection{What are the Reasons Behind This?}  

The reasons for this steady increase of overweight\index{overweight}\index{obesity}\index{symptom!overweight}\index{symptom!obesity}
and diabetes type 2\index{diabetes!type 2}\index{symptom!diabetes, type 2}
  are to be found in the ``unhealthy foods'',\index{food!unhealthy}
which are offered everywhere in abundance.
The modern foods consist more and more in energy dense foods and sweet drinks in always increasing portion sizes.
This goes along with a decreasing physical activity, as more and more transportation is done with car or public transportation.
That's the reason, why nutrition respectively energy intake and movement are more and more out of equilibrium.

Our ancestors mostly ate fresh foods, like nuts, seeds, roots, veggies, berries, herbs, wild grains,
legumes with fish or game in small portions.
Carbohydrates mostly came from the wild grains, which were eaten as whole grains and also contain plenty
of fibers\index{fiber}, vitamins\index{vitamin}, minerals\index{mineral} and trace elements\index{trace elements}.
Refined carbohydrates and sugars were not part of that diet, very much the opposite of the situation today.\index{carbohydrate}

Additionally, the nutrition of our ancestors consisted of way lees fat\index{fat} than ours, and the amount of
highly unsaturated fats\index{fat!unsaturated} (the healthy fats from plant oils) was about 3-4 times higher
than the amount of saturated fats\index{fat!saturated}
(unhealthy animal fats).
Nowadays, humans typically consume 2 to 3 times more saturated fats than unsaturated fats.
The meat of the domesticated animals contain way more fats (25-30\% for beef and pig)
compared to the game our ancestors lived off (about 4\% fat).
The game our ancestors ate also delivers big amounts of Omega--3--fatty acids\index{fatty acids!Omega--3},
which is almost entirely absent in the meat of domesticated live stocks.

Our ancestors  were also better supplied with vitamins and minerals, as the following table shows:

\begin{table}[htb]
  \centering
  \begin{tabular}{l|l|l}
    & Nutrition & Nutrition \\
    & of our ancestors & today \\
    \hline
    Folic acid (mg/d) & 360 & 170 \\
    Vitamin A  ($\mu$g/d) & 17 & 7 \\
    Vitamin C (mg/d) & 600 & 80\\
    Zinc (mg/d) & 43 & 10 \\
    Calcium (mg/d) & 2000 & 750 \\
    Potassium (g/d) & 10.5 & 2.5 \\
    Sodium (g/d) & 0.8 & 4 \\
    Fibers (g/d) & 100 & 12 \\
    Total fats (\% of calories) & 21 & 42 \\
    \hline
  \end{tabular}
  \caption[Nutrition of our ancestors.]{The nutrition of our ancestors compared with ours.~\cite{EatonAl}}
\end{table}

The human body did barely change over thousands of year, seen from a genetic perspective.
The bodies of modern humans don't work any different and aren't build any different than the ones of stone age people.
Even though our colons and digestive organs are barely different than the ones of our ancestors from the stone age,
we except it cope with a strongly refined nutrition poor in nutritional value, which consists of a lot of sugars, salt,
animal fats and countless other chemical additives.\index{additives, chemical}

In western industrialized nations, the modern food industry would be used to produce enough and valuable food.
But nevertheless, big parts of the society are nourished in a wrong way and under supplied with vital substances\index{vital substances}
like vitamins, minerals and the essential amino acids and fatty acids, which are so important for maintaining a good health up into the high age.

The big challenge for all of us consists in choosing and combining the foods which supply our body in a balanced way with all the necessary substances
from the big offer of product to buy.
Balanced means, that the body is supplied with the ideal amount of energy and on the other side with the necessary vital substances.

\begin{center}
\begin{fminipage}{4cm}
  Take time for yourself and find out, what is good for you, and what you need to feel well.
\end{fminipage}
\end{center}



\end{document}