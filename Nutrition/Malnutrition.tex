\documentclass[../main.tex]{subfiles}
\graphicspath{{\subfix{../images/}}}
\begin{document}

\section{Problems of the Acid Base Balance}

The human organism is a true marvel of nature.
An inestimable amount of feed--back loops are constantly at work to provide the ideal conditions to  produce optimal results.
Only like that can the body be par to the challenges of our life.
The acid base balance\index{acid base balance} is one of these control systems.
It constantly ensures a balance of acids and bases in our body and therefore leads to homeostasis.
Our modern way of life with increasing amounts of stress and also pollution depletes our body from the ingredients for this feed--back loop.
The effects are a lowered quality of life, additional stress and sooner or later problems with our health.

\subsection{The Latent Acidosis}

The term latent acidosis\index{problem!acidosis, latent} or chronic hyper acidity\index{hyperactivity} describes a state
of the acid base balance, when the necessary reserves of buffer to buffer the acids in the blood are already at least partially depleted,
but there's no shift in the pH of the blood yet (acute acidosis).
In contrary to the acute acidosis, which immediately threatens the life,
the latent acidosis only damages the organism after some time (often years).
The symptoms are very diffuse and unclear (tiredness\index{symptom!tiredness}, brittle hair\index{symptom!hair, brittle} and finger nails\index{symptom!finger nails, brittle}, rheumatic problems\index{symptom!rheumatic problems} or allergies\index{symptom!allergies}),
so it can stay undetected for a long time.

\subsection{Causes of the Latent Acidosis}

There many different types of causes, which contribute or lead to a latent acidosis.
We can distinguish them roughly into inner (due to metabolism) and outer (due to actions) causes.

\subsubsection{Inner Causes}

\begin{description}
\item[Stress, anger, troubles, negative thinking:] Stress leads to an increased excretion of adrenaline
  and to an constantly active sympathetic nervous system. This leads to an increase of the anaerobic metabolism
  in the cells and therefore to an increase in the formation of acids.
\item[Diseases:] For instance while having the flu, due to the increased metabolism there will be more acids produced in our body.
\item[Viscous circles of compensation:] In order to keep the pH value of the blood constant,
  there are multiple mechanism to compensate for too much acid.
  One of these mechanisms leads to the acids (also called acid slags) into the connective tissue.
  The connective tissue becomes rigid and less circulated by blood.
  This leads to a reduced oxygenation of these cells, anaerobic metabolism and an increase production of acids.
  Another mechanism holds back the acids by incorporating them into the erythrocytes (red blood cells).
  The become rigid and immobile, so they can't fit through the narrowest capillary vessels anymore.
  The supply of oxygen in the tissue decreases and with it again the anaerobic metabolism and acid production increase in turn.
\end{description}

\subsubsection{Outer Causes}

\begin{description}
\item[Malnutrition:] Too much meat, milk products, refined grains in the form of bread and pasta.
  The proportion of foods which have a surplus of bases, like vegetables and fruits, tends to shrink.
  This leads to an increased intake of acids and simultaneous reduced intake of bases,
  Both leads to an acidification of the body.
\item[Diets to loose weight:] The above mentioned malnutrition leads to a gain in weight,
  which has to be reduced by loosing weight.
  This forced burning of fats produced acids in the form of keto acids.
\item[Lack of movement:] A lack of movement leads to the muscle becoming flaccid and badly circulated with blood.
  In these weakened muscles, acids will be deposited.
\item[Fitness (delusion):] Exaggerated physical movement, as it is often practiced as recreational activity these days often leads to
  an anaerobic metabolism in the muscles and an increased production of lactates.
\item[Environmental factors:] Pollution of the air, noise, electro smog, and acidic soils are important causes of an increased acidification.
\item[But also:] Smoking, alcohol and medication.
\end{description}

\subsubsection{Symptoms of the Latent Acidosis}

Due to different compensation mechanisms, the syptoms of a latent acidosis only show up later.
The first symptoms are only seen as an effect of the compensaion mechanisms ``acid storage'' and ``demineralisation''
or then as a direct effect of the overacidification on the function of the cells and organs.

\begin{outline}
  \1 Symptoms of acid deposition (acid storage)
  \2 Allergies\index{symptom!allergies}
  \2 Pain in the muscles\index{symptom!muscle pain} (especially in the shoulder and neck region)
  \2 Weight problems\index{symptom!weight problems}
  \2 Cellulite\index{symptom!cellulite}
  \1Symptoms of demineralisation
  \2 Brittle finger nails and hair\index{symptom!hair, brittle}\index{symptom!finger nails, brittle}
  \2 Tooth decay\index{symptom!tooth decay}
  \2 Problems of the joints\index{symptom!joint problems}
  \1 Symptoms of disturbed fucntions of cells and organs
  \2 Gastrointestinal problems\index{symptom!gastrointestinal problems}: being bloated\index{symptom!bloated}, postprandial fullness\index{symptom!postprandial fullness},
  itching of the anal region\index{symptom!anal region, itching}
  \2 The looks of the skin: unpure skin\index{symptom!skin, unpure}, inflamations\index{symptom!skin, inflamations}, dermal mycosis\index{symptom!skin, mycosis} (fungal infection of the skin),
  acen\index{symptom!acne}
  \2 General wellbeing: tiredness\index{symptom!tiredness}, head aches\index{symptom!head ache}, anxiety\index{symptom!anxiety},
  sexual disinclination\index{symptom!sexual disinclination}, problems sleeping\index{symptom!sleeping problems}, difficulties focussing\index{symptom!focussing difficulties}
  and mood swings\index{symptom!mood swings}
  \end{outline}

  \subsubsection{Consequences of a Latent Acidosis for our Health}
  
  Pretty much every function of our body is based on chemical reactions, which are highly depended of the pH of the surrounding.
  Therefore, the consequences of chronic hyperacididty are very diverse.
  These days, odern traditional medicine recognizes diseases like osteoporosis\index{symptom!osteoporosis}, gout\index{symptom!gout} and kidney stones\index{symptom!kidney stones}
  as a consequence of a latent acidosis.
  Wholistic medicine also recognizes the following diseases in context with a latent acidosis (not an exhaustive list):
  \begin{description}
  \item[Cardiovascular diseases:\index{symptom!cardiovascular diseases}] disturbed blood flow in the extremities with chronic cold hands and feet,\index{symptom!hands and feet, cold}
    tension head aches\index{symptom!tension head ache}, angina pectoris\index{symptom!angina pectoris}, arteriosclerosis\index{symptom!arteriosclerosis},
    afflictions of the veins\index{symptom!veins, afflictions}, thrombosis\index{symptom!thrombosis}, hemmorrhoids\index{symptom!hemmorhoids}
  \item[Problems of the general wellness:] Chronic fatigue\index{symptom!fatigue}, states of exhaustion\index{symptom!exhausted, feeling of},
    neuralgic headache\index{symptom!neuralgic headache}, mood swings\index{symptom!mood swings}, insomnia\index{symptom!insomnia}
  \item[Diseases of the excretion organs skin, lungs, kidneys, colon:] acne\index{symptom!acne}, dermal mycosis\index{symptom!skin, mycosis},
    allergies\index{symptom!allergies}, excema\index{symptom!excema}, asthma\index{symptom!asthma}, kidney stones\index{symptom!kidney stones}, gall stones\index{symptom!gall stones},
    irritable bowel syndrome\index{symptom!irritable bowel syndrome}
  \item[Orthopedic diseases:] damage to the invertebral disks\index{symptom!invertebral disks, damage}, arthritis\index{symptom!arthritis}, arthroses\index{symptom!arthroses},
    fibromyalgia sydrome\index{symptom!fidromyalgia sydrome}
  \item[Metabolic diseases:] Diabetes\index{symptom!diabetes} and gout\index{symptom!gout}
  \end{description}

  \subsubsection{Measurements to Determine a Latent Acidosis}

  The most exact method of determining a latent acidosis is the measurement of the buffer capacity of the blood with the method of J\"orgensen and Stirum.
  This method is very laborious and can only be perfomed a doctor in a specially equipped laboratory.
  A second method is measuring the pH value of the urine.
  
  
\end{document}
