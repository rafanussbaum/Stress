\documentclass[../main.tex]{subfiles}
\graphicspath{{\subfix{../images/}}}
\begin{document}

Vitamines\index{vitamine} are vital organic compounds, which the human body needs for vital body functions.
Vitamines can't be synthesized by the human body, or only in insufficient amounts or only certain external conditions,
like for instance that sun light is present.
Vitamines have to be supplemented, as vitamines themselves or as their precursors, provitamines.
Out of these provitamines, the body can then build the active form, the vitamines.
Pretty much every food pretty much has their own set of vitamines contained and contributes to the daily needs being covered.

Vitamines can be categorized into the following two groups:
\begin{description}
\item[water soluble (hydrophilic)] are easily soluble in water
  \item[fat soluble (lipophilic)] are soluble in fats or oils
  \end{description}

  The human body can't store water soluble vitamines, or just very insufficiently.
  They have to be added to the diet on a regular base.
  One exception is vitamines B12\index{vitamine!B12}, which the body can stockpile the needs for up to multiple years.
  Water soluble vitamines in excess get right away excreted over the kidneys.

  Fat soluble vitamines on the other hand get stored in the body (ie. in the liver), and shouldn't be overdosed.

  \begin{table}[htb!]
    \centering
    \begin{tabular}[t]{ll|l}
      \multicolumn{2}{c}{\textbf{water soluble}} & \textbf{fat soluble} \\
      \hline
      B$_1$ & panthothenic acid (B$_5$) & A \\
      B$_2$ & biotin (B$_7$ or H) & D \\
      niacin (B$_3$) & B$_{12}$ & E \\
      B$_6$ & C & K \\
      Folic acid (B$_9$ or B$_{11}$) & & Beta carotene\\
    \end{tabular}
    \caption{The vitamines}
  \end{table}

  Vitamines have very diverse functions:
  \begin{itemize}
  \item they are responsible for the proper operation of the metabolism
  \item they regulate the deconstruction and reconstruction of carbohydrates, proteins and fats and are responsible for an adequate supply of energy
  \item they reinforce the immune system\index{immune system}, regulate the clotting of blood and have an anti-oxidative effect
    \item they are indispensable to build up cells, blood cells, bones, \ldots
    \end{itemize}

    \subsection{How much Vitamines Does a Human Being Need?}

    The table vitamines (table~\ref{tab:vitamines}, p.~\pageref{tab:vitamines}) shows the daily amount of vitamines,
    that an average human should consume every day in order to prevent deficiency syndromes.
    The given daily amounts corresponds to the EU--RDA--values.
    EU--RDA stands for EU recommended daily allowance and consists of suggestions which originally come from the US and which
    have been taken over by the European Union.
    The requirement of vitamines can be significantly higher, depending on life situation
    (ie. children, adolescents, pregnancy and breast feeding, disease and last but not least stress).

    \textbf{Careful:} Given that vitamines can be overdosed, you should seek counsel with a expert before substituting.

    \subsection{Natural or Synthetic Vitamines?}
    Vitamines are nowadays offered as dietary supplement in various forms or even added directly during the production of the food.
    Vitamines can come from natural sources (plant or animal based) or be synthetically prepared (in the laboratory).
    There are widely differing opinions if synthetic vitamines are of equal rank as natural vitamines.
    There's evidence that synthetic vitamines aren't as well resorbed (ingested) and that they will be excreted faster.
    On the other side, natural sources can be contaminated, for instance vitamine E by gluten (source wheat germ)
    or the environmental aspect, as the needed amount of plants.
\end{document}