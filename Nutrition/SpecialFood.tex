\documentclass[../main.tex]{subfiles}
\graphicspath{{\subfix{../images/}}}
\begin{document}


\section{Coffee}

A cup of coffee\index{food!coffee} needs \SI{24}{hours}, to pass through the kidneys and the urine tract.
When you drink multiple cups of coffee a day, then this will put a load on the internal organs.
Decaffeinated coffee is better in that respect, but it still forms acids in our body.

When you drink a coffee with your food, you force the food faster out of your digestive system
and it decreases the mobility of the colon. Undigested food is the main reason for constipation\index{symptom!constipation} or on the other side a reason for the too fast processing, diarrhea\index{symptom!diarrhea}.
Coffee and black tea are strongly acid forming.
The more acid in the blood, the more water the body remains to neutralize or dilute the acid.
Thais is a burden for the body and means a gain in weight.
Try to drink herbal teas instead.

Additionally, caffeine is a stimulant for the central nervous system --- similar to cocaine.
It is associated with a series of conditions, like heightened heart activity\index{symptom!heart activity, heightened},
changes in the vessel diameter\index{symptom!blood vessel, diameter change},
irregular blood flow in the coronary vessels\index{symptom!coronary vessels!irregular blood flow},
heightened blood pressure\index{symptom!blood pressure!heightened},
diabetes\index{symptom!diabetes},
kidney failure\index{symptom!kidney failure},
ulsters in the stomach\index{symptom!ulsters, stomach},
pancreas problems\index{symptom!pancreas problems},
noises in your ears\index{symptom!noises in your ears},
amyostasia\index{symptom!amyostasia},
feeling restless\index{symptom!restlessness},
sleeping problems\index{symptom!sleeping problems} and
digestive problems\index{symptom!digestion, problems}.
On top of that it messes up the blood sugar level by forcing the pancreas to secrete insulin.

\section{Green Tea}

In green tea\index{food!green tea} are important substances like vitamin C, important minerals and trace elements with positive effects on our health.
Additionally, green tea helps to resorb\footnote{To resorb: Take in certain substances, to sponge up, absorb.} potassium in the body.

\subsection{Caffeine in Green Tea}

A usual cup of green tea (\SI{1.5}{\deci\liter} or  \SI[parse-numbers=false]{\text{\ensuremath{\sfrac{5}{8}}}}{\cup}) contains about \SI{60}{\mg} caffeine.
That's not a negligible amount, so we should look closer at that.
In green tea is the alkaloid caffeine, which stimulates bound to polyphenols\index{polyphenols}, tanning agents.
IT doesn't get suddenly in the blood stream and doesn't put the organism into sudden excitation.
Green tea releases the caffeine in small doses spread out over time to the body.
Therefore green tea acts as a smooth and gentle tonic, which brings us into a state of relaxed attentiveness.
Other drinks containing caffeine truly act like a bomb in our body.
Our reaction to them is sudden and strong, but also it depletes quickly again and the effect passed.
Black tea is in this respect close to coffee, by the treatment of fermentation the caffeine is released.
Black tea does contain less caffeine than green tea, but it's effect is as stimulant than that of coffee.
The caffeine slowly released by green tea keeps you longer awake and in shape and doesn't give the rush.

\subsection{The Balanced Quality of Green Tea}

The relatively high caffeine content is acting strongly on the brain and not as much as other
caffeinated drinks (coffee and Coca Cola drinks) on the cardiovascular system.
A high content in polyphenols avoids a massive release of caffeine to the body.

The thiamine (vitamin B$_1$) of the green tea supplies our brain cells with sufficient energy.
At the same time it avoids that stress spreads through our system by protecting the nervous system against being flooded by external stimuli.

The essentials oils of the green tea spread a pleasant aroma.
They have the effect that the human feels well and that the muscle tension gets reduced.
The effect is similar with a mild narcotic, partially enabling to decouple negative emotions from the nervous system.
The essential oils don't only make the thoughts flow smoother, it has been shown that it leads to a calming of the human motor function.
Researched already described this effect already in the 18$^{th}$ century.
Yet another effect of these essential oils: they put the human in a peculiar slight euphoria.

In the interplay of the different effects of green tea gives it's exceedingly positive effect.

Green tea is definitely not to be counted in the category of psychotropic\footnote{psychotropic: acting on the psyche, the mind (medicine)}
plant medicine, like for instance St. John's wort or Valerian.
These two get used in a targeted way to treat mental and emotional diseases like anxiety and depression.
But as a whole, in the sum of it's pharmaceutical substances, it's taste and preparation methods can green tea have an affect on the psyche.
It can be said, that the psychoactive substances of green tea or so carefully balanced, a pharmacist couldn't do it any better.

Spoil yourself with a good green tea (pay attention to get a good quality) and prepare it in a ritual way.
The water should not be hotter than \SI{80}{\celsius} (\SI{176}{\degree F})\footnote{A convenient way to have \SI{80}{\celsius} water
is to mix one part room temperature water with three parts boiling water, or about one part ice to 6-7 parts boiling water.
Additionally, steep for 2, maximum 3 minutes or the tea tends to get bitter.},
else the precious nutrient are getting destroyed and it tastes bitter\cite{GreenTee}.

\section{Fruit and Vegetable Juices}

There's a big effort these days, to ``improve'' food with corresponding additives.

Most people aren't aware that fresh fruits\index{food!juice, fruit and vegetable} and vegetables and their juices are a very effective supplements.
A diet which contains a good amount of fresh fruit, vegetables and their juices fulfills all the needs of the body.

Juices comes closest to the value of the whole food, as they are their liquid extract.
Enjoy their delicious taste to the fullest: they are the only drinks, which also supply you with vital energy.
Also for infants there's nothing healthier than fresh juices, with the exception of mother's milk obviously.
Consider that industrial food is able to deliver toxic substances to your body, but definitely it can't help
to remove toxins from our bodies.

If you need or want to loose weight, replace one portion of fruits or vegetables per day with a freshly squeezed juice.
With it's enzymes it will help you to dismantle the excessive weight.
Always drink juices slowly and thoroughly enjoy them, mix them well with your saliva, as if you would chew solid food.

Please consider that if you consume fruits and vegetables mostly in the form of juices, that you might be missing the important fibers.
Additionally, juices deliver the sugars in a more concentrated form then the fruit itself.

\section{Milk}

On thing is certain: cow milk has a different chemical composition than human milk.

For the decomposition and digestion of milk, the enzymes rennin\index{enzyme!rennin} and lactase\index{enzyme!lactase}.
For most people, after the age of three years old we don't have these enzymes anymore.
In all types of milk, there's a certain constituent, called casein\index{casein}.
Cow milk contains 300 times more cassein than human milk.
Cassein is used to develop big bones.
Cassein coagulates in the stomach and builds a big, tough, compact clump, hard to digest and about the texture of cream cheese.
The four stomach system of a cow is adapted to deal with that goo.
Once this tough clump of cassein is in the digestive tract has the human body difficulties to get rid of it again.
In other words, a considerable amount of energy is needed to complete that.
A part of this goo hardens and sticks to the walls of the colon and therefore prevents the body from taking in the nutrients.


Result: A sluggishness of the colon\index{symptom!colon, sluggishness}.
The byproducts of the digestion of the milk leave a huge amount of slime\index{symptom!slime} in the body, which is very acidic
and partially has to be stored to be worked up later
--- if there's at this later point in time not already more slime being produced!
By the way, an excessive slime production in the throat and nose also can indicate milk consumption.

The slime production associated with milk consumption is a serious difficulty.
The mucous membranes are coated in a film, which strongly decreases their permeability.
That means nutrients aren't as well absorbed as they would in a diet without milk and milk products.
That can lead to a chronic critical shortage  of nutrients.
Bad nutrient intake also means chronic tiredness\index{symptom!tiredness, chronic}.
The body is in the pattern of a shortage of nutrients (the intake is blocked)
and a big energy loss with the load that puts on the body in order to be processed.

\subsection{Is Milk an Important Source of Calcium?}

There's a widespread belief that milk is an important source of calcium.
Calcium from cow milk can't be taken in by the human body.
Additionally, milk products are building acids in the body, that means in other words
that they deplete your mineral storage (including calcium) to get neutralized.

\textbf{Calcium can be found in other ways.}

You also find calcium\index{mineral!calcium} in:
\begin{itemize}
\item all leafy greens
\item Miso, a Japanese fermented soy product
\item Algae
\item All raw nuts
\item Raw sesame seeds (they contain a lot of calcium)
\item most fruits and vegetables
\item dried fruits, like figs, dates and plumbs
\end{itemize}

Make it a habit to sprinkle some raw sesame seeds over your salad.
Also Furikake, a Japanese spice for rice with sesame seeds and Algae can be very recommended.

Many problems of the respiration tract\index{symptom!respiration tract, problems} and allergies,\index{symptom!allergies}
especially asthma\index{symptom!asthma} are suspected to be related to
the consumption of milk products.
Ear infections in children's\index{symptom!ear infections, children} are also be caused by consumption of milk products.

\section{Meat}

The saliva of a carnivore\index{carnivore}\index{food!meat} is acidic for the digestion of animal proteins.
It doesn't contain any phytalin\index{phytalin}, which is needed to digest starch.
Our saliva is alkaline and contains phytalin in order to digest starch.


The stomach of carnivores are simple bags, which contain ten times more hydrochloric acid then the stomach of herbivores.
Our stomach has an oblong, elongated shape, a complicated structure in is connected over a turn with the duodenum.
The colons of carnivores are three times as long as the body, in order to allow a quick excretion of food which rots quickly.
Our colons are twelve times our height, in order to keep the food a long time to absorb all the nutrients.

The livers of carnivores is able to filter 10--15 times more uric acid than the liver of herbivores.
Our livers can only excrete a small amount of uric acid\index{uric acid}.
Uric acid is a very dangerous poison, that can cause a lot of mayhem in our body.
Each portion of meat produces a big amount of uric acid in our body.
But we humans don't have the enzymes uricase\index{uricase} to decompose the uric acid.
Carnivores and omnivores have that enzyme in their bodies.

Many diseases can be caused by an consumption of too much meat,
like for instance diseases of the vessels and heart\index{symptom!blood vessel}\index{symptom!heart diseases},
gout\index{symptom!gout}, rheumatism\index{symptom!rheumatism}, arthritis\index{symptom!arthritis}, osteoporosis\index{symptom!osteoporosis},
mycosis\index{symptom!mycosis} and allergies of all kind\index{symptom!allergies}.


 
\vspace{5mm}
\noindent
\begin{fminipage}{\textwidth}
  \textbf{Eating less Meat}

  Meat production is responsible for about 18 percent of the global emission of green house gases.
  The numbers of the Food and Agriculture organization of the United Nations FAO and the UNO respect the whole
  cycle of meat production, from clearing the forest, the production and transport of the fertilizers,
  the combustion of the vehicles and the emissions of the animals --- especially with cows and sheep.

  Especially problematic is the fact that one third of the world's production of grains is feed to animals,
  to produce meat products, mild and eggs.
  There are grave consequences for the developing countries: the more grains the farmer produces to export as feed,
  the less area of production they have to grow their own food\cite{pressetext}.
\end{fminipage}


\section{Salt}

Table salt\index{food!table salt} was very rare and valuable for millennia.
Our organism is optimized (for thousands of years and a few more hundred to come)  to retain as much as possible of that valuable and in the past rare substance.
In the last 220 years, humans consume between 8--20 times the amount of salt needed.
The sodium is replaced by potassium and calcium in the liver,  so that they get lost (sacrificed) instead.
In the past, humans ate mostly plant based and had typically plenty of potassium.
Nowadays, it is just the other way round.
Modern humans put too much salt on food (meat, many types of cheese and bread are very bland without adding salt)
and we don't eat enough foods with potassium.
That leeches minerals from our bones.

The effects of a too high intake of salt (sodium) are acidic edema\index{symptom!edema} 
and fluid tissue (cellulite)\index{symptom!cellulite}.
For each gram of sodium chloride that our body stores, because it can't excrete it, it needs 23 times as much cellular water.
Combined with uric acid, it leads to deposits on the bones and joints\index{symptom!deposits bones and joints},
as well as gall and kidney stones\index{symptom!gall stones}\index{symptom!kidney stones}.

\subsection[Pink Himalayan Salt]{Rock Salt from the Himalayas (Pink Himalayan Salt)}

The pink crystalline salt from the Himalayas is natural salt in it's most wholesome form, full of energy.
Most of the minerals and trace elements bound in the salt are in an organic form which is readily available for our cells.
This allows our body to keep up with the intake of the antagonists, like potassium, magnesium and calcium,
as well as all the trace elements, which are involved in this process.

Crystalline salt is in this respect wholesome, contrary to the usual table salt,
which has been refined that it only contains sodium chloride and has been robbed of the the other vital elements.

\section{Sugar}

Sugar\index{food!sugar} is poison for your body.
Almost all foods contain added sugars and it constitutes about 50 percent of the carbohydrates, which we take in with our diet.
A lot of sugar makes tired and upsets the immune system.
The more sweets we eat, the less healthy foods we consume.
This leads to an under supply of vitamins and minerals.
These days everybody knows that sugar is bad for the teeth.
Sugar might sweeten life, but it leads to an acidification of our body.
Sugar is also a vitamin B$_1$ thief, because it doesn't contain it,
but it's essential for the processing of sugar in our body.

Natural sugar from sugar cane contains a high amount of vital trace elements, minerals and vitamins.
The sugar we eat typically lost all those vital nutrients through the process of refination (purification)
and is additionally burdened by the bleaching process.
White sugar paralyzes the peristalsis\index{peristalsis}\footnote{peristalsis: A movement of the walls of the muscular hollow organs (ie. stomach, colon, ureter).
Successive sections of the wall compress and therefore transport the contents of the hollow organ.}
of the colon and burdens the immune system.
It destroys the enamel of the teeth and prepares a fertile breeding ground for bacteria.
It also inhibits the intelligence. This is often not noticed, as the modern society tends to confuse education with intelligence.
If sugar would be introduced to the market as a new product, it would never be  admitted as a food.
Sugar weakens the brain and changes our life.

Most of all, sugar overtaxes the pancreas by forcing it to produce an excessive amount of insulin, in order to lower blood sugar level again\index{blood sugar level}.
The lowered blood sugar level then in turn wakes a ravenous appetite.
Too often we then get the next sweets, which fuels the vicious circle.

\end{document}
