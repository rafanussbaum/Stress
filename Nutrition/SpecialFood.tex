\documentclass[../main.tex]{subfiles}
\graphicspath{{\subfix{../images/}}}
\begin{document}
\label{SpecialFoods} % This will be the section special foods and drinks

\section{Coffee}

A cup of coffee\index{food!coffee} needs \SI{24}{hours}, to pass through the kidneys and the urine tract.
When you drink multiple cups of coffee a day, then this will put a load on the internal organs.
Decaffeinated coffee is better in that respect, but it still forms acids in our body.

When you drink a coffee with your food, you force the food faster out of your digestive system
and it decreases the mobility of the colon. Undigested food is the main reason for constipation\index{symptom!constipation} or on the other side a reason for the too fast processing, diarrhea\index{symptom!diarrhea}.
Coffee and black tea are strongly acid forming.
The more acid in the blood, the more water the body remains to neutralize or dilute the acid.
Thais is a burden for the body and means a gain in weight.
Try to drink herbal teas instead.

Additionally, caffeine is a stimulant for the central nervous system --- similar to cocaine.
It is associated with a series of conditions, like heightened heart activity\index{symptom!heart activity, heightened},
changes in the vessel diameter\index{symptom!blood vessel, diameter change},
irregular blood flow in the coronary vessels\index{symptom!coronary vessels!irregular blood flow},
heightened blood pressure\index{symptom!blood pressure!heightened},
diabetes\index{symptom!diabetes},
kidney failure\index{symptom!kidney failure},
ulsters in the stomach\index{symptom!ulsters, stomach},
pancreas problems\index{symptom!pancreas problems},
noises in your ears\index{symptom!noises in your ears},
amyostasia\index{symptom!amyostasia},
feeling restless\index{symptom!restlessness},
sleeping problems\index{symptom!sleeping problems} and
digestive problems\index{symptom!digestion, problems}.
On top of that it messes up the blood sugar level by forcing the pancreas to secrete insulin.

\section{Green Tea}

In green tea\index{food!green tea} are important substances like vitamin C, important minerals and trace elements with positive effects on our health.
Additionally, green tea helps to resorb\footnote{To resorb: Take in certain substances, to sponge up, absorb.} potassium in the body.

\subsection{Caffeine in Green Tea}

A usual cup of green tea (\SI{1.5}{\deci\liter} or  \SI[parse-numbers=false]{\text{\ensuremath{\sfrac{5}{8}}}}{\cup}) contains about \SI{60}{\mg} caffeine.
That's not a negligible amount, so we should look closer at that.
In green tea is the alkaloid caffeine, which stimulates bound to polyphenols\index{polyphenols}, tanning agents.
IT doesn't get suddenly in the blood stream and doesn't put the organism into sudden excitation.
Green tea releases the caffeine in small doses spread out over time to the body.
Therefore green tea acts as a smooth and gentle tonic, which brings us into a state of relaxed attentiveness.
Other drinks containing caffeine truly act like a bomb in our body.
Our reaction to them is sudden and strong, but also it depletes quickly again and the effect passed.
Black tea is in this respect close to coffee, by the treatment of fermentation the caffeine is released.
Black tea does contain less caffeine than green tea, but it's effect is as stimulant than that of coffee.
The caffeine slowly released by green tea keeps you longer awake and in shape and doesn't give the rush.

\subsection{The Balanced Quality of Green Tea}

The relatively hgih caffeine content is acting strongly on the brain and not as much as other
caffeinated drinks (coffee and Coca Cola drinks) on the cardiovascular system.
A high content in polyphenols avoids a massive release of caffeine to the body.

The thiamine (vitamin B$_1$) of the green tea supplies our brain cells with sufficient energy.
At the same time it avoids that stress spreads through our system by protecting the nervous system against being flooded by external stimuli.

The essentials oils of the green tea spread a pleasant aroma.
They have the effect that the human feels well and that the muscle tension gets reduced.
Thge effect is similar with a mild narcotica, partially enabling to decouple negative emotions from the nervous system.
The essential oils don't only make the thoughts flow smoother, it has been shown that it leads to a calming of the human motor function.
Researched already described this effect already in the 18$^{th}$ century.
Yet another effect of these essential oils: they put the human in a peculiar slight euphoria.

In the interplay of the different effects of green tea gives it's exceeedingly positive effect.

Green tea is definitely not to be counted in the category of psychotrope\footnote{psychotrop: acting on the psyche, the mind (medicine)}
plant medicine, like for instance St. John's wort or valerian.
These two get used in a targeted way to treat mental and emotional diseases like anxiety and depression.
But as a whole, in the sum of it's pharmaceutical substances, it's taste and preparation methods can green tea have an affect on the psyche.
It can be said, that the psychoactive substances of green tea or so carefully balanced, a pharmacist couldn't do it any better.

Spoil yourself with a good green tea (pay attention to get a good quality) and prepare it in a ritual way.
The water should not be hotter than \SI{80}{\degree\celsius} (\SI{176}{\degree F})\footnote{A convenient way to have \SI{80}{\degree\celsius} water
is to mix one part room temperature water with three parts boiling water, or about one part ice to 6-7 parts boiling water.
Additionally, steep for 2, maximum 3 minutes or the tea tends to get bitter.},
else the precious nutrient are getting destroyed and it tastes bitter\cite{GreenTee}.

\end{document}
