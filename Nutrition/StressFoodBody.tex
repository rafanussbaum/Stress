\documentclass[../main.tex]{subfiles}
\graphicspath{{\subfix{../images/}}}
\begin{document}



In real or imagined danger will the stress hormones immediately mobilize all sugar reserves ---
as if we'd still march into battle with a club or sword.
In that case, the fast supply of of the brain (our most effective weapon against stress!) and our muscles with sugar
is a question of surviving.

It's very fortunate, that the battles are seldom fought with clubs these days, but often in ways not involving blood shed.
This means, that our muscles use less sugar than the stress hormones provide.
On top of that, nowadays we are constantly ``threatened'' everywhere by sugar.
The worst is the increased sugar intake through sweetened drinks and sweets.
Their sugar content is typically very high.
Already decades ago was it sweetened drinks and sweets the essential reasons for weight gain.
The often thoughtless drinking of the colorful sugar water or munching countless sweets disturbs the balance
of stress hormones and sugar balance like no other thing.

\subsection{Fast and Slow Sugars}

As we already know (from chapter~\ref{chap:carbs}: Carbohydrates and Glycemic Index, page~\pageref{chap:carbs}) will the body react
to fast sugars\index{food!sugar!fast} with the immediate production and release of ~\ref{chap:carbs}a big amount of insulin into the blood stream.
Suddenly high insulin levels can lead to the effect, that there's too much sugar in the muscles, more than the muscle needs
given that the physical movement needed to deplete it is missing.
But when that much sugar is mobilized from the liver to be send to the muscles will the blood sugar levels sharply decline.
This in turn can lead to the stress hormones adrenaline and noradrenaline are excreted.

To much sugar can trigger a ``false alarm'' in our body and therefore subject us to stress, which is not at all correlated to any existing threat.
This is a very good example of avoidable stress.

The fast drop of the blood sugar level often leads to ravenous hunger.
Everything within reach is eaten, and the body receives even more sugar, the he doesn't need.
This is also unfortunate for yet another reason, this excessive sugar will be stored as fat in the fat tissue. Nobody wants that.
Yet another important point is that these spikes in insulin levels due to regular consumption of foods with sugar and white flour increase in the
long run the chance to get diabetes type 2.
It used to be called adult--onset diabetes, but nowadays many people already get it as adolescents.

\subsection{Without a Question: Stress leads to Weight Gain}

\begin{itemize}
\item The stress hormone cortisol\index{cortisol} stimulates the appetite.
\item While being afflicted by stress, many people instinctively grab sweets.
  This instinct is right in it's nature: Especially dark chocolate with high cocoa content (\textgreater 70\%) is extremely damping the stress
  (it contains sugar and a lot of magnesium\index{mineral!magnesium}).
  But it is only needed in a situation of acute stress.
\item Many people tend to reward themselves and compensate for stress from the performance at work by eating big portions of food.
  That may be understandable, but it leads to more stress.
  It leads to a fast increase in weight.
  Now on top of the stressful work comes additional stress from the overweight.
  The reward from big portions doesn't last long --- the remorse due to the lack of discipline in turn much longer.
\item The stress spiral starts: stress --- ravenous hunger --- eating --- overweight --- even more stress --- even more ravenous hunger ---
  even more eating --- even more stress.
\end{itemize}

\subsection{Without a question: Stress makes Sick}

Subsiding stress leads to a high increased demand on vitamins, and minerals.
If these don't get supplied, can this lead to a deficiency of:

\begin{itemize}
\item Proteins
\item Vitamins (B1, B2, B5, C, niacin, choline)
\item Minerals  (magnesium, calcium, zinc)
\end{itemize}

\noindent The consequences are

\begin{itemize}
\item Depression\index{symptom!depression}
\item Anxiety\index{symptom!anxiety}
\item Insomnia\index{symptom!insomnia}
\item Amyosthenia, weakness of the muscles\index{symptom!amyosthenia}\index{symptom!muscle weakness}
\item Cardiovascular diseases\index{symptom!cardiovascular diseases}
\end{itemize}


\end{document}
