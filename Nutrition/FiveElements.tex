\documentclass[../main.tex]{subfiles}
\graphicspath{{\subfix{../images/}}}
\begin{document}

Passage from ``5 Elemente K\"uche'' by Ursula Wetter~\cite{5Elemente}.

\subsection{The 5 Temperatures: Yin and Yang}

Foods can be classified by the amount of heat they radiate in the human body.
Foods which are very hot in nature, like pepper, chilies or garlic have the property of heating up.\index{food!heating}
Foods which are in their nature very cold, like cold water algae, dandelion, cucumbers or melons have the property of cooling down.\index{food!cooling}
This classification corresponds to the Yin Tang principle: hot and warm are two aspects of Yang,\index{food!Yang}
cool and cold are the two Yin aspect.\index{food!Yin}
The fifth temperature corresponds to a neutral value.
Given that we only perceive temperatures in contrast to our body temperature of \SI{37}{\celsius} (\SI{98.5}{\degree\F}),
the neutral temperature is still modestly warm for our body.

The temperature of foods and herbs is in the Chinese medicine the basis that foods are not just foods but also medicine.
It acts as medicine by cooling, where's heat and warming, where there's cold.
The knowledge of the thermal action of foods also allows us to combine cooling and cold foods with warming ingredients to bring the dish into a
harmonious equilibrium.
Foods, which are cool or cold in their nature need an equilibration, so that they don't spread cold in the body.
Warming or hot foods are supplemented by cooling ingredients to not produce heat in the body.
This is the art of preparation to bring foods to an harmonic equilibrium.
In summer and when having internal heat, we choose cooling foods and cook in a ``yinizing'' way.
In winter of when we have internal cold or humidity, we cook warming foods and cook in a ``yangizing'' way.

\begin{description}
\item[Cooling to Cold Foods] are rich in water (juicy), grew in cold surroundings or contain substances, which act in a cooling way (organic
  acids, bittering agents, minerals).
  Fruit growing in warmer climates are most often of sour--sweet taste and cooling, what bring in warm climates the needed cooling.
  We stay with our discussion with foods and herbs which provide a balance for moderate climate zones.
\item[Warming and Hot Foods and Herbs] are poor in water (dry), grew in warm environments or contain substances with a hot--sweet taste,
  which act in a warming way.
\end{description}

\noindent By cooking, you can ``yinize'' or ``yangize'' a dish.
The Chinese know two modes of preparation: cooking with fire is ``yangizing'' and cooking with water is ``yinizing''.
With these methods, we can also achieve a balance to the seasonal factors, by treating cooling foods with a ``yangizing'' preparation method
to limit how cool they are.
Let's look at an example: lettuce is counted to the cooling foods, which bring a welcome cooling, especially in summer.
In colder days we can cook with the lettuce, for instance a Asian noodle or rice dish, instead of eating it raw.
Or then tomatoes: instead of eating them raw in a salad, cook a tomato sauce of them or dry them and cook them in a grain or rice dish.
``Yinizing'' and ``yangizing'' means a way of cooking, which is adapted to the seasons, but also balances the thermal effect of foods.

\end{document}
