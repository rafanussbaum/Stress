\documentclass[../main.tex]{subfiles}
\graphicspath{{\subfix{../images/}}}
\begin{document}

Passage from ``5 Elemente K\"uche'' by Ursula Wetter~\cite{5Elemente}.

\subsection{The 5 Temperatures: Yin and Yang}

Foods can be classified by the amount of heat they radiate in the human body.
Foods which are very hot in nature, like pepper, chilies or garlic have the property of heating up.\index{food!heating}
Foods which are in their nature very cold, like cold water algae, dandelion, cucumbers or melons have the property of cooling down.\index{food!cooling}
This classification corresponds to the Yin Tang principle: hot and warm are two aspects of Yang,\index{food!Yang}
cool and cold are the two Yin aspect.\index{food!Yin}
The fifth temperature corresponds to a neutral value.
Given that we only perceive temperatures in contrast to our body temperature of \SI{37}{\celsius} (\SI{98.5}{\degree\F}),
the neutral temperature is still modestly warm for our body.

The temperature of foods and herbs is in the Chinese medicine the basis that foods are not just foods but also medicine.
It acts as medicine by cooling, where's heat and warming, where there's cold.
The knowledge of the thermal action of foods also allows us to combine cooling and cold foods with warming ingredients to bring the dish into a
harmonious equilibrium.
Foods, which are cool or cold in their nature need to be brought into an equilibrium,
so that they don't spread cold in the body.
Warming or hot foods are supplemented by cooling ingredients to not produce heat in the body.
This is the art of preparation to bring foods to an harmonic equilibrium.
In summer and when having internal heat, we choose cooling foods and cook in a ``yinizing'' way.
In winter of when we have internal cold or humidity, we cook warming foods and cook in a ``yangizing'' way.

\begin{description}
\item[Cooling to Cold Foods] are rich in water (juicy), grew in cold surroundings or contain substances, which act in a cooling way (organic
  acids, bittering agents, minerals).
  Fruit growing in warmer climates are most often of sour--sweet taste and cooling, what bring in warm climates the needed cooling.
  We stay with our discussion with foods and herbs which provide a balance for moderate climate zones.
\item[Warming and Hot Foods and Herbs] are poor in water (dry), grew in warm environments or contain substances with a hot--sweet taste,
  which act in a warming way.
\end{description}

\noindent By cooking, you can ``yinize'' or ``yangize'' a dish.
The Chinese know two modes of preparation: cooking with fire is ``yangizing'' and cooking with water is ``yinizing''.
With these methods, we can also achieve a balance to the seasonal factors, by treating cooling foods with a ``yangizing'' preparation method
to limit how cool they are.
Let's look at an example: lettuce is counted as a cooling food, which bring a welcome cooling, especially in summer.
In colder days we can cook with the lettuce, for instance a Asian noodle or rice dish, instead of eating it raw.
Or then tomatoes: instead of eating them raw in a salad, cook a tomato sauce of them or dry them and cook them in a grain or rice dish.
``Yinizing'' and ``yangizing'' means a way of cooking, which is adapted to the seasons, but also balances the thermal effect of foods.

\subsection{Yin and Yang Effects of Foods}
Foods can be classified, depending of their yin or yang effect into categories of ``extreme'' and ``exotic'' to ``mild''.

\begin{description}
\item[Extreme Yins are:] soft drinks, ice cream, everything which is cooled, freezer foods, fridge foods.
\item[Exotic Yins are:] exotic fruits
\item[Native Yins* are:] salad and all vegetables, which don't support frost (root vegetables less so), berries, stone fruit
  and unripe fruits, night shades (tomatoes, cucumber, potatoes, bell pepper, egg plant).
\item[Refined Yins are:] denaturalized foods, like white flour pasta, cake and pie, baked goods, convenience foods.
\item[Cooling Yins are:] yogurt, cream cheese and curd cheese, soy products, milk based drinks, vegetable and fruit juices,
  mineral water, beer, cooling herbal infusions.
\item[Mild Yins are:] grains, legumes, sun ripened local fruit, vegetables (not artificially grown), honey,
  fruit juice syrups, malt, whole milk.
\item[Extreme Yangs are:] alcohol (high percentage/proof), coffee (dark roasts), extremely hot spices, extremely highly heated fats.
\item[Exotic Yangs are:] hot spices (pepper, chili, curry, ginger).
\item[Native* Yangs are:] garlic, onions, leek, cabbage, horse radish, cress, mustard seeds.
\item[Refined Yangs are:] grain based coffee, red wine, mulled wine, dried meats, hard cheese, sausages rich in salt, cured foods.
\item[Warming Yangs are:] meat (game more so than breed animals), warming spices an herbs
  (rosemary, thyme, dill weed, caraway seeds), nuts, seeds, pips, chest nuts, warming vegetables (leek, cabbage, root vegetables).
\item[Mild Yangs are:] Cream, butter, oil, eggs, warm water fish, fresh water fish.
\end{description}
\footnotesize{*This refers to a moderate climate zones}
\normalsize

\subsection{Raw Foods and Warmth/Cold Balance}

Raw foods contain vitamins, which are important to maintain our health.
For a human to stay healthy, he needs the energy to digest them.
Digestion processes are warmth processes.
By consuming too much raw foods, the ``fire of digestion'' gets cooled down.
Extinguished fire doesn't produce heat anymore.
A gentle cooking means to break open the plant fibers to dissolve out the important ingredients (nutrients).
The Chinese way of cooking means to stimulate the ``internal cooking process'' through it's own heat.

People with a pronounced Yang--nature (a lot of heat), supports more raw foods  and people with a pronounced Yin-nature
(a lot of cold), is able to process less raw foods.
Men are by nature more Yang oriented, women more Yin oriented.
A woman (Yin) who's prone to easily feel cold (Yin--symptom) who eats a lot of salad (Yin food) with
cooling dressings with yogurt, lemon juice and salt (Yin ingredients), drinks mineral or iced tea with that (Yin drinks)
and as a dessert enjoys an ice cream (Yin luxury food item), then not much ``fire'' will be left.


According to Chinese nutrition science, when it's warm or in summer can the ratio of raw foods be higher than in winter or in cold times.
Raw vegetables and fruit are the counter balance to the heat and dryness of summer.
In winter, in the cold, fruits are being eaten as dried fruit or cooked.
The same is true for vegetables and salads: winter salads are accompanied by a warming, reasonable spicy dressing.
Summer vegetables and salad, which aren't able to cope with frost, we eat in warm times as raw foods.

The main argument for raw foods of Western nutritional science is the supply of vitamins for our health.
The Chinese nutritional science, which doesn't focus on vitamins but on Yin and Yang, goes from the premise
that a human, in order to stay healthy, needs sufficient energy in his digestive organs in order to produce
the physiological Yin and Yang in our body.
The choice of foods inhibits or promotes digestion.
If there's a sufficient amount of  energy in our system, then Yin and Yang are properly converted.
Foods, which are very cold and lower the energy (Yin) block the energy.
Foods, which are thermally hot and energetic (Yang) have as an effect that the energy and with it important
essences are lost, like for instance water (through sweating).

The Chinese way of cooking ``rich in vitamins'' (alive), consists in adding ingredients which are full of enzymes,
like soy sauce or Miso at the end of preparation, to keep it active.
Soy sauce is a spice, which was discovered pretty much by accident by Buddhist monks is a spicy juice made from say and wheat
full of important nutrients and enzymes.
The soy beans get steamed first, the grains of wheat roasted and then ground up.
It is then mixed with water and salt.
In a fermentation and long ripening process (brewed like a good beer, in a process kept secret) we get a black juice full
of enzymes, which is mostly used a condiment for rice.

Home made soy and other sprouts are also very rich in enzymes and an alive food, which can be added to soups, salads and other dishes.
The roasting of grains (it becomes warming, where in itself it would be cooling) as many traditional cultures still practice,
unlocks the proteins and also has a big influence on taste and how well our body supports the food.
To do so, are the grains roasted in a pan without adding fats and constantly stirred with a wooden spoon, that the grains don't get burned.
Afterwards it gets cooked --- with refreshing ingredients in summer and warming ingredients in winter.
Grain flakes are getting heated in vapor at \SI{85}{\celsius} (\SI{185}{\degree\F}), which makes them not a raw food anymore, strictly speaking.

\textbf{Raw grains contain phytic acid\index{anti--nutrient!phytic acid}, which inhibits potassium intake.
  Out of that reason can the regular consumption of raw cereals lead to an insufficiency in potassium\index{symptom!insufficiency, potassium}.
  By cooking or sprouting the grains will the phytic acid be almost completely eliminated.}

In the TCM\index{nutrition!TCM} (Traditional Chinese Medicine) nutrition, soups hold a big signification.
The soup is the preparation of the stomach for everything which follows: It delivers the salts
for the hydrochloric acid in our stomach and the warmth needed to relax and activate the digestive organs.

\subfile{TableTemperature.tex}

\subsection{The Five Tastes}

Each food can be categorized into five groups.
The foods are classified in that manner, because every group of foods (or taste) acts on the organs they are are attributed to according to the theory of the five elements.

So for instance bitter tasting foods will be rerouted into the heart.
So bitter foods nourish the heart.
The nutrients and vital energy that a bitter food contains allow the heart to build up more tissue and to strengthen it and therefore improves it's function.
When you ever had 2--3 cups of strong coffee in short succession, you have experienced how your heart has been beating like crazy.
Coffee is bitter and therefore acts immediately on the heart.
On the other side coffee is barely supporting the heart with nutrients, it only causes it to beat faster.

Or in other words: spicy and hot foods are nutrition for the lungs and large intestine.
For instance cinnamon, paprika, pepper and curry are healing in case of a cold and coughing.
And for sure, these spices are the base of many house remedies against these sicknesses.

Sour foods are good for the liver.

The taste of sweet is good for the pancreas, like for instance runner beans.

Salt and salty is attributed to the kidney. Salt stimulates the kidney function.
Too much salt on the other hand, weakens and overwhelms them.
A weakened kidney, like for instance with people who suffer from edema, has to be relieved.
It doesn't need any additional stimulation like diuretics would provide them.
For a balanced diet, these five tastes should be composed the following way:

\begin{table}[htb!]
  \begin{center}
  \caption{A balanced diet and the five tastes.}
\begin{tabular}{lll}
  \textbf{Sweet} & 20\% & Pancreas \\
  \textbf{Sour} & 20\% & Liver \\
  \textbf{Bitter} & 20\% & Heart \\
  \textbf{Spicy, hot} & 20\% & Lungs \\
  \textbf{Salty} & 20\% & Kidneys \\
\end{tabular}
  \end{center}
\end{table}

The following table might be a  bit confusing.
Even if you think a food is in the wrong category: the food is classified correctly,
but it only changes to the corresponding taste inside the human organism.

\begin{description}
\item[Sweet:]
almonds, baked goods, bran, cabbage, cake, carrots, coconut, corn, cucumber, dried fruits, drinks (alcohol free), egg plant, fruit juice (freshly pressed), fruits, grains, green beans, honey, ice cream, milk, milk based drinks, para nuts, peanuts (fresh), peas, pecan nuts, pie, potatoes, pudding rice, pumpkin, red beets, romaine lettuce, sorghum, squash, sugary foods, sunflower seeds, syrup, walnuts, wheat. 
\item[Sour:]
beef, all sorts of meat products, bread, butter milk, chicken, fresh water fish, fruit juices, green cabbage, liver, mayonnaise, meat (red), rose hip, salad dressing, salami, all types of sausages, sour cream, sprouts, tomatoes, wine based vinegar, yeast, yogurt.
\item[Bitter:]
artichoke, asparagus, avocado, bamboo sprouts, black tea, broccoli, carrots, cauliflower, celeriac (celery roots), celery, chard, chocolate, cocoa, coffee, cress, endives, gelatin, leafy greens, leek, mushrooms, mustard, spirulina, vegetables,
\item[Hot/spicy:]
anise, basil, cayenne pepper, chili, curry, dill weed, garlic, ginger, mint, mustard seeds, onions, oregano, paprika, parsley, rhubarbs, thyme, truffle, vanilla, wine. 
\item[Salty:]
algae, bones, butter, caviar, cheese, eggs, margarine, olives (processed), salt, shellfish, soy sauce, tofu.
\end{description}

\end{document}
