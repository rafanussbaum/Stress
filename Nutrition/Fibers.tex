\documentclass[../main.tex]{subfiles}
\graphicspath{{\subfix{../images/}}}
\begin{document}

The term dietary fibers\label{chap:fibers}\index{fibers, dietary} encompasses different indigestible parts of plant based foods.
Fibers don't deliver energy to the body, so they used to be looked at as useless ballast and of little regarded in nutrition in the past.

\begin{table}[htb!]
  \centering
  \begin{tabular}{ll}
    \textbf{Dietary fiber} & \textbf{source} \\
    \hline
    Cellulose & fruits, vegetables, grains \\
    Hemicellulose & legumes, whole wheat and grains \\
    Lignin & fruit pits, vegetables (strings in beans), grains \\
    Pectin & fruits, vegetables (especially in apples and quinces \\
  \end{tabular}
  \caption[Dietary fibers]{The most important types of dietary fibers and where they appear}
\end{table}

Most fibers are very complex carbohydrates, which can't be cleaved into usable sugars by our digestive system.
They pass the digestive tract practically unchanged and leave the body with the stool.
These days we know that fibers fill important functions in the body and can't be left out of a healthy nutrition.

\subsection{Functions of Fibers}
\begin{itemize}
\item they inspire to chew vigorously --- we eat slower. Eating slower leads to feelings of being satiation earlier.
  Carbohydrates form fiber rich foods get slower into the bloodstream -- the satiation holds on for a longer time.
\item they make the excrenents bigger in volume and softer, they instigate the digestion and therefore lower the risk to be affected
  by constipation, hemorrhoids or diverticulitis (small protuberance in the colon).
\item they take in toxic and carcinogenic substances (eg. bile), transporting them out of the body and by doing so lower the risk of colon cancer.
  \item they also transport small amounts of cholesterol away form the body and help to lower the cholesterol levels.
  \end{itemize}

  %Gut biome importance, 99% DNA not human, 90%cells not human on human, linked to spectrum ancd ADHD, irritable bowel syndrome


  \subsection{How Much Fibers Does a Human Being Need?}

  Too much or too little fibers can be danger to your health.
  Too much fiber can lead to flatulence and problems of the colon
  and also inhibit the intake of minerals like zinc and potassium from the food.
  That's why it's not recommended to supplement the intake of fibers (eg. bran).

  The recommendation of fibers (DACH reference values 2000) is 30~g/day (bit more than an ounce a day).
  This amount can easily be obtained by substituting refined grain product by whole grain products,
  put once or twice a week legumes on the menu and by eating daily different veggies and fruits.

  
\vspace{5mm}
\noindent
\begin{fminipage}{\textwidth}
  \textbf{Profile Fibers}
  \begin{itemize}
  \item They are important to regulate the digestion.
    They make the excrement soft and prevent constipation and colon diseases.
  \item They transport toxic and carcinogenic substances out of the body and lower the risk of colon cancer.
    \item Absorb small amounts of cholesterol and help lowering the cholesterol levels.
    \item We should eat a daily amount of 30 g (1 oz) of fibers a day ---
      50\% from whole grain and legumes and 50\% from fruits and veggie.
      \item Fibers bind water and swell up in the colon, so it's important to drink sufficiently.
  \end{itemize}
\end{fminipage}

% Add about microbiome here

\end{document}