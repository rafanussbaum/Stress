\documentclass[../main.tex]{subfiles}
\graphicspath{{\subfix{../images/}}}
\begin{document}

The GLYX value always refers to the amount of a food that contains \SI{50}{\g} of carbohydrates.
Therefore, the GLYX alone has a limited usefulness.

The glycemic index has been coined in research about diabetes and is in this context very important and valuable.
But it helps healthy people only in a limited way   to choose the right foods.
In the end, not only the increase in the blood sugar levels alone decides about being slim or overweight,
but also the energy ratio between incoming and outgoing energy,
In order to take this into account, the glycemic load (GL)\index{GL}\index{glycemic load} has been developed.
The glycemic load considers  the GLYX value as well as the amount of carbohydrates taken in an a serving.
To calculate the glycemic load, the glycemic index (GI) is multiplied with the amounts of carbohydrates of a serving.

The lower the glycemic load of a food, the lower is this food's influence on the blood sugar level and
therefore the excretion of insulin is also lower.
Generally speaking, for a healthy nutrition we should if possible carbohydrate sources with a low glycemic load (GL).

Formula to calculate the glycemic load (GL):

\begin{center}
  Glycemic Load (GL) = \sfrac{Glycemic Index (GI)}{100} * CH (Carbohydrates per 100 g)
  \end{center}

According to their GL, foods are categorized into the following categories:

\begin{center}
  \begin{tabular}{cl}
    \textbf{Glycemic Load (GL)} & \textbf{Category} \\
    \hline
    \textless 10 & light weights = good \\
    10 -- 20 & middle weights = medium \\
    \textgreater 20 & heavy weights = bad \\
  \end{tabular}
\end{center}

\section{Therefore, Carrots aren't That Bad?}

The following example shows how a food which is a ``heavy weight'' according to the GLYX values, can become a light weight when we respect the glycemic load GL.

\begin{tabular}{llll}
  \textbf{Carrots, cooked} & Glycemic Index (GI) & = 85 & ``Heavy weight'' \\
  & Carbohydrates / \SI{100}{\g} & = 7 \\

  \multicolumn{3}{l}{GL cooked carrots = \sfrac{85}{100} * 7 = 5.95} & ``Light weight'' \\
\end{tabular}

As a comparison we take ``toast bread, white'' with the same glycemic I=index as the cooked carrots.
Taking the glycemic load into account, it stays a heavy weight.


\begin{tabular}{llll}
  \textbf{Toast, white} & Glycemic Index (GI) & = 85 & ``Heavy weight'' \\
  & Carbohydrates / \SI{100}{\g} & = 50 \\

  \multicolumn{3}{l}{GL toast, white = \sfrac{85}{100} * 50 = 42.5} & ``Heavy weight'' \\
\end{tabular}


\vspace{5mm}
\noindent
\begin{fminipage}{\textwidth}
  \textbf{Profile Glycemic Index (GI) / Glycemic Load (GL)}
  \begin{itemize}
  \item The glycemic index or GLYX only helps so far for choosing the sources of carbohydrates.
  \item A high GLYX value does not necessarily mean a high glycemic load GL.
  \item While choosing sources of carbohydrates, foods with low GL should be preferred.
    \item At the end of the day, the energy balance between energy intake to energy used is decisive about gain or loss of weight.
  \end{itemize}
\end{fminipage}

\end{document}
