\documentclass[../main.tex]{subfiles}
\graphicspath{{\subfix{../images/}}}
\begin{document}

The GLYX value always refers to the amount of a food that contains \SI{50}{\g} of carbohydrates.
Therefore, the GLYX alone has a limited usefulness.

The glycemic index has been coined in research about diabetes and is in this context very important and valuable.
But it helps healthy people only in a limited way   to choose the right foods.
In the end, not only the increse in the blood sugar levels alone decides about being slim or overweight,
but also the energy ratio between incoming and outgoing energy,
In order to take this into account, the glycemic load (GL)\index{GL}\index{glycemic load} has been developed.
The glycemic load considers  the GLYX value as wel as the amount of carbohydrates taken in an a serving.
To calculate the glycemic load, the glycemic index (GI) is multiplied with the amounts of cabohydrates of a serving.

The lower the glycemic load of a food, the lower is this food's influence on the blood sugar level and
therefore the excretion of insulin is also lower.
Generally speaking, for a healthy nutrition we should if possible carbhydrate sources with a low glycemic load (GL).

Formula to calculate the glycemic load (GL):

\begin{center}
  Glycemic Load (GL) = \sfrac{Glycemic Index (GI)}{100} * CH (Carbohydrates per 100 g)
  \end{center}

According to their GL, foods are categorized into the following categories:

\begin{center}
  \begin{tabular}{cl}
    \textbf{Glycemic Load (GL)} & \textbf{Category} \\
    \hline
    \textless 10 & light weights = good \\
    10 -- 20 & middle weights = medium \\
    \textgreater 20 & heavy weights = bad \\
  \end{tabular}
\end{center}

\section{Therefore, carrots aren't that bad?}



\end{document}
