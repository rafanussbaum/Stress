\documentclass[../main.tex]{subfiles}
\graphicspath{{\subfix{../images/}}}
\begin{document}

As the term says, are dietary supplements\index{dietary supplement}\index{supplement!dietary} things which supplement our nutrition,
respectively supply the body with additional vital substances.
The goal of that is to prevent a potential lack.
Dietary supplements can contain vitamins, minerals, trace elements, enzymes, secondary plant nutrients,
essential fatty and amino acids or combinations thereof.
Dietary supplements help to optimally support the body with vital substances.

\subsection{Why are Dietary Supplements Important?}
A balanced diet is an important prerequisite, but no warranty that the body is supplied with all 50
essential vital or micro nutrients required to maintain the body's health.
More and more people suffer from a lack of important ital substances.
The following reasons lead to this fact:

\renewcommand{\labelitemii}{$\star$}
\begin{itemize}
\item Nowadays, daily life requires less physical strenuous activities.
  That reduces the amount of energy required and therefore the amount of daily food.
  With this reduction of food intake, the intake of vital substances is also lowered.
\item Heightened needs of vital substances by modern human beings due to inner and outer strain, like:
  \begin{itemize}
  \item Physical and psychological stress
  \item Pregnancy, breast feeding, menopause
  \item Consumption of alcohol, caffeine, or smoking
  \item Pollution of the environment, electro smog
  \item Diseases or being on diet
    \item Consumption of medication or hormone based medicine (ie. birth control)
    \end{itemize}
  \item The more and more intensely farmed soils leeches nutrients from the soil,
    additionally will the acid rain complex trace minerals in the soils.
    This leads to the plants contain less minerals and vitamins.
  \item Foods also contain less vital substances through:
    \begin{itemize}
    \item the long supply chains and storage
    \item the industrial treatment
    \item irradiating and heat treatment to conserve foods
    \item the adding of artificial colors, flavoring agents and other additives
      \item cooking foods too long or usage of the micro wave oven
    \end{itemize}
  \end{itemize}

  \begin{table}[htb!]
    \centering
    \begin{tabular}{l|ccc|ccc|ccc}
      &\multicolumn{3}{c}{\textbf{Calcium}} & \multicolumn{3}{c}{\textbf{Magnesium}} & \multicolumn{3}{c}{\textbf{Vitamin C}} \\
      Food & 1985 & 1996 & diff. & 1985 & 1996 & diff. & 1985 & 1996 & diff. \\
      & & & \%  & & & \%  & & & \% \\
      \hline
      Broccoli & 103 & 33 & \textbf{-68} & 24 & 18 & \textbf{-25} \\
      Green Beans & 56 & 34 & -38 & 26 & 22 & -15 \\
      Fennel & 35 & 57 & +62 & 11 & 17 & +45 \\
      Carrots & 37 & 31 & -17 & 21 & 9 & -57 \\
      Potatoes & 14 & 4 & \textbf{-70} & 27 & 18 & \textbf{-33} & 20 & 25 & \textbf{+25} \\
      Apples & 7 & 8 & \textbf{+12} & 5 & 6 & \textbf{+20} & 5 & 1 & \textbf{-80} \\
      Bananas & 8 & 7 & -12 & 31 & 27 & -13 \\
      Strawberries & 21 & 18 & -14 & 12 & 13 & +8 & 60 & 13 & -67 \\
      Spinach & & & & 62 & 19 & \textbf{-68} & 51 & 21 & \textbf{-58}\\
      \hline
      \multicolumn{10}{l}{\footnotesize{Random samples from 1996 were being compared to the table with nutrient contents of the}}\\
      \multicolumn{10}{l}{\footnotesize{Swiss pharmaceutical company Geigy from 1985}}
    \end{tabular}
    \caption[Mineral content of food now and earlier]{Content of of some vitamins, trace minerals and minerals over time}
  \end{table}

  \subsection{Dietary Supplements for Stress, Tiredness and Exhaustion}

  The modern diet often contains a lot of calories, refined carbohydrates, salt and saturated fatty acids but
  is deficient in complex carbohydrates, which are mostly contained in fresh fruit and vegetables.
  This diet often causes chronic deficiency in different micro nutrients
  --- the B--vitamins\index{vitamin!B}, magnesium\index{mineral!magnesium}, \index{mineral!iron}iron, zinc\index{mineral!zinc} ---
  which are important for the energy regulation and to fight fatigue\index{symptom!fatigue}.
  In phases with a lot of work and stress, the demand of micro nutrients is significantly higher.
  By preferring whole grain products, protein sources low in fats and fresh fruits and vegetables the body is well supplied
  with B--vitamins and the minerals needed to fight against exhaustion\index{symptom!exhaustion}.

  People tend to make the mistake to trust on coffee\index{coffee} and sugar in stress situations.
  Even though they do right away deliver a lot of energy, caffeine\index{caffeine} and refined sugars\index{sugars!refined}
  can worsen chronic exhaustion
  and cause head aches\index{symptom!head ache}, irritability\index{symptom!irritability}
  and problems focusing\index{symptom!focusing problems}.
  The regulation of glucose in the blood stream is more difficult in times of stress,
  so the intake of refined sugars is to be avoided, because it can lead to strongly fluctuating blood sugar levels
  and to states to a lack of sugar (so called reactive hypoglycemia\index{symptom!hypoglycemia, reactive}).

  \begin{table}[htb!]
    \centering
    \begin{tabular}{p{2cm}p{4cm}p{5.5cm}}
      \textbf{Nutrient} &\textbf{RDA} & \textbf{Comment} \\
      \hline
      Multivitamin Supplement & Balanced formula with 5--\SI{10}{\milli\gram} iron\index{mineral!iron}, 10--\SI{20}{\milli\gram} zinc\index{mineral!zinc},
                                \SI{200}{\micro\gram} selenium\index{mineral!selenium}
                                      & Iron and zinc deficiency can cause chronic fatigue, respectively exhaustion. Heightened need of zinc and selenium during stress \\                                                                        Vitamin B Complex & Complete formula with \SI{25}{\milli\gram} vitamin B$_1$\index{vitamin!B1},  B$_2$\index{vitamin!B2},  B$_6$\index{vitamin!B6}
                                                                                                                                                                                                                                                                      (min. \SI{100}{\milli\gram}),
                                                                                                                                                                                                                                                                      niacin\index{micro nutrient!niacin} and panthothenic acid\index{micro nutrient!panthothenic acid}
                                                                                                                                                                                                                                                                      and \SI{0.8}{\milli\gram} folic acid\index{micro nutrient!folic acid}
                                      & Due to their central role in the metabolism, lack of B--vitamins can cause fatigue and exhaustion.
                                        In times of stress, heightened  activity and increased energy demand the required amount of B--vitamins also increases \\
      Vitamin C\index{vitamin!C} & 0.5 -- \SI{2}{\gram} & required amount increased during stress \\
      Vitamin B$_{12}$\index{vitamin!B12} & 25--\SI{50}{\micro\gram} (for people with resorption problems: \SI{1}{\milli\gram}; intramuscular application)
                                      & A deficiency of Vitamin B$_{12}$ can lead to anemia, fatigue and depression; common in elderly people \\
      L--Tryptophan\index{micro nutrient!L--tryptophan} & 1--\SI{2}{\gram} & Regulates sleep \\
      Magnesium\index{mineral!magnesium} & 400--\SI{600}{\milli\gram} & relaxes \\
      Coenzyme Q10\index{micro nutrient!Coenzyme Q10} & 200--\SI{300}{\milli\gram} & Antioxidant; regulates heart and circulation functions; energy regulation \\
      Amino acids (Cysteine\index{micro nutrient!cysteine}, Glutamine\index{micro nutrient!glutamine}, Glycine\index{micro nutrient!glycine})
                        & & heightened need during stress (glutatione) \\
    \end{tabular}
    \caption[RDA of micro nutrients in stress]{RDA of micro nutrients in stress, translated from  Burgerstein's handbook nutrients~\cite{BurgerNutrient}}
  \end{table}

  \subsection{Dietary Supplements for Sleeping Troubles}
  The reason for sleeping troubles\index{symptom!sleeping troubles} is often the diet.
When the micro nutrient metabolism gets disturbed, nerve cells are getting insufficiently supplied.
An optimal diet can increase the quality of the sleep.
People suffering from insomnia\index{symptom!insomnia} should take multivitamin and mineral supplements,
insomnia increases the demands of nutrients.
Dinner should be taken at least three hours before going to bed.
Foods with a good ratio of tryptophan to proteins should be preferred.
Tryptophan\index{tryptophan} is an amino acid, which is needed to build up serotonin
(a neurotransmitter, which  induces sleep).
A light dinner, rich in tryptophan can increase the quality of the sleep.


\vspace{5mm}
\noindent
\begin{center}
\begin{fminipage}{0.45\textwidth}
  \textbf{Foods with a big tryptophan/protein ratio\label{box:tryptophan}}
  \begin{itemize}
  \item Bananas
  \item Eggs
  \item Fish
  \item Milk and milk products
  \item Soy beans and soy products
    \item Walnuts
  \end{itemize}
\end{fminipage}
\end{center}

Even though alcohol\index{alcohol} has a relaxing effect and might facilitate falling asleep,
it has an effect of causing a light, restless sleep and waking up at night.
A cup of herbal tea, for instance melissa (balm), verbena or orange blossom are much more suited as a good night drink.
A high caffeine\index{caffeine} consumption heightens the risk of developing sleeping troubles.
With difficulties sleeping, the consumption of coffee, black tea or Coca--Cola drinks should be reduced to zero or at least
being reduce to a bare minimum.


  \begin{table}[htb!]
    \centering
    \begin{tabular}{p{2cm}p{4cm}p{5.5cm}}
      \textbf{Nutrient} &\textbf{RDA} & \textbf{Comment} \\
      \hline
      Vitamin C\index{vitamin!C} & 0.5--\SI{1}{\gram}, 30--60 minutes before going to sleep
                                      & helps to balance the sleeping rhythm. Heads up: some people get stimulated by vitamin C \\
      Niacin amide\index{micro nutrient!niacin amide} & \SI{1}{\gram}, 30 minutes before going to sleep & Helps falling asleep and can improve the quality of the sleep\\
      Calcium\index{micro nutrient!calcium} & \SI{600}{\milli\gram}, 30 minutes before going to sleep & Has calming effects  and can improve the quality of the sleep\\
      \textbf{Magnesium}\index{micro nutrient!magnesium}& \SI{800}{\milli\gram} magnesium orotate, 30--60 minutes before going to sleep  & Helps falling asleep and can improve the quality of the sleep\\
      \textbf{Melatonin}\index{micro nutrient!melatonin}& 0.3--\SI{3}{\milli\gram}, 30--60 minutes before going to sleep  & Especially effective in people over 50 years, suffering from chronic sleeping problems \\
       \textbf{Tryptophan}\index{micro nutrient!tryptophan}& 1--\SI{3}{\gram}, 30 minutes before going to sleep  & Helps balancing the sleep rhythm. Also use foods who have a high ratio of tryptophan compared to other amino acids (see text box above, p~\pageref{box:tryptophan}). \\
      Inositol\index{micro nutrient!inositol} & \SI{1}{\gram}, 30 minutes before going to sleep  & Balances the sleep rhythm \\
      Glycine\index{micro nutrient!glycine} & \SI{3}{\gram}, 60 minutes before going to sleep  & Difficulties falling asleep and waking up at night; during sleep, glycine concentration in the pineal gland increases \\
    \end{tabular}
    \caption[RDA of micro nutrients in for sleeping difficulties]
    {RDA of micro nutrients for sleeping difficulties, translated from  Burgerstein's handbook nutrients~\cite{BurgerNutrient}}
  \end{table}


\end{document}