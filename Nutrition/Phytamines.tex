\documentclass[../main.tex]{subfiles}
\graphicspath{{\subfix{../images/}}}
\begin{document}

The term ``secondary plant nutrients'' encompasses thousands of substances from herbs, fruits, vegetables,
potatoes, legumes and whole grain products which are not part of the primary plant ingredients (proteins, fat, carbohydrates).
Seondary plant nutrients are imoortnat for the plants, for instance as defensive mesures against pests,
to protect from UV radiation or to lure in polinators.


Based on their their chemical and functional properties, secondary plant nutrients can be classified into different categories.
Some of these categores are:
\begin{itemize}
\item carotinoids
\item polyphenoles (eg. flavonoids)
\item phytosterines
\item saponins
\item glucosynolates
\item ptotease inhibitors
\item sulfide
\item phytine
  \item phytohormones (eg. phytoestrogene)
  \end{itemize}

  It's known today, that secondary plant nutrients are also important for humans.
  However, there's not enough information an knowledge to to give some actual recommnded daily amounts.

  One thing is for sure: a lot of vegetables and fruits, partially uncooked, once in a while an espresso, occasionally a glass of red wine
  or a piece of dark chocolate (with a cocoa content $<$70\%) increases the intake of secondary plant nutrients
  and therefore adds it's own part to promote health.
  The secondary plant based nutrient are said to have the following health promoting effects:

  \noindent
  \begin{table}[htb!]
    \centering
    \begin{tabular}{p{3.5cm}p{2.5cm}p{5.5cm}}
      \textbf{Effect} & \textbf{Secondary plant nutrient} & \textbf{Contained in} \\
      \hline
      antioxidative\index{effect!antioxidative} (protects cells from harmful substances) & flavanoids
                                                          & fruits, lemons, onions, cellery, cocoa, green tea, red wine, ginko, soy beans \\
      antimicrobial\index{effect!antimicrobial} (protects form bacteria and fungi) & phenolic acid & fruits \\
      anticancerogenic\index{effect!anticancerogenic} (inhibits cancer) & caritinoids
                                                          & red, yellow and orange fruits and veggies, like apricots and carrots, green vegetables like salad or spinach \\
                      & protease inhibitors & potatoes, nuts, grains legumes \\
      antinflamatory\index{effect!antinflamatory} (inhibits inflamations) & saponins & legumes, oats \\
      blood pressure reducing\index{effect!blood pressure, reduce} & polyphenols & pomegranate \\
      digestion promoting\index{effect!digestion, promote} & polyphenols & spices\\
     lower cholesterol level\index{effect!cholesterol level, lower} & phytosterines & sun flower seeds, wheat germs, sesame, soy beans, plant oils \\
      avoid thrombosis\index{effect!thrombosis, avoid} & sulfides & garlic and other plants in the \\      onion family \\
      regulate blood sugar levels\index{effect!blood sugar level!regulate} & phytinic acid & legumes, oily seeds, grains \\
      hormonlike effect\index{effect!hormonlike effect} & phytohormones & hops, red clover, yams, soy beans (phytoestrogens)\\
      \end{tabular}
    \caption[Secondary plant nutrients]{Secondary plant nutrients and their attributed effects}
  \end{table}
\end{document} 