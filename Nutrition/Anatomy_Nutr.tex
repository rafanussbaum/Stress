\documentclass[../main.tex]{subfiles}
\graphicspath{{\subfix{../images/}}}
\begin{document}

(Translated passage from Anatomie--Atlas~\cite{AAtlas})
\section{Voyage Through the Body}
Olfactory cells detect the roast, they locate toast compounds, the marjoram and so on.
They stimulate the flow of the juices into the mouth, get you excited for the first bite.
In the hypothalamus\index{hypothalamus} in the brain is the seat of the central of the appetite, which directs all the processes.
It gets information from the body about how high the blood sugar levels are, from the fat cell if they need replenishment.
That is the seat of the somatic intelligence (body--centered intelligence)\index{somatic intelligence}.
The ancient knowledge of the body, which knows what the body needs.
Unfortunately, that part of us is asleep in most of us.
It should coordinate hunger, cravings and satiation; we move a lot, we need a lot of calories.
The hypothalamus directs a whole army of hunger--saturation hormones\footnote{Almost weekly science discovers a new one.}.\index{hormones!hunger--satiation}
Either it's 'eat more', or if we have a sedentary period it switches to the energy save program --- satiation sets in earlier.
This ancient wisdom urges us with craving of lemons, if the immune system needs a vitamin C boost or a craving for cheese if the bones needs calcium.
The problem is that we suppressed this ancient wisdom because we're not eating anymore due to being hungry but due to what time dictates.
It is suppressed because we're not eating anymore what he body needs but empty calories, jazzed up with artificial aromas.
The head says, 'I want apple pie'. But that's not what the body says, the apple pie isn't good for it. It'd prefer the apple by itself.
That's what in the genetic program, that what fits with the metabolism.

The hypothalamus in the brain, the central of appetite registers how much fat is in the fat cells and how much sugar in the blood.
If it finds something missing, it triggers hunger.

\section{The Miracle of the Digestion}

You eat pasta with veggies and shrimp.
A chemist would say: carbohydrates\index{carbohydrates}, fibers, vitamins, minerals, fat and proteins\index{proteins}.
Already in the mouth (see teeth, picture %\ref{}
on page %\pageref{}
)\index{digestion}
will the saliva (salivary gland, picture %\ref{} page \pageref{}
) cleave carbohydrates, with it's enzymes.
The longer you chew it, the sweeter will the pasta be.
The long carbohydrate chains of the starch gets cleaved into small sugar molecules\index{sugar}.
Chewing grinds up the food, and increases the surface area and therefore the exposure to the digestive tools of the body. 
In the second digestive cavity, the stomach (figure %\ref{}
), hydrochloric acid - chlorine and hydrogen ions disintegrate the past, shrimps and vegetables.
They open the cells of the plants, that the vitamins get released.
The stomach contracts, mixing the whole thing up.

Around 50 tons of food and drinks pass through the stomach in a lifetime.
The enzymes of the stomach are doing hard work.
They split up long protein chains (for instance from the shrimps) into shorter peptides\index{peptides} and into the smallest components, the amino acids\index{amino acids}.
\end{document}