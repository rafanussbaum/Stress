\documentclass[../main.tex]{subfiles}
\graphicspath{{\subfix{../images/}}}
\begin{document}

Our nutrition is composed of multiple basic building blocks, which each in turn have very different
tasks to perform in our body.

the beginning and end of a healthy nutrition is to combine the basic building blocks in a sensible way and it helps us
to maintain our body to the be lean, healthy and full of power into a high age.

These basic building blocks of our food are:
\begin{itemize}
\item Water
\item Proteins
\item Fats (lipids)
\item Carbohydrates (sugar, starch)
\item Vital substances
\item Fibers
\end{itemize}

\begin{table}[htb!]
  \centering
  \begin{tabular}{lcccc}
    \textbf{Building Block} & \textbf{kJ/g} & \textbf{kcal/g} & \textbf{kJ/oz} & \textbf{kcal/oz} \\
    Proteins & 17 & 4 & 482 & 113 \\
    Fats & 37 & 9 & 1049 & 255 \\
    Carbohydrates & 17 & 4 & 482 & 113 \\
  \end{tabular}
  \caption{Energy content of the different building blocks of our food}
\end{table}

\section{Water}

\subfile{Water.tex}

\section{Proteins}

\subfile{Proteins.tex}

\section{Fats (Lipids)}

\subfile{Lipids.tex}

\section{Carbohydrates (Saccharides)}

\subfile{Saccharides.tex}

\section{Vitamins}

\subfile{Vitamins.tex}

\section{Minerals}

\subfile{Minerals.tex}
 
\section{Enzymes}

\subfile{Enzymes.tex}

\section{Secondary Plant Nutrients (Phytamines)}

\subfile{Phytamines.tex}

\section{Dietary Fibers}

\subfile{Fibers.tex}

\end{document}