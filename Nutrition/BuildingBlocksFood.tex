\documentclass[../main.tex]{subfiles}
\graphicspath{{\subfix{../images/}}}
\begin{document}

Our nutrition is composed of multiple basic building blocks, which each in turn have very different
tasks to perform in our body.

the beginning and end of a healthy nutrition is to combine the basic building blocks in a sensible way and it helps us
to maintain our body to the be lean, healthy and full of power into a high age.

These basic building blocks of our food are:
\begin{itemize}
\item Water
\item Proteins
\item Fats (lipids)
\item Carbohydrates (sugar, starch)
\item Vital substances
\item Fibers
\end{itemize}

\begin{table}[htb!]
  \centering
  \begin{tabular}{lcccc}
    \textbf{Building Block} & \textbf{kJ/g} & \textbf{kcal/g} & \textbf{kJ/oz} & \textbf{kcal/oz} \\
    Proteins & 17 & 4 & 482 & 113 \\
    Fats & 37 & 9 & 1049 & 255 \\
    Carbohydrates & 17 & 4 & 482 & 113 \\
  \end{tabular}
  \caption{Energy content of the different building blocks of our food}
\end{table}

\section{Water}

Water doesn't contain any nutrients, but nevertheless is water of the biggest importance for the human body.
Without water, there's no life.
The human body consists our of about 60\% water (a newborn even up to 75\%).
Roughly 60\% of that water is found in the intracellular space (inside the cells).
The rest is in between the cells in the form of interstitial fluids (between the cells),
like blood plasma, lymph, urine, saliva, digestion fluids, tear liquid, nose secretes and sweat.


In the human organism, water mainly serves:
\begin{itemize}
\item to cover the needs of fluids in cells and tissues,
\item as a solvent and medium of transportation for nutrients, encymes, hormones and so on,
\item to excrete metabolism end products over the urine,
\item to regulate the body temperature.
\end{itemize}

\subsection{How much Water does a Human Being Need?}

During a day, the human body looses 2--2.5 liters (8\textonehalf\ -- 10\textonehalf\ cups) of water, of which:


\end{document}