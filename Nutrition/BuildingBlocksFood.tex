\documentclass[../main.tex]{subfiles}
\graphicspath{{\subfix{../images/}}}
\begin{document}

Our nutrition is composed of multiple basic building blocks, which each in turn have very different
tasks to perform in our body.

the beginning and end of a healthy nutrition is to combine the basic building blocks in a sensible way and it helps us
to maintain our body to the be lean, healthy and full of power into a high age.

These basic building blocks of our food are:
\begin{itemize}
\item Water
\item Proteins
\item Fats (lipids)
\item Carbohydrates (sugar, starch)
\item Vital substances
\item Fibers
\end{itemize}

\begin{table}[htb!]
  \centering
  \begin{tabular}{lcccc}
    \textbf{Building Block} & \textbf{kJ/g} & \textbf{kcal/g} & \textbf{kJ/oz} & \textbf{kcal/oz} \\
    Proteins & 17 & 4 & 482 & 113 \\
    Fats & 37 & 9 & 1049 & 255 \\
    Carbohydrates & 17 & 4 & 482 & 113 \\
  \end{tabular}
  \caption{Energy content of the different building blocks of our food}
\end{table}

\section{Water}

Water doesn't contain any nutrients, but nevertheless is water of the biggest importance for the human body.
Without water, there's no life.
The human body consists our of about 60\% water (a newborn even up to 75\%).
Roughly 60\% of that water is found in the intracellular space (inside the cells).
The rest is in between the cells in the form of interstitial fluids (between the cells),
like blood plasma, lymph, urine, saliva, digestion fluids, tear liquid, nose secretes and sweat.


In the human organism, water mainly serves:
\begin{itemize}
\item to cover the needs of fluids in cells and tissues,
\item as a solvent and medium of transportation for nutrients, enzymes, hormones and so on,
\item to excrete metabolism end products over the urine,
\item to regulate the body temperature.
\end{itemize}

\subsection{How much Water does a Human Being Need?}

During a day, the human body looses\index{water consumption} 2--2.5 liters (8\sfrac{1}{2}\ -- 10\sfrac{1}{2}\ cups) of water, of which:
\begin{itemize}
\item 1.0 -- 1.5 L (4\sfrac{1}{4}\ -- 6\sfrac{1}{4}\ cups) through the urine
\item 0.4 -- 1.0 L (1\sfrac{3}{4}\ -- 4\sfrac{1}{4}\ cups) through respiration
\item 0.1 -- 0.5 L (3\sfrac{1}{2}\ fl.oz. -- 2\sfrac{1}{8}\ cups) over the skin, through sweat
  \item 0.1 -- 0.2 L (3\sfrac{1}{2}\ -- 7 fl.oz.) through feces
  \end{itemize}

  This loss of water has to be added back to our body on a daily base, through drinking and eating.
  For a healthy, barely physical active person, that means a daily fluid intake of about 30 -- 35 mL of fluids per kg (0.45 -- 0.54 fl.oz./lbs) body weight.
  A person who weights 70 kg (154 lbs) therefore needs a fluid supply of 2.0 -- 2.5 L (8\sfrac{1}{2}\ -- 10\sfrac{1}{2} cups) per day.
  Out of this amount, about 0.5 -- 1 L (2\sfrac{1}{8}\ -- 4\sfrac{1}{4} cups) fluids will be supplied through food (veggies, fruits, \ldots).
  This results then in the recommended amount of drinking about 1 -- 2 L (one to two quarts) of fluids a day.

  Please consider, that the water usage also depends on the outside temperature, physical activity levels and the state of health.
  When it's hot, under heavy physical labor or sport, sweating will cool down the body and
  in case of disease (ie. fever, diarrhea, vomiting) the body will also loose a lot of additional water.
  All these losses need to be compensated with additional fluid intake.

  \subsection{Is Water the Best Drink?}

  As drinks counts foods, which pretty much only furnish water to our body. 
  Fruit and veggie juices, milk, yogurt drinks and soft drinks deliver next to water plenty of nutrients and energy and  don't really count as drinks.
  Alcoholic drinks are non--essential food items, stimulants and are very bad at hydrating our body.
  On top of that, they are delivering a lot of unwanted energy (about 7 kcal/g, roughly 200 kcal per oz.).

% Swiss tap water - US tap water?
  
  Pure water is by far the best and cheapest drink. Tap water, depending on location filtered, is the best option to hydrate.
  In order to change things up, can water be flavored in different ways:
  \begin{itemize}
  \item Fruit or herbal tea
  \item Black or green tea, coffee\footnote{See special drinks in the section \ref{SpecialFoods}.}
  \item Broth
    \item Lemon water (juice of one lemon for one Liter (quart) of water)
  \end{itemize}

  People who are depended on bottled water, should prefer still water.
  The carbonation leeches vital oxygen from our body and adds unneccessary acids to our system.

  \subsection[Drinking too little, too much or the wrong drinks]{People who drink not enough, too much or the wrong things risk their health}

  Humans can survive multiple weeks without food, but without fluids only three days.
  Already a loss of fluids of 1\% or our body wieght leads to the first syptoms like head ache, difficulties focussing and thirst.
  
\end{document}