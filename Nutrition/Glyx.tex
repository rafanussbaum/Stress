\documentclass[../main.tex]{subfiles}
\graphicspath{{\subfix{../images/}}}
\begin{document}

\label{glycemic}

\section{Get and Stay Lean According the the Gly\-ce\-mix Index}

The glycemic index (GLYX)\index{glycemic inde}\index{GLYX} gives the physiological base to classify foods acoording to
what effect the food has on the blood sugar level.
That GLYX principle is easy: By choosing the right foods the production of insulin gets regualated and fats get burned instead of stored.

\section{How do You Measure the Glycemic Index (GLYX) of Foods?}

The blood sugar levels go up by specific amounts after consupmtion of food and decrease again.
This increase, followed by the decrease can be graphed as a curve.
The area underneath the curve is compared to the curve of glucose: with the increse and decrease that pure glucose triggers.
When you eat white bread, your blood sugar level will increse more and the area underneath is gibber than when you eat an apple.

Researcher attributed the reactions that \SI{50}{\gram} of glucose (= \SI{50}{\gram} pure carbohydrates) produces in the human body the GLYX value of 100.
Then they measure the increase of he blood sugar level after intake (area underneath the curve)
of \SI{50}{\gram} carbohydrates of other foods, like potatoes, potato chips, lentils, corn flakes, beer and so on --- and calcuates from that the specfic GLYX value.

\begin{table}[htb!]
  \centering
  \begin{tabular}{cl}
    \textbf{GLYX value or GI} & \textbf{Category} \\
    \hline
    \textless 51 & light weight = good \\
    51 -- 70 & middle weight = medium \\
    \textgreater 70 & heavy weights = bad \\
  \end{tabular}
  \caption{Food categories according to GLYX values}
\end{table}

\section{GLYX and Insuline}

\end{document}
