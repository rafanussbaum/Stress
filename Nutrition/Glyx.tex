\documentclass[../main.tex]{subfiles}
\graphicspath{{\subfix{../images/}}}
\begin{document}

\label{glycemic}

\section{Get and Stay Slim According the the Gly\-ce\-mix Index}

The glycemic index (GLYX)\index{glycemic index}\index{GLYX} gives the physiological base to classify foods according to
what effect the food has on the blood sugar level.
That GLYX principle is easy: By choosing the right foods the production of insulin gets regulated and fats get burned instead of stored.

\section{How do You Measure the Glycemic Index (GLYX) of Foods?}

The blood sugar levels\index{blood sugar level} go up by specific amounts after consumption of food and decrease again.
This increase, followed by the decrease can be graphed as a curve.
The area underneath the curve is compared to the curve of glucose: with the increase and decrease that pure glucose triggers.
When you eat white bread, your blood sugar level will increase more and the area underneath is gibber than when you eat an apple.

Researcher attributed the reactions that \SI{50}{\gram} of glucose (= \SI{50}{\gram} pure carbohydrates) produces in the human body the GLYX value of 100.
Then they measure the increase of he blood sugar level after intake (area underneath the curve)
of \SI{50}{\gram} carbohydrates of other foods, like potatoes, potato chips, lentils, corn flakes, beer and so on --- and calculates from that the specific GLYX value.

\begin{table}[htb!]
  \centering
  \begin{tabular}{cl}
    \textbf{GLYX value or GI} & \textbf{Category} \\
    \hline
    \textless 51 & light weight = good \\
    51 -- 70 & middle weight = medium \\
    \textgreater 70 & heavy weights = bad \\
  \end{tabular}
  \caption{Food categories according to GLYX values}
\end{table}

\section{GLYX and Insulin}

Insulin\index{insulin} is vital for the human body. If the pancreas\index{pancreas} stops the production of insulin,
the Peron suffers from diabetes\index{symptom!diabetes} and has to inject insulin.
But too much insulin increases the body weight\index{symptom!overweight}.
Insulin is the reason that people say: ``I barely eat anything at all and nevertheless gain weight''.
That's true, these people don't eat much, but the wrong things: foods with high GLYX.
If a person eats or drinks every two to four hours high GLYX foods, then the guardian insulin is standing in front of the fat deposits and won't let 
the fat molecules get back out of the fat depots. In this case, we can't loose weight.
That develops into the widespread disease: insulin resistance,\index{symptom!insulin resistance} the precursor to diabetes.
Many people are affected by this condition --- often without knowing it and wondering why they get fatter and fatter.

\subsection{Insulin makes hungry}

That's no news. Science knows that for decades: fats and proteins protect from overeating, they lead to saturation.

Carbohydrates beckon the insulin. Insulin makes hungry. The more we grow in weight, the more insulin we produce.
Already the French man Anthelme Brillat--Savarin (1755--1826), who wrote a famous theory of the joys of eating recognized in bread, rice and potatoes
foods which lead to gain weight.
The glycemic index (today GLYX) emerged in the late 70ies.
Dr.~David Jenkins, Professor for nutritional science at the university of Toronto developed that concept.
He gave foods a number form 1 to 100, depending on how much insulin they produce in the body upon consumption.
He then recommended his patients to eat only foods with a number less than 50. The result: they quickly lost weight.
The same recommendation also goes for us!

\subsection{Low GLYX makes slim}

The formula for GLYX is very easy: avoid high GLYX foods. That's the ones who contain a lot of sugars and starch.
You will get and stay slim\index{effect!loose weight}.

\subsection{Low GLYX: pure medicine}

Eating foods with low GLYX has the following beneficial effects:

\begin{itemize}
\item They are true fat burners.\index{effect!fat burner} Low GLYX means the the fat burns up in the muscles.
\item Hunger brake:\index{effect!hunger brake} The blood sugar levels stay in the healthy range. There's no episodes of raging hunger.
\item Doping you IQ:\index{effect!IQ doping} You supply your brain continuously with sugar. There's no lows in performance, not even at 10 pm.
\item  A brake for infarcts:\index{effect!stroke, prevent} When eat mainly foods with low GLYX also lowers their blood fat levels.\index{effect!blood fat level, lower}
  That protects from infarcts and strokes.
\item Good for the immune system:\index{effect!immune system strengthen} Low GLYX means often a lot of fibers.\index{food!fibers}
  They are very good for your biggest immune system; the colon.
  That makes you resistant to colds, sickness and cancer.
\item No gout:\index{symptom!gout} Low GLYX acts positively on the uric acid balance. There's no painful crystal of uric acid being formed in the joints.
\item Protection of muscles:\index{effect!muscle, protect} Low GLYX foods avoid that the body is decomposing the muscles.
\item Fountain of youth:\index{effect!fountain of youth} Excessive sugar reacts with an aggressive protein molecule to a tenacious mass,
  which glues onto cells and blood vessels:
  Advanced Glycosylated End Products, in short AGEs\index{AGEs}.
  AGEs lead to coronaries\index{symptom!coronary}, strokes\index{symptom!stroke}, Alzheimer's.\index{symptom!Alzheimer's}
  They are visible on the skin as age blotches.\index{symptom!age blotches on skin}
\item Prevents diabetes:\index{effect!diabetes, prevention} Eating high GLYX foods all the time means risking diabetes. The cells don't react to insulin anymore.
  At one point the pancreas yields and barely produces insulin anymore as a result.
  The kidneys excrete sugars. The blood sugar level are constantly heightened and destroy smaller and bigger blood vessels.
  Results: diabetic feet\index{symptom!diabetic feet}, amputation\index{symptom!amputation},
  kidney failure\index{symptom!kidney failure} and blindness\index{symptom!blindness}. 
\end{itemize}

\subsection{But --- GLYX alone doesn't lead to being slim, healthy, happy and in shape!}

There's one factor in the  whole discussion about the widespread disease obesity which is often overlooked:
A wrong ratio between energy intake and physical activity.\index{physical activity level}
Without additional movement you won't get or stay slim and in shape.

\subsection{Energy Density and Nutrient Density, example refined sugars}

Sugars in foods increases the energy density\index{food!energy density} (the calorie content) and simultaneously
decreases the nutrient density\index{food!nutrient density} (ratio of vitamins and minerals to calories).

Sugar, which consists purely of the carbohydrate saccharose, has a nutrient density of zero.
They are sometimes called ``empty'' calories\index{empty calories}, in a simplifying manner.
Strawberries and broccoli for instance have a low amount of calories in comparison, but they are rich at vitamins and minerals,
and therefore have a high nutrient density.

The advantage is that most low and medium GLYX foods have a high nutrient density.

\end{document}
