\documentclass[../main.tex]{subfiles}
\graphicspath{{\subfix{../images/}}}
\begin{document}

With malnutrition over a longer time, we run the risk of getting sick, in any form, physical or psychologically.
The risk of a nutrition caused disease can be significantly reduced through a  healthy and adequate diet,
and by doing so we provide the requirements for the daily high usage to body and psyche.

\subsection{What is a Healthy Diet?}

The basis of a healthy nutrition is a supply of energy and nutrients in the sense of a full--fledged diet,
based on the needs and the load of the individual.
Under full-fledged we mean a quantitaviely and qualitatively balanced selection of diverse foods,
which are beneficial to our health and supplies our organism with all vital nutrients.
A full--fledged diet also considers technological, ecological and social aspects.
It differs from the rgular mixed diet by consciously choosing foods which:
\begin{itemize}
\item are regional
\item seasonally available
\item which do without additives
\item are produced ecologically and/or livestock--friendly (organic products)
\item which are processed, packed and transport in an environment friendly manner
  \item warrant a fair price/income (ie. coffee from Max Havelar)
  \end{itemize}

  That means in short: Food as close to it's natural state as possible, seasonal and from the region
  prepared in a gentle manner and in the amount, which correspond to the amounts given by the food pyramid (MyPlate).

  \subsection{The Food Pyramid}

  
\begin{figure}[htb!]
\centering
  \includegraphics[width=14cm]{FoodPyramid}
  \caption{A revised version of the food pyramid~\cite{FoodPyramid}}
\end{figure}

\begin{table}[htb!]
  \centering
  \begin{tabular}{ll}
    \textbf{Nutrient/Food component} & \textbf{Target area (per day)} \\
    \hline
    Energy & 7.4--10.6 MJ respectively 1800--2500 kcal \\
    Carbohydrates & about 50\% of energy$^{***}$ \\
    Fibers & about \SI{30}{\gram}. (1~oz.)$^{***}$ \\
    Saccharose (table sugar) & moderate use (about 10\% of the energy) \\
    Proteins & 10--20\% of the energy$^{**}$ \\
    Fats & about 30\% of the energy \\
    Long chain, saturated fatty acids & \textless 10\% of the energy \\
    Mono--unsaturated fatty acids & about 10\% of the energy \\
    Poly--unsaturated fatty acids & about 7\% of the energy \\
    Ratio linolic to $\alpha$--linonenic acid & \sfrac{5}{1} \\
    Cholesterol & about \SI{300}{\milli\gram}$^{***}$ \\
    Vitamins & 100\% of the DACH reference values for 19--65 year olds \\
    Buld and trace elements & 100\% of the DACH reference values for 19--65 year olds \\
    Table salt & \SI{6}{\gram} \\
    Water intake over drinks and foods & 250--\SI{270}{\milli\liter}/MJ respectively 1.0 - \SI{1.1}{\milli\liter}/kcal \\
    & (inclusive oxidative water) \\
    \hline
    \multicolumn{2}{l}{\footnotesize{* PAL = physical activity level}}
  \end{tabular}
  \caption[Target areas of food pyramid]{Target areas of the food pyramid for adults of 19--65 years old (PAL*1.4)}
\end{table}
\end{document}