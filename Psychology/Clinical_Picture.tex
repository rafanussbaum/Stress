% !TeX TXS-program:bibliography = 
% !TeX spellcheck = en_US
% !TeX root = ../Book.Stress_regulation.tex
\documentclass[../main.tex]{subfiles}
\graphicspath{{\subfix{../images/}}}
\begin{document}

The following remarks are not meant to encourage stress regulation strainer or anybody reading these notes to become themselves active in a therapeutic manner in the case of mental disorders. 
This is purely reserved for psychiatrists or psychologists.
But on the other hand, stress regulation trainer will have with a high probability clients, which suffer form mental or other disorders.
These notes serve to get sensitized for these cases and then seek the collaboration with a specialized mental health professional.

\section{Anxiety and Panic Disorder}

Fear is a daily phenomena.
We all know fears, worries and doubts. 
But normally, we can deal with them in an appropriate way.
About 15\% of the people suffer at least once from severe and/or lasting fears, which limit them in their daily lives and decrease the life quality.
Most anxiety disorders start between the 15th and 35th year of live and women are more often concerned than men.

Anxiety disorders\index{anxiety disorder} are the most common mental disorders.
Very often, they don't get diagnosed by doctors or just very late.
There are often years between the first occurrence of symptoms and a correct diagnosis and therefore a start of a therapeutical treatment.
In the doctor's cabinet a lot of patients state different physical symptoms (dizziness, pressure on the chest area, fast heart rate, lessened capacity, \ldots) and don't say: "I'm afraid".

\epigraph{The worst fears are the ones which are unexpressed. A lot of progress has been made by putting your fears in words.}{translated by author from \textit{Lewis Richmond}}

A study showed, that only 6\% of panic patients, which consulted a medical doctor, panic attacks even have been considered.
26\% of the patients received the wrong and irritating answer that "nothing" is wrong with them.
This missing awareness among many medical professionals is very regrettable, especially as anxiety disorders are well treatable with psychotherapy.

The roots of the words fear and angst come from Latin and stand for "narrowness" and "to gag, to choke".
That definitely describes very well the physical symptoms of fears. 
Anxiety in the original sense describes an undefined and not oriented feeling of being threatened and is therefore different from \emph{fear}\index{fear}, which relates to a concrete threat.

In psychiatry, there's an important differentiation made exactly in this point:

\begin{itemize}
	\item In \emph{phobias}\index{phobia}, the fear emerges in specific situations or in contact with certain objects, animals and so on, which by themselves are not dangerous. These objects or situations are avoided as much as possible.
	\item With the \emph{anxiety disorders}, the fear appears independent independently of situation. 
\end{itemize} 

\epigraph{The art of living is to always get back up once more than you have been knocked down.}{attributed to \textit{Winston Churchill}, translated by author}

We all know, that fears are important for our survival (especially for the survival of our ancestors).
Out of that reason, we have a "biological alert system", which parallels in many ways the warning system "pain".
Fear prepares our body for fast actions (fight or flight), by rapidly triggering the "stress cascade" of hormones.
Fear is an integral trigger an component of stress.
Considered in this way is an anxiety disorder nothing else than an exaggerated version of a normal biological reaction.

Fear can manifest in very different ways. 
But every single time it consists of three main elements:

\begin{enumerate}
	\item A physical part (vegetative: sweating, feeling of pressure, trembling, \ldots)
	\item A part which concerns thinking and feeling ("There's a catastrophe about to unfold", "Let's get out of here", \ldots )
	\item A part, which expresses in behavior ("fight or flight", aggression or yielding, \ldots)
\end{enumerate}

These different parts are individually perceived in different ratios. Some people mostly feel the physical symptoms (palpitation of the heart, pressure on the stomach, \ldots) where as other mainly perceived the emotions and the thoughts of fear.

The exact cause for anxiety disorders isn't known. 
By the fact, that this mental disorder is distributed equally over the layers of society in the whole world, there's a conclusion that these disorders also have a genetic factor.
Anxiety disorders are then created, when outside burdening factors meet this "inner" genetic base ("vulnerability").
(This "vulnerability model" is nowadays assumed to be true for pretty much all mental disorders.)
	
Most of the patients don't only suffer from a \underline{single} anxiety disorder. 
More than 50\% of the concerned people also fulfill the criteria for at least one more anxiety disorder.
Also, there's a very tight connection between anxiety and depression. 
Fear is often a symptom of depression, but depression is also often an effect of anxiety disorder.

Untreated, anxiety disorder leads to a chronification with increasing isolation and the heightened risk of addiction and suicide.



\paragraph*{Forms of anxiety disorders}

\begin{description}
	\item[Agoraphobia:]\index{agarophobia} Avoidance of public paces and crowds. Avoidance of longer travels or traveling alone.
	Here, it's less the situation itself which is perceived as threatening, but the fear is in case of anxiety not being able to flee from the situation or to not getting any help. 30\% of these patients are severely impaired in their mobility (like for instance in leaving their apartments). \\
	Orientation question: "Are you avoiding public places, crowds and bigger travels (or then traveling alone) out of fear of anxiety attacks, or then not being able to leave the situation, in case of an anxiety attack?"
	\item[Social Anxiety:]\index{anxiety:social} Concerned people avoid situations, in which they could find themselves in the focus of people's attention.
	This disorder is seen as the most often overlooked anxiety disorder.
	In the past, these people just have been called "awkwardly shy".
	These fears often turn around doing or saying something that other people might consider as embarrassing or  ridiculous.
	Concerned people don't only avoid public speaking, but also meals with other people, visits,~\ldots \\
	Orientation question:
	"Are you afraid of or avoiding situations, in which you can be observed and judged by other people, like for instance public speaking, gatherings, parties or talks?"
	\item[Specific Phobias:]\index{phobia} This is an ongoing, severe fear of certain situations, things or animals, which cause significant suffering to the affected person.
	The fears can be linked to animals (spiders, mice, \ldots), environmental events (storm, darkness, fire, \ldots), the sight of blood, syringes,~\ldots, usage of elevators, planes, being high up, \ldots
	\item[Panic Disorder]\index{panic disorder}
	Main criteria are reoccurring surprising panic attacks, which are not caused by external circumstances. 
	They start suddenly and go along with strong physical symptoms (see below). The following things can be linked to it: a feeling of estrangement, alienation, the fear of death, agony and the anxiety to "loose the mind".
	Panic attacks reach their maximum in the first three minutes and normally subside within 10 to 30 minutes. \\
	Orientation question:
	"Do you sometimes suffer from sudden heart palpitations, sweating, breathlessness or other symptoms and does this lead to anxiety?"
	\item[Generalized Anxiety Disorder:]
	In this case, the fear isn't linked to a specific situation, but is described as being "freely floating\footnote{floating: hoovering, fluctuating, intertwining}".
	The content of the fears are enduring, strongly exaggerated haunting fear about he financial situation, the own health or the one of close people,~\ldots  
	On top of that comes the "fear of the anxiety attack". \\
	Possible orientation question:
	"Are you often worried about different things? Are you unable to or is it hard to control these fears and worries? (for instance family, professional, financial, health matters) 
\end{description}

\epigraph{The big paradox in life is that when things shouldn't stay the same old, then they can't stay the same old.}{\textit{Franz Xaver von Baader}, translated by author}

\paragraph*{Symptoms}

\begin{description}
	\item[Physical Symptoms:]
	Heart palpitation, sweating, trembling, breathlessness, feelings of asphyxiation, pain in the chest, dizziness, dryness of the mouth, feeling dazed, a tingling feeling, numbness, general weakness,~\ldots
	\item[Mental, Psychological Symptoms:]
	A feeling of unreal, to not be yourself anymore, fears to loose control, loosing your mind, dying.
	\item[Social and Health Related Effects:] 
	Inability to do certain things, inability to work, strong dependency on attachment figures, depression (50\%), suicide (15\%), substance abuse (alcohol, tranquilizers).
\end{description}

\section{Obsessive Compulsive Disorder}

With obsessive compulsive disorder there's the inner urge to think or do certain things (compulsive thoughts, compulsive acts).
Concerned people are fighting against these compulsions, perceive them as excruciating and themselves see them as exaggerated and not making too much sense.
Patients try to suppress these urges (stereotypies) and often fail to, which in it's own right leads to anxiety.

Most of us know "compulsive acts", which can be stronger in case of stress or exhaustion.
Examples can be repeated checking if the stove is off, or the door is locked, not being able to let go of certain melodies, counting the steps of the stair,~\ldots

Compulsive acts in the frame of a mental disorder can hinder to such an extend, that the job or the hobbies are very restricted.

\noindent Orientation question: 
"Are there uncomfortable or nonsensical thoughts or actions, which you can't get out of your head respectively have to put on action? Even when you try resisting the impulse?"

\paragraph*{Symptoms}

\begin{description}
	\item[Compulsive thougths:] Affected people  imagine certain catastrophes or feel the urge to do things, which are strongly against their beliefs (compulsive worries, compulsive urges).
	That could for instance be a mother, who has the recurring thought of killing her beloved baby.
	With \emph{compulsive brooding}, the patient is urged to think through the same thought over and over again. 
	To find decisions or solutions is impossible in this state.
\end{description}

\end{document}