\documentclass[../main.tex]{subfiles}
\graphicspath{{\subfix{../images/}}}
\begin{document}

\section{The Term}

In the following parts we will use most often the term ``psychological disorders'' instead of ``psychological illnesses''.
The reason for that is that terms like ``healthy'' and ``sick'' are especially diffuse and nebulous in the domain of psychology.
Also, the term ``psychological illnesses'' invites more likely to stigmatize the the concerned person.
Terms like ``mentally ill'' and ``insane'' aren't far from there anymore.
A disorder on the other side is more neutral and hints towards the aspect that is could be a temporary condition.

In short, psychological disorders can be defined as significant deviation of the \emph{experience} and/or \emph{behavior}
of a person from psychologically healthy people. Specifically concerned are the areas of \emph{thinking, feeling} and \emph{actions}.

The term of illness is fundamentally problematic in medicine.
Next to the deviation from an earlier defined norm, there's also the subjectively experienced suffering to be considered.
These terms ``norm'', ``objectivity'' and ``subjectivity'' are very difficult to handle for gauging psychological disorders.

The ``average standard'' describes the behavior that the majority of humans of a certain gender and a certain age shows
inside of a certain group/society/culture in certain situations.
All behavior which deviates from that would therefore be ``abnormal'' respectively ``deviating''.
Often the purely quantitative term ``statistical rarity'' is linked to that concept.

The fact that norms are important for an ordered cohabitation in society is undisputed.
They bring the individual protection, security and a feeling of safety.
\end{document}
