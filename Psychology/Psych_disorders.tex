\documentclass[../main.tex]{subfiles}
\graphicspath{{\subfix{../images/}}}
\begin{document}

\section{The Term}

In the following parts we will use most often the term ``mental disorders''\index{disorder!mental} instead of ``psychological illnesses'' or ``psychological disorder''.
The reason for that is that terms like ``healthy'' and ``sick'' are especially diffuse and nebulous in the domain of psychology.
Also, the term ``psychological illnesses'' invites more likely to stigmatize the the concerned person.
Terms like ``mentally ill'' and ``insane'' aren't far from there anymore.
A disorder on the other side is more neutral and hints towards the aspect that is could be a temporary condition.

In short, mental disorders\index{disorder!mental!definition} can be defined as significant deviation of the \emph{experience} and/or \emph{behavior}
of a person from psychologically healthy people.
Specifically concerned are the areas of \emph{thinking, feeling} and \emph{actions}.

The term of illness is fundamentally problematic in medicine.
Next to the deviation from an earlier defined norm, there's also the subjectively experienced suffering to be considered.
These terms ``norm''\index{terms!norm}\index{terms!normal}, ``objectivity''\index{terms!objectivity} and
``subjectivity''\index{terms!subjectivity} are very difficult to handle for gauging mental disorders.

The ``average standard'' describes the behavior that the majority of humans of a certain gender and a certain age shows
inside of a certain group/society/culture in certain situations.
All behavior which deviates from that would therefore be ``abnormal'' respectively ``deviating''.
Often the purely quantitative term ``statistical rarity'' is linked to that concept.

The fact that norms are important for an ordered cohabitation in society is undisputed.
They bring the individual protection, security and a feeling of safety.
Certainly, there are different perspectives about what ``normal'' behavior is.
Society  has the desire, that the person integrates into it,
takes over responsibility and ``functions'' without ruffling feathers.
On the other hand, the person wishes to be content and happy.
The therapist or the stress regulation trainer might have a different view for the client,
in terms of a healthy personality structure, an efficient regulation of stress and a
desired development of the personality.
These three perspectives of ``normality'' might not forcibly be the identical.

On the other hand, it's also true, that not every deviating behavior must be a sign of illness.
It can even be very adequate to become sick under certain conditions
(for instance food poisoning or immense grief).

\noindent These four criteria should be generally be able to be asked of a ``normal'' behavior\index{normal!behavior}:

\begin{enumerate}
\item Self sufficient, according to the age\index{self--sufficiency}
\item Behavior, which is adequate to the situation\index{behavior!adequate}
\item Capacity to create relationships\index{relationship!capable of}
  \item Harmonic interplay between thinking, feeling, wants and actions.
\end{enumerate}

\noindent According to experience, problems in the following sections give general hints of a mental disorder:

\begin{enumerate}
\item Capacity to enjoy\index{problem!capacity to enjoy}\index{symptom!capacity to enjoy}
\item Capacity to maintain relationships\index{problem!capacity to maintain relationships}\index{symptom!capacity to maintain relationships}
  \item Capacity to perform\index{problem!capacity to perform}\index{symptom!capacity to perform}
\end{enumerate}

The manual of mental disorders DSM--IV (compare to \ref{sec:classif} Classification) defines seven categories,
by which behavior can be labeled as being ``deviating'':

\begin{enumerate}
\item \textbf{Performance pressure or handicap}\index{problem!performance pressure}

  A person experiences personal performance pressure or constraints in the mental functions,
  which lead to a deterioration of the bodily or mental state or cause a loss in the capacity to act.
  
  For instance a woman with an anxiety disorder, which can't leave her house and lead a professional or social life anymore.
  Even going shopping around the corner can become an insurmountable obstacle.

\item \textbf{Maladaptation}\index{problem!maladaptation}

  A person behaves in a way which avoids her from reaching their own goals, doesn't care about the personal well--being,
  holds other back from reaching their goals or  doesn't measure up to society's needs.
  A man who's heavily dependent on alcohol will very probably let himself go internally and externally.
  A structures life with maintaining an income won't be possible anymore.
  His kids maybe don't get the support and sustenance they need.

\item \textbf{Irrationality}\index{problem!irrationality}

  A person talks and behaves in a way that they appear crazy or incomprehensible.
  A young person reacting to and audibly answering to voices, which nobody except him can hear, behaves irrationally.

\item \textbf{Unpredictability}\index{problem!unpredictability}

  A person behaves in a unpredictable way or erratically changing from situation to situation, as if that person wouldn't have any control over it.
  A young person pushing an unknown person in front of a subway train without any acceptable reason behaves irrational (not only that!).

\item \textbf{Extraordinary and statistical rarities}\index{problem!statistical rarity}

  A person shows behaviors, which appear statistically seen rarely occur and infringe the social standards of what's acceptable or desirable.
  Rare occurrence alone isn't sufficient to classify a behavior as deviating.

\item \textbf{Discomforting for spectators}\index{problem!discomfort for spectators}

  A person evokes emotional discomfort in others, which feel threatened or concerned.
  For instance a person, which monologues in a aggressive or cussing way, causes unease in other people.

\item \textbf{Infringing moral and societal norms}\index{problem!infringing moral/societal norm}

  A person offending the expectations of how to behave in respect to social norms.
  According to this criteria, people who don't want to work could be classified as deviating.
  
\end{enumerate}

\noindent The more criteria are met, the easier the diagnosis of ``deviating behavior''.\index{behavior!deviating}

\vspace{5mm}

Looking at these criteria it becomes evident, that judging behaviors has a very subjective note.
The person, who is allowed to decide over such criteria, has an instrument of power in their hands.
In our society, these people are psychiatrist, lawyers,\ldots

\section{Classification}\label{sec:classif}

Psychiatry is trying for more than a hundred years to describe mental disease patterns in a clear and unambiguous way.
And regardless of all critical voices in this point, there are characteristic symptoms.
An experienced investigator can recognize these symptoms in a coherent and reproducible way and be
put in psycho--pathological\footnote{according to the science of mental diseases} categories. 

Classical psychiatry has very strong biological roots and classifies the mental disorders in a
triadic\footnote{divided in three} system:


\begin{enumerate}
\item \textbf{Exogenous mental disorders}\index{disorder!mental!exogenous}

  The cause is an observable physical disease, like for instance in the brain.
  Example are: poisoning, also with alcohol, brain tumor, insufficiency of the thyroid gland. 
  Treatment is primarily to combat the physical ailment, which is the cause.

\item \textbf{Psychogenic mental disorders}\index{disorder!mental!psychogenic}

  These disorders is seen as mental reaction to external events or as the effect o fa neurotic conflict (see below).
  Very probably they are statistically very strong deviations from the human psyche.
  Examples are ``light depressions'', anxiety, OCD,\ldots
  Treatment consists mainly of psychotherapy.

\item \textbf{Endogenous mental disorders}\index{disorder!mental!endogenous}

  An endogenous disorder is a one, where no exogenous or psychological causes can be found.
  These diseases remind in their intensity of physically caused mental disorders.
  According to today's knowledge, these disorders are caused by a combination of a genetic disorder and
  a ``disrupted'' hormonal balance in the brain (neurotransmitter like dopamine, serotonin,\ldots).
  Examples are ``heavy'' depression, mania and schizophrenic psychosis.
  The basis of the treatment is medication which helps to regulate the neurotransmitter balance.
\end{enumerate}

This triadic system gets a lot of criticism these days.
Mostly because the causes of mental disorders are often much more complex than this simplification makes you believe.
So it is know now, that also psychogenic mental disorders have a biological dimension.
That means, also in this case can changes in the neurotransmitter balance, as well as a changed pattern of blood circulation
and activities of certain regions of the brain.
On the other side, also endogenous mental disorders have a pronounced mental component
and are partially accessible to psycho--therapeutic treatments.

\epigraph{To be of age isn't the one, who believes to be able to surmount fear, sadness or desperation,
  but the one who is able to see through it and grow due to it.}{translated by author from \textit{Karlfried Graf D\"urckheim}}

The classification into the triad has been maintained to a big degree up to this day, especially as they are easily understandable and 
easy to handle.
This classification also has the advantage of showing, what kind of disorders belong exclusively into the care of therapeutic
specialist (psychiatrist, partially psychotherapist) and which also can be treated or
the treatment supported by stress regulation trainer and similar professions.
This second category is exclusively the category of the psychogenic mental disorder.

Also the terms \emph{psychosis}\index{disorder!psychosis} and \emph{neurosis}\index{disorder!neurosis} can be be classified into these three categories.
These terms come from Sigmund Freud's psychoanalysis\index{psychoanalysis}.
Even though these terms are deemed dated and aren't contained in modern diagnosis systems, they are nevertheless still widely used to this day.
These two terms aren't uniquely defined and definitely not scientifically proven.

The \emph{neurosis} corresponds to a certain extend to the \emph{psychogenic mental disorders} in the triadic system,
where as \emph{psychosis} could be seen as being equivalent to the \emph{endogenous mental disorders}.
The discussion above about different ``primary caregiving groups'' also goes for psychosis and neurosis.
In the specific case, the differentiation between a psychosis and neurosis can be difficult.
So is for instance the ``blooming'' symptoms of schizophrenic psychosis (delusional ideas, scattered thinking processes) pretty clearly to categorize,
but this can already be much harder in the case with depressive symptoms.
There's also to consider that also ``neurotic--depressive'' patients can commit suicide: the big worry of the care of people with mental disorders.

\subsection{Definitions}

\begin{description}
\item[Neurosis]\index{disorder!neurosis} A mental disorder, without any detectable physical source.
  In the neurosis as opposed to psychosis, the control or reality is not or just little disturbed,
  or then as a mental disorder caused by the circumstances in life.
\item[Psychosis]\index{disorder!psychosis} Severe mental disorder with a structural change in perception (as opposed to the neurosis).
  Stands for the everyday expression of ``being crazy''.
\end{description}

\vspace{5mm}

In today's understanding of mental disorders, individual \emph{symptoms}\index{disorder!mental!symptom} (effects of a disorder)
or \emph{syndromes}\index{disorder!mental!syndrome} (groups of symptoms) don't get associated with specific causes.
More so, they are collectively described as syndromes, whose causes are often \emph{multi--factorial} (involving different causes).
This method corresponds to modern research, which shows that the human brain can react to outside factors to an advanced age.
Which type of factors they are isn't always decissive for the therapy.
So can for a depression the medication with anti--depressiva,
increased physical activity or conversation based psychotherapy therapy be as successful, dependng on circumstances .
The boundaries between body and mind, between natural  and psychological science are getting succesively more vague.

\setlength{\epigraphwidth}{0.8\textwidth}
\epigraph{We can also help children without giving them pills, like for instance with simple changes of everyday life.
  A useful example is the story of a young Englishman, who went to school at the end of the 19th century.
  According to modern standards, he would have been classified as hyperactive.
  In order to burn off excessive energy, this restles mind negotiatied with his teachers that he was allowed to run once around the
  school after every hour.
  Sure enough, everyday life became like this much more supportable --- as much as for the student also for his teachers.
  Later on in life, this Englishman totally renounced sports.
  His name: Winston Churchill.
}{translated by author from \textit{J\"org Blech, Die Krankheitserfinder}}
\setlength{\epigraphwidth}{0.4\textwidth}


\end{document}

