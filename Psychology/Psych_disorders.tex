\documentclass[../main.tex]{subfiles}
\graphicspath{{\subfix{../images/}}}
\begin{document}

\section{The Term}

In the following parts we will use most often the term ``psychological disorders'' instead of ``psychological illnesses''.
The reason for that is that terms like ``healthy'' and ``sick'' are especially diffuse and nebulous in the domain of psychology.
Also, the term ``psychological illnesses'' invites more likely to stigmatize the the concerned person.
Terms like ``mentally ill'' and ``insane'' aren't far from there anymore.
A disorder on the other side is more neutral and hints towards the aspect that is could be a temporary condition.

In short, psychological disorders can be defined as significant deviation of the \emph{experience} and/or \emph{behavior}
of a person from psychologically healthy people. Specifically concerned are the areas of \emph{thinking, feeling} and \emph{actions}.

The term of illness is fundamentally problematic in medicine.
Next to the deviation from an earlier defined norm, there's also the subjectively experienced suffering to be considered.
These terms ``norm'', ``objectivity'' and ``subjectivity'' are very difficult to handle for gauging psychological disorders.

The ``average standard'' describes the behavior that the majority of humans of a certain gender and a certain age shows
inside of a certain group/society/culture in certain situations.
All behavior which deviates from that would therefore be ``abnormal'' respectively ``deviating''.
Often the purely quantitative term ``statistical rarity'' is linked to that concept.

The fact that norms are important for an ordered cohabitation in society is undisputed.
They bring the individual protection, security and a feeling of safety.
Certainly, there are different perspectives about what ``normal'' behavior is.
Society  has the desire, that the person integrates into it,
takes over responsibility and ``functions'' without ruffling feathers.
On the other hand, the person wishes to be content and happy.
The therapist or the stress regulation trainer might have a different view for the client,
in terms of a healthy personality structure, an efficient regulation of stress and a
desired development of the personality.
These three perspectives of ``normality'' might not forcibly be the identical.

On the other hand, it's also true, that not every deviating behavior must be a sign of illness.
It can even be very adequate to become sick under certain conditions
(for instance food poisoning or immense grief).

\noindent These four criteria should be generally be able to be asked of a ``normal'' behavior:

\begin{enumerate}
\item Self sufficient, according to the age
\item Behavior, which is adequate to the situation
\item Capacity to create relationships
  \item Harmonic interplay between thinking, feeling, wants and actions.
\end{enumerate}

\noindent According to experience, problems in the following sections give general hints of a psychological disorder:

\begin{enumerate}
\item Capacity to enjoy
\item Capacity to maintain relationships
  \item Capacity to perform
\end{enumerate}

The manual of mental disorders DSM-IV (compare to \ref{sec:classif} Classification) defines seven categories,
by which behavior can be labeled as being ``deviating'':

\begin{enumerate}
\item \textbf{Performance pressure or handicap}

  A person experiences personal performance pressure or constraints in the mental functions,
  which lead to a deterioration of the bodily or mental state or cause a loss in the capacity to act.
  
  For instance a woman with an anxiety disorder, which can't leave ehr house and lead a professional or social ife anymore.
  Even going shopping around the cornercan become an insurmountable obstacle.

\item \textbf{Maladaptation}

  A person behaves in a way which avoids her from reaching their own goals, doesn't care about the personal well--being,
  holds other back from reaching their goals or  doesn't measure up to society's needs.
\end{enumerate}


\section{Classification}\label{sec:classif}

\end{document}
