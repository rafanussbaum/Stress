\documentclass[../main.tex]{subfiles}
\graphicspath{{\subfix{../images/}}}
\begin{document}

\epigraph{Humankind's true field of study is the human itself.}{\textit{Johann Wolfgang von Goethe}}

Can depressions be treated?
What is a borderline personality disorder?
Why is the schizophrenic person ``split''?
What can we do against fears and anxiety?
How can we deal with the sadness of the partner?
Why am I so stressed in office, while my colleague isn't?

Questions about how our soul works are as old as humankind.
The interest in psychological questions is big, also outside of the academic field of psychology.
Each book store has many books with psychological advice, journals have their own psychological self--help sections
and in movies we often hear stories about people who are suffering fro a mental affliction.

Sure enough it holds a big fascination to occupy yourself with how mental processes work and to explain them.
We often do that also in our immediate surrounding.
Nevertheless, the recording, description and explaining of mental phenomena often meets resistance:
the phenomena of the soul are often not very clear cut and hard to grasp.

What makes the situation difficult is the fact, that mental disorders
can't be diagnosed by an x-ray or through a lab analysis.
Also, the perception of how the psyche works and what's ``normal'' and ``healthy''
is always highly influenced by the time people are living in, society and culture.
In that sense in some cultures (for instance shamanic cultures),
epilepsy and schizophrenia was understood to be a religious phenomena.
The affected people were seen as people who are especially close to the gods.
For us modern people, they are diseases which urgently require treatment.
Additionally, there are indications that in the christian medieval time,
anorexia found it's expression in religious fasting practices.
These days, anorexia is seen as being the psychiatric affliction with the highest mortality rate.

At this point, questions come up about the definition or disease and health.
Who's in charge of deciding, if an individual is sick and needs treatment?
Is it okay to be psychologically sick without seeking out treatment?
When does the mourning for a loved one become depression, an adaptation disorder or even
post--traumatic stress disorder?
Or when is it just mourning?

\epigraph{Only when the ratio will be king and the heart the priest shall this world get healthy.}{\textit{Khalil Gibran}}

These questions show us that the classification of mental disorders isn't always easy.
And it leaves us wondering if not many of us did suffer temporarily under phenomena,
which somebody else would have had classified as a ``mental disorder''.

Especially in the case of mental afflictions, health and disease present as a continuum,
in which we tend to fall a bit more on one and another time on the other side.

Up to this date, there's no generally accepted psychological theory.
There's only model ideas, which are supposed to explain certain mental phenomena,
which at times even directly contradict each other.



\end{document}
