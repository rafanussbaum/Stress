%%%%%%%%%%%%%%%%%%%%%%%%%%%%%%%%%%%%%%%%%
% Beamer Presentation
% LaTeX Template
% Version 1.0 (10/11/12)
%
% This template has been downloaded from:
% http://www.LaTeXTemplates.com
%
% License:
% CC BY-NC-SA 3.0 (http://creativecommons.org/licenses/by-nc-sa/3.0/)
%
%%%%%%%%%%%%%%%%%%%%%%%%%%%%%%%%%%%%%%%%%

%----------------------------------------------------------------------------------------
%	PACKAGES AND THEMES
%----------------------------------------------------------------------------------------

\documentclass{beamer}

\mode<presentation> {

% The Beamer class comes with a number of default slide themes
% which change the colors and layouts of slides. Below this is a list
% of all the themes, uncomment each in turn to see what they look like.

%\usetheme{default}
%\usetheme{AnnArbor}
%\usetheme{Antibes} % My theme for this talk
%\usetheme{Bergen}
%\usetheme{Berkeley}
%\usetheme{Berlin}
%\usetheme{Boadilla}
%\usetheme{CambridgeUS}
%\usetheme{Copenhagen}
%\usetheme{Darmstadt}
%\usetheme{Dresden}
%\usetheme{Frankfurt}
%\usetheme{Goettingen}
%\usetheme{Hannover}
%\usetheme{Ilmenau}
%\usetheme{JuanLesPins}
%\usetheme{Luebeck}
%\usetheme{Madrid}
%\usetheme{Malmoe}
%\usetheme{Marburg}
%\usetheme{Montpellier}
%\usetheme{PaloAlto}
%\usetheme{Pittsburgh}
%\usetheme{Rochester}
\usetheme{Singapore}
%\usetheme{Szeged}
%\usetheme{Warsaw}

% As well as themes, the Beamer class has a number of color themes
% for any slide theme. Uncomment each of these in turn to see how it
% changes the colors of your current slide theme.

%\usecolortheme{albatross}
%\usecolortheme{beaver}
%\usecolortheme{beetle}
%\usecolortheme{crane}
%\usecolortheme{dolphin}
%\usecolortheme{dove}
%\usecolortheme{fly}
%\usecolortheme{lily}
%\usecolortheme{orchid}
%\usecolortheme{rose}
%\usecolortheme{seagull}
%\usecolortheme{seahorse}
%\usecolortheme{whale}
%\usecolortheme{wolverine}

%\setbeamertemplate{footline} % To remove the footer line in all slides uncomment this line
%\setbeamertemplate{footline}[page number] % To replace the footer line in all slides with a simple slide count uncomment this line

%\setbeamertemplate{navigation symbols}{} % To remove the navigation symbols from the bottom of all slides uncomment this line
\setbeameroption{show notes}
}

\usepackage{graphicx} % Allows including images
\usepackage{booktabs} % Allows the use of \toprule, \midrule and \bottomrule in tables
\usepackage{enumitem}
\usepackage{hyperref} 
\usepackage{tikz}
\usetikzlibrary{shapes.geometric, arrows}


\tikzstyle{decision} = [diamond, minimum width=3cm, minimum height=1cm, text centered, draw=black, fill=green!30]
\tikzstyle{process} = [rectangle, minimum width=3cm, minimum height=1cm, text centered, draw=black, fill=orange!30]
\tikzstyle{startstop} = [rectangle, rounded corners, minimum width=3cm, minimum height=1cm,text centered, draw=black, fill=red!30]
\tikzstyle{io} = [trapezium, trapezium left angle=70, trapezium right angle=110, minimum width=3cm, minimum height=1cm, text centered, draw=black, fill=blue!30]
\tikzstyle{arrow} = [thick,->,>=stealth]

\graphicspath{{images/}} % Location of the slide background and figure files

%----------------------------------------------------------------------------------------
%	TITLE PAGE
%----------------------------------------------------------------------------------------

\title{Exercises} % The short title appears at the bottom of every slide, the full title is only on the title page

\author{Raphael Nussbaum} % Your name
\institute[AJ] % Your institution as it will appear on the bottom of every slide, may be shorthand to save space
{
AJ Tutoring \\ % Your institution for the title page
\medskip
\textit{raphaelnussbaum@ajtutoring.com} % Your email address
}
\date{\today} % Date, can be changed to a custom date

\begin{document}

\begin{frame}
\titlepage % Print the title page as the first slide
\end{frame}
%Learning

\begin{frame}
\frametitle{Learning}
\begin{itemize}
\item[-] \href{run:./PACE_presentation.pdf}{\underline{PACE}}
(\href{run:./PACE_Handout_student.pdf}{\underline{Student}} and 
\href{run:./PACE_Handout_instructor.pdf}{\underline{Instructor Handout}})
 (\textit{crossover, integrating both hemispheres, meridians, proprioception})\\
 learning capacity and readiness, hand--eye coordination, waking up,  reconnects emotions and neocortex, calm.
 
\item[-] \href{run:./Metacognitive_Instructor.pdf}{\underline{Metacognitive Exercises}} 
(\href{run:./Metacognitive_Student.pdf}{\underline{Student Handout}} and
\href{run:./Metacognitive_Exercises.pdf}{\underline{Practice Sheets}})
 (\textit{train mental capacities})\\
Train intellect, perception, thinking capacity and psyche, improve thinking speed, awake state and stress dominant traits.
 \end{itemize}
\end{frame}
%------------------------------------------------------
\begin{frame}
\frametitle{Physical Exercises}
\begin{itemize}
%General 
\item[-] Sun salutation
\item[-] \href{run:./Gravity_Glider.pdf}{\underline{Gravity glider}} 
 (\textit{stretch, relaxation})\\
Blood and lymph circulation, posture, equilibrium, coordination, \\
Comprehension.
 \end{itemize}
\end{frame}
%------------------------------------------------------
\begin{frame}
\frametitle{Posture}
\begin{itemize}
\item[-] \href{run:./Posture.pdf}{\underline{Posture}}
(\href{run:./Posture_Handout.pdf}{\underline{Handout}}
and 
(\href{run:./Posture_wNotes.pdf}{\underline{Instructor version}})  
\item[-] 
\href{run:./Active_Stance.pdf}{\underline{Active Stance}}
(\href{run:./Active_Stance_Handout.pdf}{\underline{Handout}}) 
(\textit{Posture, abductor training, stretch})
Stretches the shortened muscles and trains the abductors.


\end{itemize}
\end{frame}
%------------------------------------------------------
\begin{frame}
\frametitle{Equilibrium and Statolith exercises}
\begin{itemize}
%Statolith
\item[-] \href{run:./Equilibrium_Exercises.pdf}{\underline{Equilibrium exercises}} 
with theory
(\href{run:./Equilibrium_Exercises_Handout.pdf}{\underline{Student handout}})
 (\textit{equilibrium, activate statolith ball, proprioception, posture})\\
Train statolith and equilibrium, induce relaxation.


\item[-] \href{run:./Headstand.pdf}{\underline{Headstand and preeliminary exercises}} 
 (\textit{activate statolith ball, equilibrium, posture})\\
Train statolith and equilibrium, induce relaxation.

\item[-] \href{run:./Statolith_rotation.pdf}{\underline{Statolith rotation}} 
 (\textit{activate statolith ball, equilibrium, proprioception})\\
Train statolith and equilibrium.
\end{itemize}
\end{frame}
%------------------------------------------------------
\begin{frame}
\frametitle{Mindfulness and Meditation}
\begin{itemize}

\item[-] \href{run:./Mindfulness_Meditation.pdf}{\underline{Mindfulness and Meditation}} 
(\href{run:./Mindfulness_Meditation_Handout.pdf}{\underline{Handout}})
 (\textit{meditation theory, mindfulness})\\
% Presentation with big blue half of screen, no drawings

%Meditation
\item[-] \href{run:./Sun_Breathing.pdf}{\underline{Sun Breathing}} 
 (\textit{breathing, meditation, mood, inner smile})\\
Removing negative emotions, filling up with good energy.
\item[-] \href{run:./Sitting_Meditation.pdf}{\underline{Sitting Meditation}} 
 (\textit{breathing, meditation, mood})\\
The basic meditation.
% With nice hand drawn pictures, ok
\item[-] \href{run:./Body_Scan.pdf}{\underline{Body Scan}} 
 (\textit{bodily sensations, focus length and precision, meditation,})\\
Scanning the body, developing the focus of attention.

\item evtl Autogenic Training
\end{itemize}
\end{frame}

\end{document} 