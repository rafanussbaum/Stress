\documentclass[12pt]{article}
\setlength\parindent{0pt}
%\usepackage{gensymb} % for the \degree sign in front of C
%\usepackage{amsmath} % Ttypeset Math
%\usepackage[makeroom]{cancel} % To have the ability to use \cancel and \bcancel to cross out units
%Gummi|065|=)
\title{Heart Coherence Exercise}
\author{Heartmath Institute, Kilian Schmid, \\translated by Raphael Nussbaum}
\date{}
\begin{document}

\maketitle
\section{Introduction}
\begin{itemize}
\item Sit down comfortably, you can lean against the back rest, if it's easier for you. 
\item If you feel like it, you can close your eyes.
\item Try to relax your body despite the upright position. Relax your face muscles, your neck and shoulder muscles, your arms and hands, your legs and feet.
\end{itemize}
\section{Exercise}
I inhale and exhale completely relaxed. 

Thereby I feel how my abdominal wall rises and lowers. Completely relaxed rises and lowers.

\vspace{6mm}
Every exhale helps me to relax even further, to let go in the muscles, to let go of the thoughts.

\vspace{6mm}
I put a hand on the region of my heart and try to feel my heart beat.

\vspace{6mm}
I'm completely focussed with my thoughts on my heart.

\vspace{6mm}
I imagine the shape of my heart, it's size, it's color.

\vspace{6mm}
I calmly breathe into my hearth and out of it, in and out.

\vspace{6mm}
The inhale brings fresh oxygen to my heart. The exhale relaxes it and relieves it of waste products or eventual pain.

\vspace{6mm}
I feel how my attention is good for my heart.

\vspace{6mm}
It beat calmly, strongly and regularly.

\vspace{6mm}
One more time, I inhale into my heart and exhale out of it --- and gradually I start orienting myself again in the present.

\vspace{10mm}
Slowly I count from the number 7 down to 1 and are back in the here and now.

7 -- 6 -- 5 -- 4 -- 3 -- 2 -- 1 -- we slowly open our eyes again and stretch.
\end{document}
