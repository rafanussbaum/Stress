%------------------------------------------------------------


\begin{frame}
\frametitle{Song of Myself.  Walt Whitman: Leaves of Grass}
``\textit{I exist as I am, that is enough, \\
If no other in the world be aware I sit content,\\
And if each and all be aware I sit content.\\
One world is aware, and by the far the largest to me, and that is myself,\\
And whether I come to my own today or in ten thousand or ten million years,\\
I can cheerfully take it now, or with equal cheerfulness, I can wait.}''
\end{frame}
%------------------------------------------------------------
%------------------------------------------------------------
\section{Mindfulness}
\begin{frame}
\frametitle{Meditation}
\begin{itemize}
\item Most people, when they hear meditation, they think of \structure{Transcendental Meditation\texttrademark} or similar relaxation practices.
\item In these practices you  \structure{direct the attention to an object}, normally the perception of the inflow and outflow of the \structure{breath or mantra} (a special sound or string of words which is repeated in your thoughts.
\item All other mental activity is considered \structure{distraction}, which shouldn't be pursued.
\item Can induce deep states of \structure{calmness} and \structure{develop a steady focus}.
\item Also known as concentration meditation or \structure{single focus meditation}.   
\end{itemize}
\end{frame}
%------------------------------------------------------------
%------------------------------------------------------------
\begin{frame}
\frametitle{Mindfulness}
\begin{itemize}
\item Mindfulness, also known as \structure{insight meditation} is another major approach to meditation.
\item In the beginning you use single focus attention to cultivate calmness and steadiness.
\item After that by \structure{widening your object of observation}, you bring in an element of exploration.
\item \structure{Thoughts or feelings}, which surface, don't get ignored, suppressed, analysed or judged by their content.
\item They get consciously \structure{observed}, as much as possible \structure{without judgement}, how they appear as events in the field of awareness.
\end{itemize}
\end{frame}
%------------------------------------------------------------
%------------------------------------------------------------
\begin{frame}
\frametitle{Benefits}
\begin{itemize}
\item Ironically, this all--encompassing perception of the thoughts which appear and disappear in the mind, leads to \structure{less getting tangled up} in them.
\item The observer gets a \structure{deeper insight} into his reaction patterns to daily events and difficulties.
\item Through the distancing from the thoughts and observing them, you can get many insights what happens \structure{inside the mind}.
\item You can \structure{name the content} of the thought, the feelings linked to them, as well as the reaction to those feelings.
\end{itemize}
\end{frame}
%------------------------------------------------------------
%------------------------------------------------------------
\begin{frame}
\frametitle{Possible Insights}
\begin{itemize}
\item Intentions,
\item Fixations,
\item Preferences,
\item Antipathy,
\item Incoherences behind your own ideas,
\item Insights into what drives us,
\item How we perceive the world,
\item What we think,
\item Who we are,
\item Insight into our fears and wishes.
\end{itemize}
\end{frame}

%------------------------------------------------------------
%------------------------------------------------------------
\begin{frame}
\frametitle{The Key to Practice}
\begin{itemize}
\item The object of attention doesn't really matter,
\item The key is the \structure{quality of attention}, which we give every moment.
\item It's important that the attention is like \structure{quietly watching}, 
like observing without taking side, 
without passing judgement
or commenting the inner experiences.
\item It's a \structure{pure perception}  of the momentary experience, unclouded by judgements,
\item \structure{without changing}, censoring or intellectualizing it
\item or getting lost in never--ending thinking.
\end{itemize}
\end{frame}

%------------------------------------------------------------
%------------------------------------------------------------
\begin{frame}
\frametitle{A Connection to Life}
\begin{itemize}
\item This \structure{observing and investigating} approach to everything which gets created in this moment is the characteristic of Mindfulness and it's at the same time the biggest difference to other forms of meditation.
\item The goal of mindfulness is to \structure{be more aware and connected} with life and what happens in your body and your mind in this very moment.
\item If a thought or feeling is disturbing or we feel real physical \structure{pain} we try to resist the temptation to withdraw from unpleasant experience.
\item Instead we try to see and accept it as clearly as possible, just because it's \structure{present in the moment}. 
\end{itemize}
\end{frame}

%------------------------------------------------------------
%------------------------------------------------------------
\begin{frame}
\frametitle{Acceptance}
\begin{itemize}
\item Acceptance obviously doesn’t mean being passive or not caring.
\item On the contrary: By \structure{accepting the moment} totally the way it is we \structure{open ourselves} up to the experiences of life   
\item and we get more capable to \structure{react appropriately} to every situation that life presents us with.
\end{itemize}
\begin{block}{-- Friedrich Georg J\"unger, transl. from German}
``\textit{The stars reflect as well in dirty ponds, puddles and manure run--offs}''.
\end{block}
\end{frame}
%------------------------------------------------------------
%------------------------------------------------------------
\begin{frame}
\frametitle{Negative Feelings}
That means, we have to direct our attention even to 
\begin{itemize}
\item our tensions,
\item stress,
\item physical pain,
\end{itemize}
and our mental states, like
\begin{itemize}
\item fear,
\item anger,
\item frustration, 
\item insecurity,
\item and feeling of worthlessness.
\end{itemize}
When these states emerge, \structure{we have to meet them}, accept them and even bid them welcome.
\end{frame}
%------------------------------------------------------------
\begin{frame}
\frametitle{Why would I do that?}
\begin{itemize}
\item The acceptance of reality, the way it presents itself, if positive or negative, is the the \structure{first step towards changes} of this reality and our relation to it.
\item Avoiding meeting things upfront often leads to getting stuck somewhere and having difficulties to change us and to grow.
\end{itemize}
\end{frame}
%------------------------------------------------------------
\begin{frame}
\frametitle{Don't shoot the messenger!}
Pain is only the messenger, telling you very precisely and efficiently \structure{where something is not okay} and \structure{what not to do} because it would make the injury worse.

It's job is to \structure{get your attention} no matter the situation, to prevent further injury to the body. It is seriously good at it. And let's be honest, at times it is really hard to get our attention.

Listen to the message it has to deliver and \structure{acknowledge it}, thank the pain even for the job well done and reassure it that everything will be taken care of as good as possible. 

Often that helps with the \structure{perception of the pain}. It is still there, but it's manageable and not that needle sharp focussed thing as before.
\end{frame}

%------------------------------------------------------------
\begin{frame}
\frametitle{Our mind is like the ocean}
There's always \structure{waves on the ocean}, sometimes they are big, sometimes they are small. 

Many people think the goal of meditation is to avoid the waves, so that the surface gets calm and peaceful. That's a misleading assumption.

\begin{center}
You can't stop the waves, \\
but you can learn how to ride them.
\end{center}

\begin{block}{-- Origin disputed.}
``\textit{Watch your thoughts, they become words;\\
watch your words, they become actions;\\
watch your actions, they become habits;\\
watch your habits, they become character;\\
watch your character, for it becomes your destiny.}''
\end{block}
\end{frame}
%------------------------------------------------------------
\section{Practice}

\begin{frame}
\frametitle{How can you practice Mindfulness?}
There's \structure{two ways} to practice Mindfulness and both are important to integrate it in our live.
\begin{itemize}
\item In the \structure{formal practice} specific methods get applied. It helps us to stay present in the moment for a longer period of time.
\item Then there's the \structure{informal practice}. There it's all about remembering to be present during daily activities and to check once in a while, if you're really present. 
\end{itemize}
Mindfulness is more a \structure{mode of being} than a technique. Basically it's the question, if and to what degree do you want to \structure{be present at the unfolding of your own life}.  
\end{frame}

%------------------------------------------------------------
\subsection{Formal Practice}
\begin{frame}
\frametitle{Formal Practice}
There's many different types of formal meditation. 
\begin{itemize}
\item The so--called \structure{Body--Scan} (Perception of the whole body, reclined)
\item \structure{Sitting Meditation}, 
\item and different position of Hatha-Yoga, (they are executed in a slow, gentle and mindful manner). For instance the sun salutation. 
\end{itemize}
They are just different doors, which all lead into the same room.

%I will focus here on the Sitting Meditation, because it can be practised everywhere. 
\end{frame}

%------------------------------------------------------------

\begin{frame}
\frametitle{Common Traits}
\begin{itemize}
\item The \structure{mindfulness of breathing} is common to these three methods.
\item Every formal Meditation practise has a \structure{focus point} onto which the attention gets directed.
\item Normally it doesn't take long for the the \structure{mind to wander off}, no matter how motivated we were to not let it happen. Every single time this happens, we \structure{take notice} (as unjudgemental as possible) to where the mind has wandered, before we direct the mind \structure{back to the object of focus}.
\item We might notice that we were busy with a memory, a future fantasy or a bodily perception or that we were affected of feelings of boredom, impatience or fear.
\end{itemize}
\end{frame}

%------------------------------------------------------------
\subsection{Informal Practice}
\begin{frame}
\frametitle{Informal Practice}
The time and effort dedicated to formal practice helps to \structure{stay present} in very the moment in daily life. 

\begin{itemize}
\item  In theory, it is very easy, but it is hard to make it into a \structure{daily routine}. We have the habit to pass a good amount of our lives mindlessly on \structure{autopilot}, entangled with our own thoughts, and feelings, our moods and reaction to things. Only seldom we're able to see the bigger picture.
\item We are dealing with \structure{habits} here. It is hard to break habits. The best way to go about it is to practice and \structure{empower a new, alternate habit}.
\end{itemize}
\begin{block}{Quote Marc Aurel}
``\textit{ Inside of you! Inside of you is a source of goodness which never stops to bubble, as long and you don't stop digging.}''
\end{block}
\end{frame}
%------------------------------------------------------------

\begin{frame}
\frametitle{Life is your teacher}
Mindfulness is just a \structure{moment--to--moment awareness}, therefore every ``mundane'' task can be an opportunity to practise:
\begin{itemize}
\item  Eating
\item Brushing your teeth
\item Cleaning the dishes,
\item Walking
\item Driving
\item Tax declaration and cleaning up after your dog in the park...
\end{itemize}
The wonderful advantage is that we don't need to put additional time aside.
\end{frame}
%------------------------------------------------------------

\begin{frame}
\frametitle{Change in Consciousness}

\textbf{All what's needed is a change in consciousness. A flick of the switch from a habit--driven blind  mode of being to \structure{awake presence}.}

In this mode, daily \structure{life becomes practice, our meditation master and coach}.


\vspace{3.5cm}
Back to \href{run:./Exercises.pdf}{\underline{exercises}}.

\end{frame}

%------------------------------------------------------------


