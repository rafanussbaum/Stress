\documentclass[../main.tex]{subfiles}
\graphicspath{{\subfix{../images/}}}
\begin{document}

Remember the instructions for the first unit, in chapter~\ref{InstructionExercises}, page~\pageref{InstructionExercises}.
Continue to follow the guidelines there!

Does that cause stress in you?
Regulate this stress, by just dissolving it!

\section[Question]{Questions to Feelings and Emotions}

\begin{enumerate}
\item Can you describe, what the term ``feeling'' means?
\item How about ``emotion''?
\item What happens, when we're in a conflict of interest between our feeling and our thinking?
\item Happily, we have a cognitive brain, else \ldots
\item But there's as well a danger lying in this!
\item And when the emotional brain dominates?
\item Homeostasis is a state worth attaining. What is homeostasis?
\item We have different ways of access to our emotional brain. What are the direct access routes?
\item What's the importance of different perceptions of feelings?
  \vspace{4mm}
\item Music can help a lot.
\item Crying is often helpful, too.
  \vspace{4mm}
\item Our body indicates long suppressed feelings according to his possibilities.
\item How do you recognize a sincere, true smile?
\item Self--handicapping. Where did you experience it and where do you still experience it?
  What is strengthening your respect for yourself?
\item Do you know the endorphins?
\item Laughing has almost only advantages.
\item Laughing about yourself.
\item What are disadvantages of laughing?
\item Children are way more cheerful that adults. Can you express this in numbers?
  \vspace{4mm}
\item Ten euphoric and dysphoric feelings have been named by Caroll E. Izard. Can you name them? %Reseach, missing content
\item In older views, feelings were sorted in four main categories. %Reseach, missing content
\item  How do you distinguish between feelings and affects? %Reseach, missing content
\item It's a big aspiration to deal with your own feelings.
  Which ways to do so (mental/physical) do you have at your disposition?
  The question could be phrased: What is \ldots? Insert the blank yourself.
  \vspace{4mm}
\item What's the relation between emotions and stress?
\item Explain the autonomous nervous system.
  \vspace{4mm}
\item The neocortex has it's own special capacities.
  \vspace{4mm}
\item What does it mean to be mindful with your own body?
  \vspace{4mm}
\item According to Howard Gardner and Daniel Golemann, the emotional intelligence can be separated into five domains.
  Explain this!
    \vspace{4mm}
\item Coherent and incoherent frequency patterns are often caused by emotions.
  \vspace{4mm}
\item Instruct a heart coherence exercise according to David Servan--Schreiber!
  \vspace{4mm}
\item Distinguish between the two forms of empathy!
  \vspace{4mm}
\item Explain the six steps of the Dynamind technique to your study partner.
\item How can you use Dynamind to regulate stress?
  \vspace{4mm}
\item From the viewpoint of Dynamind, we can recognize, that stress situations often have to do with resisting.
  \vspace{4mm}
\item Simple breathing exercises can help regulate the stress.
  \vspace{4mm}
\item How do you feel about affirmations, like with the temporal tap?
  Regardless of how you see it, can you nevertheless instruct the temporal tap in a convincing manner?
\item Emotional bait was mentioned. One example mentioned are commercials and adds.
  There's plenty of others.
  Group them by type of effect and show possibilities to not become influenced by them.
  Are there some, to which your are completely or partially exposed to?
\end{enumerate}

\section[Tasks]{Tasks About Feelings and Emotions}

  \begin{enumerate}[label = \Alph*]
  \item Daily practice the heart coherence exercise for the next few weeks.
    If possible choose the same time of the day every time.
    Is something changing inside of you?
  \item Imagine, you would have the possibility to present to a group of employees in a company to the topic stress.
    You want these people to feel some of their own stress situation.
    You will probably tell a lot of the things you learned so far and then include physical exercises, like for instance the sun salutation.

    Can you get with these people into good contact, so that they listen to you attentively and ask questions?
    Or maybe succeed in leading them into a inspired and efficient group exercise, you already achieved a lot.

    Maybe you don't immediately find the access to those people,or you at least doubt that you will succeed in it.
    There are many possible reasons for this, which we won't look at here.

    But there are other type of communication, which can be called intermediate forms.
    They most often are invitations, prompts or other hints, which invigorate the interaction.
    One of these forms is a paper, that the audience have to fill out.

    Author a worksheet to the topic stress or another treated topic or subtopic.
    Ask meaningful questions, which can be answered or an option selected.
    Write up a detailed instruction to fill out the form, in order to avoid to have to explain a lot when you distribute the forms.
    Create enough space, maybe multiple pages make more sense than just one.

    Are your audience members supposed to write into a box or just cross off an answer choice.
    Are you looking for a Yes/No answer or do you want more differentiated answers, like for instance fitting/most of the time/more or less/barely/not at all or whatever you choose.
    Are you combining things, by making columns and labeling them reason/problem/effects/needs for action or however fits.
    Do you have an idea how to get your audience back, after collecting the forms?
    Was it a quiet solitary task, which concerns nobody else, or do you want them to selectively present parts of it to the group, or re they forming groups and will be talking about, or \ldots?

    % mail in
    If you want to go further than that, put together other ideas which could serve as this intermediate form of communication.
    Preparations of this kind are like canned food, which you can use later on in it's original or an adapted form.
    
    That is brain cell exercise and we all need that.
    Are you able to meet these tasks with a certain feeling of joy?
\end{enumerate}
\end{document}