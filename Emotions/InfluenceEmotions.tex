\documentclass[../main.tex]{subfiles}
\graphicspath{{\subfix{../images/}}}
\begin{document}


All beings who are equipped with a central nervous system have an emotional brain\index{brain!emotional} with it's seat in the limbic system.
During evolution, the cortex and the neocortex (the ``thinking'' brain) start to show up, humans found their upright gait and conscious processes of thinking got bootstrapped.
Various connections between the limbic system and the younger parts of the brain got simultaneously formed.
As a result of evolution, which constantly progresses, the processing of emotionally relevant stimuli mostly happens in the neocortex.
This one sorts out, compares, weights things up, and sends it's conclusion to the limbic system, which saves up emotional memories and associated reaction patterns based on that information.

The limbic system\index{;limbic system} has the position of the master and has the overview with veto right, given that a small part of the incoming information gets directly there.
Like that it can happen, that the limbic system initiates actions without consulting with the neocortex.
This happens when the limbic system identifies a state of exception or emergency.
In the case of an existential threat, it switches the neocortex off\index{neocortex}, because it works slower in it's manner of gauging  situations.

Reactions caused by the neocortex would be too late in case of a fight for survival.
Therefore, the limbic system directly takes action by making the human do primitive and archaic actions: fight, flight or reaction patterns concerned with the survival of the own person.

\epigraph{Intuition always is instantaneous, only thinking takes long.}{\textit{Bert Hellinger}}


\end{document}