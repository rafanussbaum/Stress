\documentclass[../main.tex]{subfiles}
\graphicspath{{\subfix{../images/}}}
\begin{document}


\mytextbox[0]{Translated from David Servant--Schreiber, Die neue Medizin der Emotionen.~\cite{MedEmotionen}
  \vspace{5mm}
   
  \noindent Our task consists in getting our heart under control.\index{emotions!taming}
  This way, we can learn how to tame our emotional brain und the other way around.
  At this point it is important to look closer at the word control.
  Talking about control, we mean a conscious examination.
  For instance, if I control a process I can tell if a system works in an optimal manner.
  Control is meant in the sense of a conscious handling with respect to ourselves.
  Taming the emotions then means approaching our true self.

  When we want to understand, how what is acting in us, we need to learn to read the predetermined code of our body.
  Our goal should be to avoid damage to ourselves and practicing listening very closely, when our body talks to us.
  Our drinking water goes through filter and sieves to get cleaned from debris before it comes out of the faucet, if we open the faucet.
  Why should we offer to our body all this debris, which waits for us after a storm warning?

  A conscious interaction with our body also means to take the responsibility and analyses what is influencing us right now.
  People who distinguish between what is good for them and what is bad for them and can put that differentiation into action successfully completed his apprenticeship as a worker;
  they are able to use the tools at their disposal with success.
  That what we have at our disposal, suddenly is fun using.
  If we people want to make good progress on the street of life and don't want to be ejected from our seats in unexpected turns
  we will need to make use of the gas pedal and the brakes.
Both have to be in impeccable shape and on the same level of performance, so that they can balance out depending on the needs of the situation.
}


\end{document}