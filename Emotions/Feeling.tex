\documentclass[../main.tex]{subfiles}
\graphicspath{{\subfix{../images/}}}
\begin{document}

\section{Definition}

It isn't possible to find a sufficient definition, which describes in a few words what feelings are.\index{feeling}
Generally speaking, it is about the perception of the mental and physical state and the inner movement (= emotion) of a human.

Neurology describes a feeling as the subjective experiencing of excitation and brain--chemical activity.
Love, hate, envy, jealousy, disgust, antipathy, fear, anxiety, sadness, feeling of security, hope are clear designation for the state of a feeling.
Feeling can trigger, reinforce or inhibit actions and increase or numb perception.

\epigraph{Jimmy Carter told me: ``I saw you on TV again on a horse. How come that you look that young?''. I replied to this: ``That's very easy, Jimmy, I'm only taking old horses''.}{\textit{Ronald Reagan}}

\section{A Feeling is Followed by the Emotion}

The media often uses the word ``emotions''.\index{emotion}
It seemingly has a way bigger signification than the word ``feeling''.
People seem to prefer talking about an emotion than talking about a feeling; this last one is often perceived as being more ``intimate''.
This perception is based on some truth: the feeling is the beginning of every emotion.

It's important in this context to clear up the signification of both words: first, we always perceive a feeling.
This feeling in turn causes an emotion.
First and foremost, we should look at how feelings and the reactions caused by them (=emotions) can be changed inside of ourselves.

In this way, we train the instrument that our body is;
an instrument which is able to produce most wonderful melodies, when we're willing to put in the practice to play it properly.
So in the same way that a musician will do their daily finger exercises and continues to develop an ever more intimate relation to his instrument,
in the same way we should invest in the relation to ourselves.
The musician is mastering his instrument, not the other way round.

Stress in all it's form it appears in is in many cases also responsible for feelings and emotions.
A feeling is almost always pleasant or uncomfortable.
That means, we perceive a feeling as positive or negative\index{feeling!positive or negative}.
Situations or people, which provoke pleasant feelings in us tend to attract us.
If people or situations cause uncomfortable feelings, we try to avoid them.

Every human being gets influenced by different things in their lives.
Depending on upbringing, culture, religion and media as well as personal experiences we learned more or less efficiently to express our feelings, to show them or to hide them\index{feelings!hiding}.

\epigraph{Anybody can become angry -- that is easy. But to be angry with the right person and to the right degree and at the right time and for the right purpose, and in the right way -- that is not within everybody's power and is not easy.}{\textit{Aristotle}}


At times, we have a vague ``bad feeling'', but we don't know how to handle it in the concrete situation.
There are also mixed feelings.
For instance with love and pity, the contribution of each of these feelings can be in very different ratios.
A feeling can get so strong that it overwhelms us (tears, anger).
Sometimes we feel afraid of certain feelings in us.
Due to this, we devalue these feelings very strongly.
In situations like these, we perceive ourselves as a plaything of our feelings.

Certain feelings provoke specific reactions, which we control anymore.
Intuitively we notice that we're in a cul-de-sac, in a situation, which seems to have no way out.
Generally we can think clearly, but in certain situations the ever important thinking processes fail us and stop working.
In this manner, we block our own freedom of possible actions and limit ourselves.

At times, we are caught in the discrepancy between our feeling processes and our thinking.
We experience this state as a stalemate.
We're lacking the capacity to make decisions, because both sides, the feelings and the thinking are demanding the fulfillment of their needs.
We lost the thread and any help we had in how to orient.
In these cases, more and more decisions fall by the wayside, which makes us more and more frustrated and unhappy.
When the feelings faded away and we acted according to our patterns of reactions, we only ``become really aware'' of what just happened.
We might get angry at ourselves, because we lost once more our way in a situation like this.
As a matter of fact, we forgot or unlearned to consult on both centers of perception at the time of finding a decision.
But first we have to learn, to get rid of our accumulated ballasts, so that both centers in our brain can start to work again optimally for us.

\section{When Both of Our Brains Aren't Getting Along}

\epigraph{A mind without feelings is inhumane. Feelings without a mind is stupidity.}{\textit{Egon Bahr}}

Both our brains,  \emph{the emotional and the cognitive brain} register\index{brain!emotional}\index{brain!cognitive} the information coming from outside world virtually simultaneously.
After that, they either collaborate well or fight over the control over thinking, feelings and behaviors.
The result of this interaction determines what we feel and how our relation to our surroundings and other people are.
The different forms of rivalry between the two different brains make us unhappy.

\section{What's More Important: Feeling or Thinking?}

\vspace{1cm}

\mytextbox[0]{Translated from~\cite{Berne}
  \vspace{5mm}
   
  \noindent The recipe

  The natural parent--I in the mother and the father is up to a certain degree biologically programmed and has a natural educative and simultaneously a protective character.
  At the very base of it, both parents only want the best for their child.
  They encourage the child in a manner which is meant to bring it success and well--being, according to their world--views and theories of life.
  They transmit it recipes, which they often took over from the grand--parents.
  Examples for the thinking patterns of the middle class are for instance: ``Work hard!'', ``Be a good girl!'', ``Save your money diligently'' and ``Be always on time''.
  At the same times, every family has their own motto, like ``Don't eat any food rich in starch'', ``Never sit down in a public restroom'' or ``masturbation will dry out the spine!''.
}

The cognitive brain controls the conscious attention as well as the skill of dampening the emotional reactions.
This control of the feelings by the cognitive brain avoids a tyranny of the feelings and a life which is completely dominated by instincts and reflexes.

The control of our feelings by the thinking process is a double--edged matter: if it is exerted too often, you might loose the capacity to hear the cries for help of the emotional brain.
People who experienced in their childhood that feelings aren't valid, learned to suppress them.\index{feeling!suppress}
A classical example is: ``Boys don't cry!''.
An exaggerated control of the feelings can lead to insensitivity.

A brain which forbids emotional information to exert an influence also causes other problems.
It will be hard to make decision when you don't have any preferences, no ``inner voices'' which comes from the heart or the belly --- those parts of our body which cause an ``irrational echo''.

\textbf{Goal: A permanent well--being on the base of an inner harmony.
  That's the case, when the emotional and cognitive brain complete each other.
  The emotional brain gives the direction, in which we want to create our life.
  The cognitive brain helps us to go forward in this direction as clever as possible.}

\textbf{Training: Excessively emotional situations can be encountered with cognitive elements and the other way round.}

When a feeling indeed takes the upper hand, it's a advisable to occupy yourself with rational facts.
Just simple arithmetic in your head can have huge effects.
If the cognitive strain is too much we can try to restore the equilibrium with music,
for instance with a slow passage from one of Mozart's pieces (example: Piano concert n. 21, Andante).\index{feeling!regulate}\index{cognitive strain!regulate}


\epigraph{Music expresses that which cannot be put into words and that which cannot remain silent.}{\textit{Victor Hugo}}

\section[Homeostasis]{Homeostasis: I'm Exactly Where I Want to Be in My Life.}

The limbic system\index{system!limbic} in the brain is the control central, which constantly receives information from the diverse body parts and reacts to them,
by controlling physiological equilibrium of the body:
respiration, heart rhythm, blood pressure, appetite, sleep, libido, excretion of hormones, immune system.
The responsibility of the limbic brain is to keep the different functions in a dynamic equilibrium.
Claude Ernanrd, a French researcher and father of modern physiology described that state at the end of the 19th century as \emph{homeostasis}.\index{homeostasis}

\textbf{Homeostasis described the equilibrium between all physiological body function, so the relation of blood pressure, body temperature, body temperature, pH value of the blood, etc.}


From this perspective, our emotions are nothing else but the conscious experiencing of a big interplay of physiological reactions.
They supervise the activity of the biological systems of the our body and adapt those to the needs of the inner and outer surrounding.

The emotional brain knows our body much better than our cognitive brain.
Therefore it is much easier to reach our feelings over our body than over language.
The main emphasis lies on the movement of the body.
Further examples are Gymnastics, Dynamind, Jin Shin Jyutsu, nutrition and acupuncture.\index{emotions!regulating}


Once again: Emotional relations have a strong physical component; we experience them in a physical manner.
That's why this approach to the emotional brain is more direct and efficient than the one over thinking and language.

\vspace{1cm}

\mytextbox[0]{Bruce Lipton, translated from~\cite{LiptonCell}
  \vspace{5mm}
   
  \noindent If you believe, that the genes determine your life and you know, that you have no influence over which genes you received at your conception, then you have all the right in the world to feel as a victim of genetics.
  ``I have no fault that I work so slowly, that I can't keep appointments --- it happens to be my genetic disposition!''

  Since the beginning of the genetic age we have been indoctrinated, that we're under the power of our genes.
  The world is full of people who live in the fear that their genes will one day turn against them.
  How many people live in the fear of being walking time bombs --- they only wait that the cancer will explode into their life, the way it did in the lives of their mother, sister or aunt.

  Millions of people think their fragile health isn't the result of a combination of mental, physical, emotional or spiritual reasons, but attribute it to the insufficient biochemistry of their body.

  Are your kids being naughty?
  More and more parents first reaction is to medication to therapy a ``chemical disequilibrium'',
  instead of taking it on themselves to get to the ground of what is happening in the body, brain and soul of their child.

  There's no doubts that certain diseases, like the Huntington disease, thalassemia major und cystic fibrosis are due to a genetic defect.
  But less than two percent of the population are affected by such diseases.
  By far the bigger part of humanity gets born with genes which would allow them a healthy and happy life.
}

\section{The Power of the Physical Component}

When we're sad, counseling words can express compassion.
More effective and healing on the other hand is a big good hug.
Our language accounts for these bodily reactions with expressions like ``holding the hand'', ``cross your fingers''.
In soccer, the mutual hugs after scoring a goal are legendary.
Lovers feel happy all around, because they permanently communicate on a physical level.
The bigger is the emotional crash, when these physical affections suddenly stop.

\epigraph{The most difficult thing in life is to get the heart and the brain to collaborate. In my case, they are not even on friendly terms.}{/textit{Woody Allen}}

True physical happiness is therefore only possible, when it comes from your heart.
In the animal kingdom you can see how important physical contact is.
For Koalas hugs are the first and foremost survival strategy.
Given that the young Koalas grow up on the back of their bother, hugs are vital for them.
Koalas are very tender with each other and ahve a well develloped social life.
They stay closely connected for their whole life.

Our posture will in most cases expresses our feelings and is able to support a feeling.
When we're standing in a relaxed manner and are in equilibrium, we exactly spend the amount of energy, which is needed to counteract earth's gravity
and which allows us to fully perceive our whole organism.

In this case, we can orient 

\end{document}