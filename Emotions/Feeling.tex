\documentclass[../main.tex]{subfiles}
\graphicspath{{\subfix{../images/}}}
\begin{document}

\section{Definition}

It isn't possible to find a sufficient defnition, which describes in a few words what feelings are.\index{feeling}
Generally speaking, it is about the perception of the mental and physical state and the inner movement (= emotion) of a human.

Neurology describes a feeling as the subjective experiencing of excitation and brain--chemical activity.
Love, hate, envy, jealousy, disgust, antipathy, fear, anxiety, sadnes, feeling of security, hope are clear designiation for the state of a feeling.
Feeling can trigger, reinforce or inhibit actions and increase or numb perception.

\epigraph{Jimmy Carter told me: ``I saw you on TV again on a horse. How come that you look that young?''. I replied to this: ``That's very easy, Jimmy, I'm only taking old horses''.}{\textit{Ronald Reagan}}

\section{A Feeling is Followed by the Emotion}

The media often uses the word ``emotions''.\index{emotion}
It seemingly has a way bigger signification than the word ``feeling''.
People seem to prefer talking about an emotion than talking about a feeling; this last one is often perceived as being more ``intimate''.
This perception is based on some truth: the feeling is the beginning of every emotion.

It's important in this context to clear up the signification of both words: first, we always perceive a feeling.
This feeling in turn causes an emotion.
First and foremost, we should look at how feelings and the reactions caused by them (=emotions) can be changed inside of ourselves.

In this way, we train the instrument that our body is;
an instrument which is able to produce most wonderful melodies, when we're willing to put in the practice to play it properly.
So in the same way that a musician will do their daily finger exercises and continues to develop an ever more intimate relation to his instrument,
in the same way we should invest in the relation to ourselves.
The musician is mastering his instrument, not the other way round.

Stress in all it's form it appears in is in many cases also responsible for feelings and emotions.
A feeling is almost always pleasant or uncomfortable.
That means, we perceive a feeling as positive or negattive\index{feeling:positive or negative}.
Situations or people, which provoque pleasant feelings in us tend to attract us.
If people or situations cause uncomfortable feelings, we try to avoid them.

Every human being gets influenced by different things in their lives.
Depending on upbringing, culture, religion and media as well as personal experiences we learned more or less efficiently to express our feelings, to show them or to hide them.



\end{document}