\documentclass[../main.tex]{subfiles}
\graphicspath{{\subfix{../images/}}}
\begin{document}



\mytextbox[0]{Translated from Howard Gardner and Daniel Golemann
  \vspace{5mm}
   
  \noindent In a society, which is controlled by the thinking brain, little importance is given to the emotional education.
  With increase and escalation of violence it will become more and more important to look at the role and possible ways of taking control of the emotions.
  Neglecting emotional intelligence\index{intelligence!emotional} leads to increasing deficits, especially in the domains of empathy and gratitude,
  the readiness to take responsibility and also the capacity of dealing with conflicts.
  It's important to further study emotional intelligence, so that it finds it's way into training programs and a fixed place in education and also in all learning processes.
  We need concepts, which show specific attempts for a solution for the existing problems of our time  on the base of heart intelligence or of the ethical responsibility.

  \begin{description}
  \item[1. Self--perception:] To recognize the own emotions and being able to name them.
  \item[2. Self--management:] Regulation of emotions. Stop escalations, dissolve unwanted feelings and reinforce positive emotions.
  \item[3. Self--motivation:] To put emotions into action and into service to a goal. Self--motivation, self--control and being the master of yourself are being part of this point.
  \item[4. Empathy:]\index{empathy} Being able to put yourself into somebody else's shoes, put also to be able to keep your emotional distance regardless.
  \item[5. Capable of relationships:] Being able to cope with the emotions of other people.
  \end{description}

  These capacities are essential for leadership qualities, success in the job and private life and for a fulfilled and harmonious life.
  The emotional intelligence is way more of an indication for success in life than a high IQ.
}


\end{document}