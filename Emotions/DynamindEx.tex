\documentclass[../main.tex]{subfiles}
\graphicspath{{\subfix{../images/}}}
\begin{document}


\label{Ex:Dynamind}\index{exercises!dynamind}
Many of the physical, mental and emotional problems show themselves as \emph{tensions in our body}.
If we are able to \emph{dissolve these tension layer by layer},\index{tension!dissolve}\index{symptom!tension} we have found an efficient tool for the stress regulation.

The Dynamind technique (DMT) origins form ancient Hawaiian tradition.
With the help of a special mixture of words, touches and breathing techniques
problems\index{symptom!problem (physical, emotional or psychological)} of various type could be dissolved or lessened.
DMT can be applies by itself or in combination with traditional medicine or complementary or alternative medicine treatments.
It can be used with children and adults, even animals react very positively to DMT.

\mytextbox[0]{According to Serge Kahili King~\cite{KahiliKing} you use DMT in six steps:
  \vspace{5mm}
   
  \noindent 
\begin{enumerate}
\item \textbf{Choose a physical, emotional or psychological problem on which you want to work.
    First estimate the intensity of your problem, on a scale from 0 to 10. For example: ``My headache burdens me with an intensity of 8.''}

  Therefore use an intensity scale to express, how badly you are burdened by the chosen problem in the moment.
\item \textbf{Put your hands together in front of you, so that all finger tips touch and the the thumbs and index finger form a triangle.}

  This position is also used in meditation techniques. It almost is as if you would hold a little earth in your hands.
\item \textbf{Make the following statement, aloud or silently:}\newline
  \textbf{``I have a problem with \ldots''} \newline
  \textbf{``I have a problem with \dots, and that can change!''} \newline
    \textbf{``I have a problem with \ldots, and  I want that problem to go away.!''} \newline

    By doing so, you consciously acknowledge the problem and accept it.
    State as precise as possible the type on symptom, it's intensity , the location and the feeling that it triggers.
    You state your expectation that the problem is able to dissolve and reinforce that with an order.
    It doesn't matter, who is going to execute this order, if this is your body, your unconsciousness, the brain or something else.

  \item \textbf{With two or three fingers tap these points 7 times each: \\
      the center of your chest; \\
      the outer area between the thumb and index finger of both hands; \\
      the bone at the base of your neck.}

To touch your body anywhere creates an energetic response which has effects on the bodily, mental and emotional systems.
If you touch your body on specific places in a specific way, then it will trigger a response which is at the same time energizing and relaxing.

The choice of the point to touch is not random
\begin{itemize}
\item Touching the region around the thymus gland\index{thymus gland} below the breast bone (sternum)
  can appease anxiety and fears (``My chest is constricted'').
  This in turn relaxes the chest \index{effect!chest muscles, relax} and lung muscles\index{effect!lung muscle, relax}  and stimulate the immune system\index{effect!immune system, stimulate} .
  During stress, the thymus gland contracts and blocks the transport of leukocytes\index{leukocytes!transport} (white blood cells).
\item At the hands, meridian points\index{meridian!points!stimulation}, which can appease head aches \index{symptom!head ache} and which are said to
  revitalize the whole body\index{effect!body!revitalize}.
\item The slightly protruding bone above the first thoracic vertebra is said to have a relaxing effect on the whole upper body\index{effect!upper body, relax},
  the spine\index{effect!spine, relax} and the hip region\index{effect!hip region, relax}.
\end{itemize}

Find out for yourself, if tapping seven times is the right type of touch for you.
Maybe a slight vibration or a longer touch (without pressure) or a longer touch with humming (relaxes the bones and stimulates the circulation!)
\index{effect!bones, relax}\index{effect!blood!circulation, stimulate} corresponds better to you.
\item \textbf{Inhale with your attention focused on the top of your head; exhale with your
    attention on your toes.}

  This breathing technique (Piko Piko: from center to center) has a vitalizing and relaxing effect. \index{effect!body!vitalize and relax}
  When cells have a lack of oxygen, it can cause pain.\index{effect!blood!oxygenate}
\item \textbf{Again say out loud, how strong the problem is still being a burden for you, on a scale from 0 to 10.}

  After this sequence you check again if the symptoms partially or entirely disappeared. You can choose to either stop the process or to repeat the process.
  You might want to adapt the process, for instance by changing the intention or when the type or position of the symptom changed.
\end{enumerate}
}

Repeat this exercise as much as you like, hopefully until the symptoms dissolved.
Always have a little break between two cycles.


\end{document}