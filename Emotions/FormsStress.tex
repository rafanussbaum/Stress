\documentclass[../main.tex]{subfiles}
\graphicspath{{\subfix{../images/}}}
\begin{document}

Stress situations as we humans experience them, can be classified into three categories.
As we just have seen for Dynamind, we catch ourselves again at the same action: \emph{resisting the stress.}

\section{Physical Stress}
\index{stress!physical}Press your right hand against the left hand. What happens? ---
Almost certainly you didn't follow the instructions but pressed with your left hand against the right one.
This is a pattern of behaviors, the pattern of resistance.\index{resistance}
If I always resist, then this will lead to tensions in the muscles, the organs, the blood and the chemistry and can lead to aggression.\index{symptom!muscle tension}

Examples:
\begin{itemize}
\item Push your chin forward and then imagine to writing a love letter. Are you able to do it? Probably not.
  The tension in your lower jaw doesn't allow anything to flow anymore, which leads to your feeling for love getting affected.
\item Push your tongue in a tense way against a resistance (roof of the mouth, teeth,\ldots).
    Can you feel, how your breath doesn't flow anymore, and how your focus dwindles?
  \item Make your hands into fists while walking, feel into your breath and your emotions. Are you relaxed?
    Can you easily walk? Do you have good feelings?
  \end{itemize}

  \section{Emotional  Stress}

  \index{stress!emotional}There is the one thing happening, which I didn't want to happen.
  Worries, anger or fear come up, which in turn leads to more tension.
  If somebody asks you then how you're doing you might answer ``Fine, well'' or then ``Darn, none of your business''.
  Can you fell, which of these two statements brings up resistance and which one allows flowing?

  This stressor doesn't come from the outside, you produce it yourself.
  Test it!
  Sit down and relax, imagine your fear in a specific situation.
  Just fell into it, take it in, you will feel the tensions starting to form.
  Then consciously exhale, let the breath flow and your muscles will change.

  \section{Mental Stress}

  Mental resistance\index{Stress!mental}, criticizing, refusing ideas, always know everything better, always first see the negative aspects, \ldots ---
  these types of stressors don't come from the outside world either.
  It's on us to remove them by ourselves, hopefully.

  \epigraph{There are two types of sinners: the so--called mean people and the people who wish them bad things to happen. The second group is called the righteous ones.}{\textit{Bert Hellinger}}
\end{document}