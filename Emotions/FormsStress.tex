\documentclass[../main.tex]{subfiles}
\graphicspath{{\subfix{../images/}}}
\begin{document}

Stress situations as we humans experience them, can be classified into three categories.
As we just have seen for Dynamind, we catch ourselves again at the same action: \emph{resisting the stress.}

\subsection{Physical Stress}
\index{stress!physical}Press your right hand against the left hand. What happens? ---
Almost certainly you didn't follow the instructions but pressed with your left hand against the right one.
This is a pattern of behaviors, the pattern of resistance.\index{resistance}
If I always resist, then this will lead to tensions in the muscles, the organs, the blood and the chemistry and can lead to aggressions.

Examples:
\begin{itemize}
\item Push your chin forward and then imagine to writing a love letter. Are you able to do it? Probably not.
  The tension in yor lower jaw doesn't allow anything to flow anymore, which leads to your feeling for love getting affected.
\item Push your tongue in a tense way against a resistance (roof of the mouth, teeth,\ldots).
    Can you feel, how your breath doesn't flow anymore, and how your focus dwindles?
  \item Make your hands into fists while walking, feel into your breath and your emotions. Are you relaxed?
    Can you easily walk? Do you have good feelings?
  \end{itemize}

  \subsection{Emotional  Stress}

  \index{stress!emotional}There is the one thig happening, which I didn't want to happen.
  Worries, anger or fear come up,
  
\end{document}