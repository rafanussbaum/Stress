\documentclass[../main.tex]{subfiles}
\graphicspath{{\subfix{../images/}}}
\begin{document}

Feelings develop in an interplay of different components in our body.
The neurotransmitter dopamine reinforces the learning effect from positive experiences.
The early warning system in our brain, the Anterior Cingulate Cortex (ACC)\index{anterior cingulate cortex}\index{system!early warning}
works like a sixth sense and learns to permanently anticipate difficulties and to warn us correspondingly.
The amygdala\index{amygdala} processes feelings.
Even pictures which have been perceived subconsciously will thoroughly influence how feelings are formed.
Fear is localized there.\index{fear}
In the process of learning fear, in two parts of our brain messenger chemicals play an essential role:
The hormone which releases gastrine as well as the neurotransmitter GRP; the gene which is able to dampen fear.

\section{Our Longing for the Always Nice Feelings}


\setlength\epigraphwidth{.7\textwidth}

\epigraph{The warmth \\ \vspace{5mm}
  In spring, kids come to the forest and find a snow man.
  The snow man is crying and contorts his face, because the sun is shining.
  The kids caress him, because they want to help him.
  But their warm hands makes him melt his size even more.

  When are looking for him again later on, they find tender spring blossoms where he was standing.}{\textit{Bert Hellinger}}
\setlength\epigraphwidth{.4\textwidth}

Music is the door to our feelings.
Every person can prescribe themselves their daily dose of feelings.
Our favorite music is capable to to evoke beautiful memories in the different regions of our cerebral cortex.
Music always played an important role in the lives of humans.
By the means of the memory storage i9n our brain, which goes back to the times of our ancestors,
we recorded \emph{rituals and ancient sounds}, which nourish our thinking processes.
All central achievements of society have been accompanied with music.
The ancient musical repertoire of a culture still connects people to their ancestors and gives the security and closeness that we crave.
The healing aspect of music is successfully applied in pain therapy.

When dark feelings and thoughts control a human being, he's prone to fall prey to the fast happiness.
Alcohol is promising fast happiness, at least for a short amount of time.
Most often after  there follows a crash into sadness and depression.

\epigraph{Sorrows don't drown in alcohol. They can swim.}{\textit{Heinz R\"uhmann}}

\section{The Chemical Viscous Circle of Addiction}

The driving force behind the deadly circle of addiction\index{addiction} is pure brain chemistry.
A intricate system of dopamines, gamma--alpha--butyroic acids, peptides resembling opium and serotonin is needed to excrete a feeling of happiness.\index{happiness}
Further information to this topic follows in the module medicine
(Part~\ref{part:medicine} starting on page~\pageref{part:medicine}). % reference to medicine

\section{Crying as the Ideal Painkiller}

Tears\index{crying}\index{tears} are the most obvious signs of our emotional life.
When feelings overwhelm us, crying, but as well laughing give us big relief.
It has an effect of soothing pain, similar to morphine.
Tears, coming form crying or laughing relaxes the muscles and liquefies pain.
With the help of tears, homemade stress can be dissipated.
Tears often often express a wish for fulfillment, closeness or consolation.
Especially in religions, tears are seen as a sign of authenticity and integrity.
After all, crying is an almost exclusively human skill.

But not all tears are alike.
When an object hits an eye or cries while cutting onions, then those are tears due to a stimulus.
They are composed totally differently from the tears spilled over emotions.


\end{document}