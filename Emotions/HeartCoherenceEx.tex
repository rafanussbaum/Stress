\documentclass[../main.tex]{subfiles}
\graphicspath{{\subfix{../images/}}}
\begin{document}
\label{Ex:HeartCoherence}
\mytextbox[0]{A special exercise in three steps (from David Servan-Schreiber):\index{exercises!heart coherence}\index{heart!coherence exercise}
  \vspace{5mm}
   
  \noindent 
\begin{enumerate}
\item \textbf{Inhale twice slowly and deeply. Execute each breath consciously, up to the end of the exhale. Take a break of a few seconds before the next inhale.}

  Like in all relaxation techniques, the attention gets focus towards the inside.
  The one who enters has to learn first to withdraw from the outer world, to commit yourself to pushing for a few minutes all sorrows beside.
  In the center is your breath --- the rest can wait.
  This procedure gives the brain and the heart the time needed to find their inner agreement.

  \item \textbf{After ten to fifteen seconds of stabilization, focus on the region of your heart}

    Imagine that you breathe through your heart, or if this is too hard to imagine, then through the central chest region.
    While you're slowly and deeply breathing, perceive with all your sense how the inhales and exhales stream through this region.
    The inhale provides the vital oxygen, the exhale blows out all the unnecessary waste products.

  \item \textbf{Get familiar with the perception of warmth and expansion which fill up the chest and support it in your thoughts with your breath.}

    The heart is very sensitive and needs a lot of care and affection, also your own.
    Even if they are coming in from the outside, they are of little use if we're unable to accept them.
    That means for our physical training that we will need a lot of patience.
    In order to change, our body needs time and continuous and gentle investment.
\end{enumerate}
}

According to David Servan--Schreiber, the heart is in this inner dialogue something like a bridge to our ``belly-me''.
In a certain way it acts like a translator for the emotional brain, which has through that almost instantaneous communication.

Get clarity if the emotional brain is urging in a different direction, than the one for which you decided rationally.\index{effect!clarity}
In that case, try to calm down the emotional brain on other levels, so that will be no conflict with the cognitive brain.
This would influence our capacity for thinking and  lead to a physiological chaos, which in the end leads to chronic energy deficit.

\epigraph{When you sit with a nice girl for two hours it seems like two minutes. When you sit on a hot stove for two minutes, it seems like two hours. That's relativity.}{\textit{Albert Einstein}}
\end{document}