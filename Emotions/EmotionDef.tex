\documentclass[../main.tex]{subfiles}
\graphicspath{{\subfix{../images/}}}
\begin{document}





\section{Classification of Emotions}
% Here is the gap p 23 & 24

Older theories put the feelings into four main categories: fear, anger, joy and sadness.

\epigraph{There's nothing more silent than a loaded cannon.}{\textit{Heinrich Heine}}

Another system of classification is to distinguish between feelings and affects.\index{affect}
According to this way of classifying emotions, feelings are the emotions which unite us, and affects are emotions which separate us.
in the category of feelings would then be: love, friendship, compassion, connection and feeling of community.
Under affects go: envy, hate, fear, jealousy, inferiority feeling and a feeling of guilt.

Studies of different cultures showed, that feelings aren't necessarily identical with the shown emotion.
Cultural factors influence the observable expression of emotions in contrast to the inner, felt reality of the feeling,
how a feeling is individually experienced in their inner world.

In each culture exists many basic emotions.\index{emotion!basic}
They are closely linked to the simultaneously occurring neuronal processes.
Nowadays the assumption is made, that the fundamental emotions are in a narrow connection with the corresponding facial expression.
In a study across many different cultures has for instance always attributed anger to lowering and contracting the eye brows,
pulling the eyes into slits and squeezing the mouth together.
The mimic expression of a basic emotion is looked at as being universal.

\section{Ways to Regulate Emotions}

Emotions often get triggered by situations, people, places or memories and often are focus on a specific object.\index{emotions!regulate}
A certain person can trigger wonderful joy, or then debilitating fear.
Cognition's like music, smells or pictures can also trigger feelings or change them.
For instance, smells often trigger memories of past feelings.
Emotions are get triggered very quickly or then build themselves up slowly.
They are not able to be directly influenced.
But we can learn, how to deal with our emotions.

\epigraph{An eye for an eye will only leave the whole world blind.}{\textit{Mahatma Gandhi}}

Our behavior has a certain feedback on our feelings.
In this sense, there are 3 ways for regulation open to us:
We can reinforce our emotions through thoughts and actions,
slowly changing them up
or let them slowly subside.
On a physical level are for instance \emph{meditative exercises, sleep, hunger, feeling of being full, being tense, jogging or yoga} influencing our emotions.

Emotions have a strong influence on the performance of a human, without a doubt.
In this sense, the so--called \emph{emotional intelligence}\index{intelligence!emotional} gets more and more in the focus of our interests.
The validity of the construct ``emotional intelligence'' is nevertheless still very much controversial in
empirical\footnote{empirical: coming from experience, observations and the experiment.} psychology.

\section{The Physiology of the Conflict Scale}

An animal has to be in a state of high action potential in case of imminent danger.\index{action potential}
A vegetative chain reaction gets triggered by the excretion of adrenaline by the adrenals in order to provide the energy necessary.
This leads in turn to an increase in blood pressure, and blood sugar levels and the tension in the muscles.

The aim is to provide a maximum amount of energy for survival.
This inherited stress reaction is causing most stress problems for modern humans.
We're pretty much all the time on the flight from the tiger.
Thai means, we're trapped constantly in a trigger--reaction scheme, into which there's a permanent input of too many stimuli, without the vital breaks.
We often have not enough possibility of arranging things ourselves.
The inner level of tension is constantly being increased and we perceive that as uncomfortable, as a prison of stress.\index{tension}

With our biological configuration we react the same way as our ancestors in the Savannah.
The exterior world has changed, became more complex and more confusing in it's layout.
Surprising escalation of violence, completely out of balance, as they appear more and more often seem to be a consequence of the biological mechanisms.
These escalations of violence seem to be triggered by the perception of existential threats.\index{anger reaction}

In this context, the similarity of a stimulus with an earlier experienced situation is decisive for the initial triggering of these archaic reactions.
A real existing danger is not at all decisive.
People get more and more often in a state of emergency, let themselves being carried away to outbreaks of anger and violent actions,
without the cognitive brain having a chance to examine these actions in more detail.\index{viscous circle!anger reaction}

\epigraph{A sensitive person is a person who, because they have corns himself, always treads on other people's toes.}{\textit{Oscar Wilde}}

The more or less fast escalation of rage and anger in situations of conflict is way more often to be observed  than the above mentioned instantaneous derailment of the emotions.
This process is also able to be explained on a physiological level.
By angry or hostile reactions to an event will the limbic system be put in a state of excitation through the excretion of certain chemicals.
This state continues hours after the original events which triggered that state ended.
In this time, the threshold for a next anger reaction is lowered.

For instance the irritated atmosphere after a day at work with a lot of anger continues and is able to to trigger us to get angry in an immoderate way
over the driver in front of us in traffic or a fly on the wall.
If acute anger reactions follow, more stress hormones get released.
The excitation level of the limbic system continues to increase in this manner.
Given that these reactions Durer for a while, the effects overlap.
Anger escalates due to the lowered tolerance threshold faster and faster, up to the point where the emotional brain declares a state of emergency.
That way, a latent elevated adrenaline level gets maintained in our body, and frustration and depression is then not far.

People who mismanage themselves over an extended period of time, more and more withdraw in their own snail shell.
The fears of change then grow higher and higher around that snail shell.
The affected person then fears loosing himself, and collapsing.
Nobody wants to volunteer loosing face.
But many people afflicted by that fear don't realize that they are unable to loose face, but just a mask.\index{loosing face}
And even more a mask which isn't even fitting to them in the first place.
Behind that, the \emph{freedom} is growing.
The goal is to reach equilibrium; in this case by \emph{taming the emotions}.

\vspace{1cm}

\mytextbox[0]{Translated from Bert Hellinger
  \vspace{5mm}
   
  \noindent The clearing
  \vspace{5mm}

  Somebody lives in a little house and during the years a lot of clutter accumulates in the rooms.
  Many guests brought their things, and when they push along they left many suitcases there in the little house.
  It's almost as if they are still there, even if they left a a long while ago.

  The owner also accumulated things and those things stayed in the house.
  Nothing should be over or get lost.
  Memories are also attached to broken things, so they stay and take the space away from better things.

  Only when the owner almost asphyxiates, he starts to clean up.
  He starts with his books.
  Does he still want to look at the old pictures and understand the lessons and stories of others?
  He gets out of his house, what is worked off a long time ago and in his rooms it gets bright and airy once again.

  Then he starts opening suitcases of others and looks if he can find something which he still can put to use.
  Thereby he discovers a few valuable things and put them aside.
  The rest he gets out of the house.

  He throws the old stuff in a deep ditch, carefully covers it with soil and then he sows grass over it.
}







\end{document}