\documentclass[../main.tex]{subfiles}
\graphicspath{{\subfix{../images/}}}
\begin{document}

Under the term empathy\index{empathy} we understand \emph{the readiness and the capacity to feel yourself into the attitudes of other people}.
It is one of the capacities, which make a human capable of living and surviving.
During interaction with others we try to draw conclusions from the verbal and nonverbal signals.
We try to get a picture of the intentions, feelings and interests of the other person.
As a reaction, we adapt our reactions and actions to the probably expected reactions and actions of our interaction partner.

Successful communication and actions are essentially dependent on the capacity, to put yourself into other people, to read them.
The spice of empathy make personal relations, professional success and the path up the ladder into a management position ripen and and flourish.

We distinguish tow types of empathy: inductive and deductive empathy.\index{empathy!deductive}

The form of \emph{inductive empathy} \index{empathy!inductive} shows in a person, who radiates energy and warmth around themselves.
A human with true empathy has the ability to encounter their surroundings and the totality of nature with a positive empathy.

The form of the \emph{deductive empathy} shows up in a person who cannot recognize the part about feeling into another person.
They react with resistance.
The types of reactions and actions coming from the different types of empathy determine also the social behavior\index{social behavior}.

People who trained themselves in inductive empathy have no difficulty to insert themselves into the net of living together, without being in any way peculiar.
They are lovable and therefore popular.

The sign of a human having a deductive empathy are as clear.
These people tend towards being eye--catching.
Their behavior shows a certain egocentricity\index{egocentric} up to a egomaniac\index{egomaniac} behavior.
These people are also closed off and unapproachable and have the tendency to be a difficulty in every possibility and not to see a possibility in every difficulty.

\epigraph{Insight is gained by unison.}{\textit{Bert Hellinger}}
\end{document}