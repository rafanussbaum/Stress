\documentclass[../main.tex]{subfiles}
\graphicspath{{\subfix{../images/}}}
\begin{document}

\mytextbox[0]{Gertrud Hirschi, translated from ``Yoga ganz einfach''~\cite{YogaEinfach}:
  \vspace{5mm}
   
  \noindent \textbf{Respiration}

  The respiration\index{respiration} is of biggest importance for our health and well-being.
  A very nice property of the breath is the fact that we can influence our breath\index{breath} and therefore also our health and well-being.
  When we focus, are happy, bored or in a hurry, exercise, are angry, getting upset or are sad, our breathing\index{breathing!feelings} goes along with the feeling.
  It always adapts.
  It is also connected with the unconscious bodily functions.
  Breathing, depending on how it is practiced, increases power and balances out deficits and excesses.

  Conscious breathing helps to better perceive the own body and through that the own ``I''.
  It's the instrument that you direct precisely and consciously through your body.
  At the same time, conscious breathing leads you into a a world of love, given that you're connected to everything and everybody with your breath.

  For the purpose of regulating stress is breathing the most immediately active tool.\newline
  You can use it everywhere --- and it's free.

  Generally: The inhale builds up, gives power and straightens up.
   The exhale relaxes, calms down and dissolves.
  \begin{itemize}
  \item  If you want to refresh, build yourself up --- increase the inhale and stretch the break afterwards
  \item If you want to relax and calm down --- then slow down and deepen your breathing
    \item If you want to restore the inner balance --- pay attention to the regularity of the inhale and exhale
    \end{itemize}

    Don't force anything, just observe the rhythm in a relaxed way and when the thoughts go elsewhere, just gently bring them back to the breath.

    According to newest scientific research, an optimal breathing rhythm would be \newline
    6 breaths a minute, therefore 10 seconds per breath, distributed in
    \begin{tabular}{l l}
      1 -- 4 & inhale \\
      5 -- 8 & exhale \\
      9 -- 10 & break
    \end{tabular}

    \subsubsection{Tips for Breathing}

    \begin{itemize}
    \item \textbf{Shallow or only superficial breathing (upper chest area, clavicular breathing):}
      \index{breathing!clavicular}\index{breathing!shallow}\index{symptom!breath!shallow}

      Try pushing with both hands on your belly and let afterwards the inhale freely flow into your body.
      You can achieve the same affect after your belly inhale by pressing your hands on the ribs (on the side)
      or onto the tip of he lungs (upper space).
      And you already did a perfect full breath, isn't that easy?
      This technique allows you to lead your breath to any part of your body. Just try it out.
      If you have pain somewhere,\index{symptom!pain} lead your
      breath consciously to this part and feel how that feels good.
      You are becoming your own healer.

    \item \textbf{Not well being able to exhale:}

      There is also a simple exercise for that: While exhale , intone a long sound like ``sh'' in ship.
      Your exhale will be longer.

      To really optimize your inhale and exhale you can do a ``breath spa moment'', three times minutes a day.
      You will quickly start to feel better and more alive and your capacity to focus and your mood improves.
      \index{effect!feel!better}\index{effect!feel!alive}\index{symptom!focus capacity}\index{effect!mood improve}
    \end{itemize}
    

\noindent \textbf{Alternating breathing for balance, composure and strong nerves:}
    \index{effect!balance}\index{effect!composure}\index{effect!nerves!strong}

\noindent    Close the right nostril with the  right index finger and inhale on the left side \newline
    Pause your breath for a short while on the full lung \newline
    Close now the left nostril with the left index finger and exhale on the right side \newline
    Pause your breath for a short while on the empty lung \newline    
    Close now the left nostril, inhale on the right side and exhale on the left side, and so on \newline
    Continue changing up the directions, 6 to 12 times \newline
    To finish up make 6 deep breaths, consciously done and perceived, through both nostrils.

    \vspace{2mm}
    \noindent \textbf{Breathing when tired or down:}
\index{symptom!tiredness}\index{symptom!feeling down}

6--12 times inhale right and exhale left \newline
    After each inhale, make the breaks longer

    \noindent \textbf{Breathing when nervous or restless:}
    \index{symptom!nervous}\index{symptom!restlessness}

  6--12 times inhale left and exhale right \newline
  After each exhale, make the breaks longer

      \vspace{2mm}
  This breathing technique acts immediately, but the effects hold for a longer time, if you are prescribing yourself a treatment program.
  That means you practice daily 4 times 5 minutes over a time of about three weeks.
  }

\end{document}