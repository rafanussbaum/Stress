\documentclass[../main.tex]{subfiles}
\graphicspath{{\subfix{../images/}}}
\begin{document}

The laughing muscles are pressing on a nerve and this nerve transmits signals to the brain, which in turn realizes that their human is smiling.\index{laughing}\index{smile}
Caused bu this reaction, the brain excretes ``joy hormones'', the so called \emph{endorphins}.\index{endorphins}

The saying goes, that ``laughing is the best medicine''.
This wisdom got scientifically confirmed by laughter researcher, the so called gelotologists (Greek gelos = laughing).
Gelotology\index{gelotology} is concerned with the physical and psychological effects on humans of laughing and smiling.
A humorous event, which seduces us to laugh or at least smile, is called an exhilaration.
This can be triggered by a plethora of unrelated stimuli.

\epigraph{Humor is the umbrella of the wise ones.}{\textit{Erich K\"astner}}

Already Aristotle\index{Aristotle} (Greek philosopher) and Cicero\index{Cicero}
(Roman politician, writer and speaker and mediator in the sense of Greek education) were researching the sources of humor.
They were attributing laughing to the perception of defects of a human which was perceived as inferior (degradation theory).\index{degradation theory}
Each comedy is constructed by this pattern.
The beginning of clowns and jesters are based on this.

Research of laughter became popular at the end of the 70ies, when the science journalist Norman Cousin\index{Norman Cousin} laid the decisive foundation with the report of his experiences.
The doctors gave him a very unfavorable prognosis as Cousin was suffering from a severe and very painful inflammation of the spine.
He  knew about the negative effects of a pessimistic attitude on the progression of diseases.
He changed the hospital room up into a hotel room.
He was consciously consuming funny movies and humorous books, to systematically get himself laughing.
Thereby he made a fantastic discovery: after ten minutes of intense laughing, his pains subsided and he was able to sleep for two hours without problems.
Cousin completely recovered, even though nobody could really explain why.

\section{Laughing is Balm for the Body and the Soul}

Laughing has a pain--relieving effect and strengthens the immune system.\index{immune system}\index{laughing!effects}
For instance are the values of the natural killer cells and antibodies higher in people, who just have seen a funny movie.
Japanese researcher were able to treat this way successfully allergies, by amusing the patients with movies.

\epigraph{There's a fool--proof way to distinguish between truly great men and the ones which only seem great: All great men have a sense of humor.}{\textit{Ludwig Reiners}}

Laughing is almost like internal jogging: we activate up to 80 muscles, when we're laughing out loud.
This has also a positive effect on the heart--circulation system.
We breathe more intensely and deeply from our belly.
The lung gets well oxygenated, which leads to higher oxygen levels in the blood.
This in turn activates the combustion processes in the body.
First, the heart rate increases, and soon after decreases again, which leads to a decrease of our blood pressure.
The muscles, which were tensed up as we started laughing, relax in a sustainable manner.

Laughing also lowers the production of the stress hormones adrenaline and cortisol.
Instead, the body excretes endorphin.
In hospitals clowns are officially working in the sense of the happiness hormones.

\textbf{A minute of heart--felt laughing is as refreshing as 45 minutes of relaxation training.}

The list of positive effects of laughing is almost endless.
According to gelotologists, laughing also inspires creativity.
The craziest and most entertaining ideas are often born when the meeting degenerated into good-humored silliness.
At least very impressive and extraordinary commercials often got created in this manner.

Truly a means for everything!
Laughing keeps us fit, helps against spring exhaustion and furthermore increases the sexual performance.
Regardless of all these successes, the modern psychotherapy still meets humor with a certain skeptical attitude.

Imagine laughing would be the only thing by which we let ourselves be infected by!

Regardless of the incredible power of laughter, we tend to loose that capacity more and more.
The statistic shows clearly: \emph{While kids laugh up to 400 times during a day, adults only laugh about 15 times a day.}

A few hints, how we can bring it back up to 400 times a day:
\begin{itemize}
\item Even a smile is already enough.
\item Listen to laugh CDs.
\item Look for really funny comic strips in the internet and share them with others.
\end{itemize}

\vspace{1cm}

\mytextbox[0]{Roland Schutzbach and his partner Christine about laughing~\cite{Lachen}
  \vspace{5mm}
   
  \noindent ``We are so happy and cheerful and laugh the whole time''.
  It is called laughing as a habit: laughing became internalized,
  ``and when something goes wrong, we start by laughing about it and then as a second action, we prefer to stop it''.
  
  Many communities still have this old, ``problem oriented'' trait, which signifies:
  ``We have to rescue the world and we have to help the others and we are so important''.
  For the 'foolosophe' it is very clear, that ``they take themselves to be way too important.''
  And exactly that is the main insight from laughing, that ``nothing is really important, the least of all me''.
  The most important things always have been here and always will be be here.
  
  \noindent www.patchadams.org,

  \noindent www.laughteryoga.org,

  \noindent www.wakeuplaughing.com, 
}


\end{document}