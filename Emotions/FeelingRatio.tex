\documentclass[../main.tex]{subfiles}
\graphicspath{{\subfix{../images/}}}
\begin{document}

A person who's letting himself being led by his feelings isn't always behaving in an intelligent way.
A term used in this context is the so called ``bipolar dysfunction``.
Feelings have a big hold over our behaviors and at times coax us into actions, which don't necessarily benefit us.

\epigraph{Many people know, that they are unhappy. But even more people don't know, that they are happy.}{\textit{Albert Schweitzer}}

For instance, when a person feels disadvantaged, the bad feelings gain the upper hand.
In this situation, the influence of rational thinking is strongly inhibited.
On the other hand, if a person suppresses their feelings,\index{feelings!suppress} can count on a whole series of heavy health and mental health problems.

And that's not the only effect.
People who control their feelings, can remember emotional experiences significantly less that people who allow their feelings to express.
Every form of control of the feelings will as a consequence influence our perception.

\section{The Power of the Feelings}


Feelings often also have an influence in economic decisions.
Even wall street broker lets his rational decision, when he gets the slightest feeling that he's about to loose money.
So when for instance a bad feeling takes the upper hand, people tend towards relying more on this bad feeling than on their rational thinking.
Even when their decision doesn't bring them any advantages or even makes them suffer bad consequences.

\epigraph{Illnesses do not come upon us out of the blue. They are developed from small daily sins against Nature.
  When enough sins have accumulated, illnesses will suddenly appear.}{\textit{Hippocrates}}

Suppressed feelings\index{feelings!suppressed} can trigger a whole series of heavy health problems and heavily burden the soul.
If we don't want to understand their language, the only way open to get to us is over our boy, given that they indiscriminately mean it well with us.
We have to get sick --- at least as long as we need to start understanding.

\vspace{1cm}

\mytextbox[0]{David Servant--Schreiber, translated from~\cite{MedEmotionen}
  \vspace{5mm}
   
  \noindent \textbf{``Flow state'' and the smile of the Buddha}\index{flow state}\index{smile}

  \vspace{5mm}

  To harmonically live together with other people, it is important to achieve and maintain an equilibrium between our immediate emotional --- instinctive -- reaction and our rational reaction.
  These rational reactions  maintain social bonds over a longer term.
  The emotional intelligence find their expression in the most appropriate way, when both brain systems, the cortical and the limbic system keep collaborating.

  In this state, the thoughts, the decisions and the gestures shape and realize themselves in a  natural way
  and proceed without us having to give it any kind of special attention.
  We know at any time, which choice we have to make and we pursue our goal without any strain in a state of natural focus, given that we act according to our values.
  We constantly strive towards this state of well--being: the visible and complete harmony between the emotional brain,
  which gives us the energy and the direction and the cognitive brain, which regulates the execution.
  
  The great American psychologist Mihaly Csikszentmihalyi, who grew up in Hungary's post--war turmoil dedicated his life to try to understand the nature of well--being.
  He called this state ``Flow''.

  Oddly enough, there is a very simple physiological indication of this harmony of the brains: the smile.
  Already Darwin was researching it's fundamentals more than a century ago.
  A fake smile -- to one you force yourself to for social reasons -- stimulates only the muscles of the cheekbone.
  Those are the muscles which uncover the teeth when you purse your lips.

  In contrast, a genuine smile mobilizes the muscles around the eyes.
  Those muscles aren't able to contract deliberately, by the means of our cognitive brain.
  The order to do so has to come from the primitive, deeper lying limbic domains of our brain.
  That is why eyes never lie: the ripples around our eyes show if a smile is genuine or fake.

  A heart--felt, genuine smile shows us intuitively if our conversation partner is in this moment in a state of harmony, between what he thinks and feels,
  between cognition and emotion.

  The brain has an innate tendency to go towards the ``flow state''. The most universal example is the smile of the Buddha.
  The aim of the natural methods, which I will present in the following chapters is to allow this to happen.
  In contrast to the IQ, which barely develops to a higher level during our life time, the emotional intelligence can be cultivated and developed at any age.
  It's never too late to learn, how to better deal with your feelings and your relation to other people.

  The first approach described here is without any doubts the most fundamental one.
  It's about optimizing the heart rhythm, in order to withstand the stress, to get the anxiety under control and to maximize our innate vitality.
  This is the first key to the emotional intelligence.
}

\section[Bodily Processes and Emotions]{Bodily Processes and Emotions are Highly Correlated}

The killer cells of the immune system are the first defensive line of the organism.
As most of our bodily functions, the activity of the killer cells is also regulated by our emotional brain.\index{emotions--body correlation}
Positive feelings like calmth and well--being activates them, while fear, stress and depression inhibits their activity.

\epigraph{I think you are unhappy, because you  never were unhappy. You strode through life without ever facing adversity.
  Nobody knows what you are capable of, not even yourself.}{\textit{Lucius Annaeus Seneca}}

It is impossible to make no painful experiences in life.
If a ``bad'' experience was eventually really bad and if we want to learn from it something for our life, that is totally up to ourselves.

It's desirable to recognize the own potential and to unfold it.
The fear which might flame up during the first steps in a new direction are old acquaintances.
Out of that reason, we should include them into our lives as helpers: whenever I feel at unease, I know that something isn't in order.

Fears have this wonderful trait, that they dissolve into thin air, when you \emph{meet them on eye level}.\index{fears}
They actually want that we do that again and again.
Looking at it this way, it is a marvelous thing.
We are never alone and have a marvelous navigation system in our brain, which knows everything that we are supposed to know.
When the feelings gets too often sacrificed to the ratio, it will take revenge on totally different places.
We fall that much easier for emotional bait (commercials, entertainment industry, political activities).

\section{Self--Handicapping: Good or Bad?}

\vspace{1cm}

\mytextbox[0]{from Zimbardo, Psychologie~\cite{ZimbardoPsych}
  \vspace{5mm}
   
  \noindent One of the main assumptions of the research into the concept of self is that people try to determine themselves clearly,
  because they are searching for a coherent\footnote{coherent: connected, interrelated} identity definition of themselves.
  The age of youth the middle age of being an adult are times of intense search, caused by the fact that a lot is changing or gets questioned.
  Big parts of social psychology are based on the idea, that people want to have exact information about their surroundings and their own place in this world.

  Surprisingly, a few of the interesting current research showed that people are actually doing the opposite.
  For instance, people who are suffering from anxious insecurities and doubts about themselves can only keep up their sense of their own competence,
  if they avoid a decisive check of their own competence.
  There is an irony in this that the advantages of avoiding a check of their competence are weighted heavier than the advantages of a clear, meticulous and exact self--assessment.
  The phenomena of self--handicapping gives us a good illustration of this.

  During self--handicapping\index{self--handicapping}, the goal of the person who's doubting himself is to minimize the influence of the own capacities (more exact: the lack of own capacities) as plausible
  explanation for bad performance.
  We are living in a world in which a big part of what we do will get judged by ourselves and others, with the focus on the competences which are at the base of the observable behavior.

  The performance on the tennis court, in the class room, at work and in society is meticulously judged to determine the athletic, intellectual, artistic or social competence.
  But what happens, if that person to be judged has a handicap --- a broken tennis racket, a heavy cold, a sleepless night or a clammed up conversation partner?
  Isn't it in this case unfair to see failure as an expression of the self?
  A person who handicaps themselves follows this logic --- he/she erects a barrier on the path to successful actions.

  Self-handicapping is, expressed briefly, ``a reaction to an anticipated loss of self--respect''.\index{self--respect!protection}
  It is therefore a phenomena which can be classified under the time--honored motive to protect the self--respect or the perceived value of yourself.

  A person who is putting themselves at an disadvantage must have an endangered and fragile, but not a totally negative concept of self.
  An eternal looser who has no or little self--respect after a long and painful history of inadequacies isn't the candidate for self--handicapping.
  A decisive basis for the strategy of self--handicapping is that the person has something to protect.

  If you're assuming, that many different behaviors can be signs of self--handicapping, then you're on the right track.
  An overview over 3 dozen studies which exist to this topic since 1978 reports cases of pusillanimity\footnote{tendency to hesitate, out of a tendency towards anxiety.},
  laziness, abuse of drugs and alcohol\index{abuse!substance}\index{abuse!alcohol}, moodiness, attempting the impossible and many other incarnation of self--handicapping in daily life.
  Even though self--handicapping aren't as widespread as the search of an understanding of the self,
  it nevertheless seems to be clear, that humans often prefer ambiguity over clarity of themselves ---
  especially when that clarity could be not that charming.
}


\end{document}