\documentclass[../Book.Stress_regulation.tex]{subfiles}
\graphicspath{{\subfix{../images/}}}
\begin{document}

\subsubsection{The Importance of Movement}
\index{stress regulation!movement}
Physical movement is paramount to be able to {reduce stress} --- at {every age}. First, it needs physical movement to {remove the stress hormones}\index{stress hormones!burn} from the system.
If not removed form the system, the {stress hormones accumulated} through a sedentary lifestyle correspond to a {over exploitation of the energy reserves}.

The {smallest effort gets the stress hormones active} and they get the energy from the blood and the liver.
The resulting {exhaustion} from this permanent stress has a big {psychological impact}, too.
People try to self--medicate with {excessive eating or alcohol consumption}.

\subsubsection{Importance of Movement --- Fitness Level}
\index{fitness level}
The {lack of flexibility, endurance, strength and the trained respiration}, will bring the person {stress}, because certain everyday activities are hard to manage, like tying your shoes, running if you have to hurry, difficulties lifting objects (impaired mobility and strength) and difficulties going up the hill or stairs.

\subsubsection{Importance of Movement --- Psyche}
By {releasing tension in the body}, {tension in the psyche} gets released too.
 Consciously executed and consciously perceived movements train the {proprioception} and increase the body awareness. This influences the mind in a positive way in a feedback loop.
Increased Body--awareness heightens the sensitivity.
That can help to {perceive stressors} which influence you (too much coffee, acidification of the body, bad  atmosphere in the room,..)
Movement in nature helps with {light exposure}\index{light exposure}, which is very important for our psyche\footnote{In other climates, seasonal effective disorder is a thing.}.  

Movement in {nature} opens the contact to animals and plants. After a while, colors, sounds and smells can be {perceived more intensely} and through that, {new feelings} can be discovered.
That helps to {incorporate feelings} into the daily routine which in turn helps to diminish stress and sickness and to {perceive more joy of life}.
Movement releases {endorphins}\index{endorphins}, which are pain killers and make us happy.

{A good awareness and feeling of our body invariably makes {happy!}}

\epigraph{Just as you ought not to attempt to cure eyes without head or head without body, so you should not treat body without soul.
}{\textit{Socrates}}
\end{document}
