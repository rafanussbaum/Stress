\documentclass[../Book.Stress_regulation.tex]{subfiles}
\graphicspath{{\subfix{../images/}}}
\begin{document}

Movement and the corresponding exercises play an essential role in this book. %course
A few thoughts on the topic.

%\subsubsection{The Importance of Movement}
\index{stress regulation!movement}
\noindent\textbf{Physical movement is paramount to be able to {reduce stress} --- at {every age}.}

\begin{itemize}
\item It needs physical movement to {remove the stress hormones}\index{stress hormones!burn} from the system.
\item A lack of physical flexibility and mobility, endurance, force or trained breathing bring people in stress, because certain everyday activities are hard to manage.
  For instance tying your shoes, running if you have to hurry, difficulties lifting objects (impaired mobility and strength)
  and difficulties going up the hill or stairs are good examples.\index{fitness level}
  \item Stress hormones don't get taken out of the system but add up over time. 
If not removed form the system, the {stress hormones accumulated} through a sedentary lifestyle correspond to a {over exploitation of the energy reserves}.
The {smallest effort gets the stress hormones active} and they get the energy from the blood and the liver.
The resulting {exhaustion} from this permanent stress has a big {psychological impact}, too.
People try to self--medicate with {excessive eating or alcohol consumption}.
\end{itemize}

\newpage
\noindent We can't ignore the mental aspect of movement, given that our psyche is based in our body cells.

\begin {itemize}
  \item By {releasing tension in the body}, {tension in the psyche} gets released too.
  \item Consciously executed and consciously perceived movements train the {proprioception}\index{proprioception!training} and increase the body awareness.
    This influences the mind in a positive way in a feedback loop.
\item Increased body--awareness heightens the sensitivity.
  That can help to {perceive stressors} which influence you (too much coffee, acidification of the body, bad  atmosphere in the room,..)
  \item Movement in nature helps with {light exposure}\index{light exposure}, which is very important for our psyche\footnote{In other climates, seasonal effective disorder is a thing. There's more depression the the cold and dark months of winter than in Summer.}.  
\item Movement in {nature} opens the contact to animals and plants. After a while, colors, sounds and smells can be {perceived more intensely} and through that, {new feelings} can be discovered.
That helps to {incorporate feelings} into the daily routine which in turn helps to diminish stress and sickness and to {perceive more joy of life}.
\item Movement releases {endorphins}\index{endorphins}, which are pain killers and make us happy.
\end{itemize}

Modern science traces neuronal networks (so called connectomes\index{connectome}) through the body.
They are discrete, hardwired neuronal networks which connect the motor areas of the cerebral cortex with the adrenals.
They further connect the cognitive and affective areas of the brain in the cerebral cortex.
This shows that there are indeed dedicated neuronal networks which link these aspects of our lives.
Furthermore, this sheds a new light on psychosomatic diseases, which were so far rather frowned upon and explained away as being ``just in the head''.~\cite{Connectome}


\noindent \textbf{A good awareness and feeling of our body invariably makes {happy!}}

\epigraph{Just as you ought not to attempt to cure eyes without head or head without body, so you should not treat body without soul.
}{\textit{Socrates}}
\end{document}
