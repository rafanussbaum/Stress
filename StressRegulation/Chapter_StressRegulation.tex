\documentclass[../main.tex]{subfiles}
\graphicspath{{\subfix{../images/}}}
\begin{document}


\chapter{Goals}
\subfile{Goals} % Done, 
\chapter{Stress}
\section{What is Stress?}
\subfile{Stress_Theory}% Done,

\chapter{Medical Background}
\section{What Happens in the Body?}
\subfile{Stress_Body.tex} % okay

\section{Anticipatory Coping}
\subfile{Coping.tex}

\section{General Adaptation System GAS}
\subfile{GAS.tex}

\section{Limbic System}\index{system!limbic}
\subfile{Limbic_System.tex}% Done,

\section{Stimuli}
\subfile{Stimuli.tex}% Done,

\section{Sensory Organs}
\subfile{Sensory_Organs.tex}% Done,

\section{Why Makes Stress Thinking Impossible?}
\subfile{Stress_Thinking.tex}

\chapter{Psychological Background}
\section{Psychological Strain and Burden}
\subfile{Strain.tex}

\section{Defensive Reactions}
\subfile{Defensive_Reactions.tex}


\chapter{Stressors}
\section{Stress Scale}
\subfile{Stress_Scales.tex}

\section{Classification of Stressors}
% Here are the missing pages
\subfile{List_Stressors.tex}% 

\chapter{A Wise Contest}
\subfile{Wise_Contest.tex}

\chapter{Stress Regulation}
\section{Solution Oriented Thinking}
\subfile{Salutogenesis.tex}

\chapter{Stress and Movement}
\section{Introductory Thoughts on Movements}
\subfile{Movement.tex}
% Write about neural connection, why movement stress
% https://getpocket.com/explore/item/why-one-neuroscientist-started-blasting-his-core

\newpage
\section{Sun Salutation ---  Surya Namaskar}
\subfile{Sun_Salutation_intro.tex} % Done

\chapter{Mindfulness}
\subfile{Mindfulness.tex} %
\section{The Sitting Meditation} % p. 49-59
\subfile{SittingMeditation.tex}

\section{The Bodyscan}

\subfile{BodyScan.tex}

%\section{Early in the Morning}
% \section{Exercise: Consciously Waking Up}

\newpage
%\subsection{Definition of Stress}

\mytextbox[0]{Here Prof. Dr Sepp Porta\index{Porta, Sepp}~\cite{PortaStress} describes how to get the stress a bit under control.

Simple, but makes a lot of sense:

  \vspace{5mm}

\subfile{Porta_Stress-Hormones_text}}
\epigraph{I prefer meeting a happy person than finding a five pound bill. A cheerful human being spreads good mood around himself and when she/he enters a room, it's as if a light got lit.}{\textit{Robert Louis Stevenson}, translation from German}



\chapter{Summary}
\subfile{Stress_Summary.tex} %

\chapter{Literature Recomendations}
\subfile{Stress_Literature}

\chapter{Practice your knowledge}
\subfile{Stress_Practice.tex}

\end{document}