\documentclass[../Book.Stress_regulation.tex]{subfiles}
\graphicspath{{\subfix{../images/}}}
\begin{document}


Stress is the attempt of the body, to {adapt} to different types of pressure. 
 In science, this process is called the {General Adaptation System} GAS\index{system!general adaptation system GAS}.
 {Long--lasting massive stress} situations induce in the organism three distinct phases.

 \epigraph{It is only with the hearth that one can see right, what is essential is invisible to the eye.}{\textit{Antoine de Saint-Exup\'ery}}
 
 \begin{enumerate}
 \item The \emph{Alarm reaction}\index{reaction!alarm reaction} is the first phase.
   It puts the body in a state of shock\index{shock} directly after the stressor effect registers.
   A lot of {physiological parameters regulate down eventually fail}. These are a lot of small units, which co--create and define the processes of life.
Soon after, the body starts the {counter--regulation}\index{regulation!counter regulation}, in order to be able maintain the body--mind system's integrity. The vegetative nervous system tries to counteract and balance the two effects.
\item The \emph{Resistance Phase} is triggered by {repeated stress} influence or by {ongoing} influence
The body then activates all available means to overcome the stress reaction. This is called the \emph{psycho physiological adaptation}.\index{adaptation!psycho physiological}
\item In the Exhaustion Phase the organism {decompensates}.\index{decompensation}
The {regulation functions} which have been build up by the body during the alarm reaction disappear and the {adaptation} to the stress situation {crashes}.
In the meanwhile the {immune functions} get down regulated and {organic impairments} occur.
 \end{enumerate}
\end{document}
