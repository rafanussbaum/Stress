\documentclass[../main.tex]{subfiles}
\graphicspath{{\subfix{../images/}}}
\begin{document}

The point of departure of Hans Selye\index{Selye, Hans} was an animal in an critical situation of danger. The animal must be in a heightened action potential, the readiness of the muscle system and the circulation but as well the attentiveness and the capacity to make decisions\index{stress!thinking}. Out of this reason the adrenals release adrenaline which causes a vegetative chain of actions, which increases the blood pressure, the sugar content in the blood and the tension in the muscles. For survival it's imperative to provide the maximum amount of energy. It's this inherited reaction which causes modern humans most of the stress problems.

In case of danger, the relatively slow processing in the cerebrum gets inhibited. Priority is given to the brain stem. This is caused by a change in the secretion patterns of the inhibitory serotonin\index{serotonin} and the activating noradrenaline\index{noradrenaline} in the corresponding parts of the brain. This allows us to act faster, but with higher margin of error. The more accurate analysis of the cerebrum would in case of danger often be too slow.

Stress also impacts our ability to think in a negative way, it gets limited. In stressful situations, we should act, not think. Which leads to consequences in the modern lifestyle, as an example a blackout\index{blackout} in a test. Due to stress, the neurotransmitter in the synapses are blocked. Normally they would be responsible for transmitting and processing the information. The more we try to get our capacity to think back under control, the more stress we get. The only way out is to relax. (translated from Interview with Mark Schmid--Neuhaus)

\end{document}