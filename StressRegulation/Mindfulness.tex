\documentclass[../Book.Stress_regulation.tex]{subfiles}
\graphicspath{{\subfix{../images/}}}
\begin{document}

\setlength\epigraphwidth{.55\textwidth}

 \epigraph{I exist as I am, that is enough \\
If no other in the world be aware I sit content,\\
And if each and all be aware I sit content.\\
One world is aware, and by the far the largest to me, and that is myself,\\
And whether I come to my own today or in ten thousand or ten million years,\\
I can cheerfully take it now, or with equal cheerfulness, I can wait.}{\textit{Walt Whitman}: Leaves of Grass}
\setlength\epigraphwidth{.4\textwidth}
%includegraphics drawing myself
\section{What is Mindfulness?}

Most people, when they hear meditation, they think of {Transcendental Meditation\texttrademark} or similar relaxation practices.
In these practices you  {direct the attention to an object}, normally the perception of the inflow and outflow of the {breath or mantra} (a special sound or string of words which is repeated in your thoughts.
All other mental activity is considered {distraction}, which shouldn't be pursued.
Can induce deep states of {calmness} and {develop a steady focus}.
Also known as concentration meditation or {single focus meditation}.   

Mindfulness\index{mindfulness}, also known as {insight meditation} is another major approach to meditation.
In the beginning you use single focus attention to cultivate calmness and steadiness.
After that by {widening your object of observation}, you bring in an element of exploration.
{Thoughts or feelings}, which surface, don't get ignored, suppressed, analyzed or judged by their content.
They get consciously {observed}, as much as possible {without judgment}, how they appear as events in the field of awareness.



Ironically, this all--encompassing perception of the thoughts which appear and disappear in the mind, leads to {less getting tangled up} in them.\index{effect!thoughts, getting less tangled up}
The observer gets a {deeper insight}\index{effect!insight, deepen}\index{insight} into his reaction patterns\index{insight!reaction pattern}
to daily events and difficulties.\index{insight!difficulties}
Through the distancing from the thoughts and observing them, you can get many insights what happens {inside the mind}.\index{insight!mind}
You're able to {name the content} of the thought,\index{effect!thougths, naming} the feelings\index{effect!feeling, link to body} linked to them,
as well as the reaction to those feelings.\index{effect!feeling, reaction to it}
We get aware of our intentions\index{awareness!intentions}, fixations,\index{awareness!fixations} preferences\index{awareness!preferences},
antipathies\index{awareness!antipathies}, incoherences behind your own ideas\index{awareness!incoherence in your ideas}.
We might get insight into what drives us,\index{insight!drives} how we perceive the world\index{insight!perception of the world},
what we think\index{insight!thoughts, nature of} and who we are\index{insight!ourselves} ---
Insight into our fears\index{insight!fears} and wishes\index{insight!wishes}.

\epigraph{Music expresses that which cannot be put in words and that which cannot remain silent.}{\textit{Victor Hugo}}


{The key to practice of mindfulness} isn't so much in the object of our attention but the {quality of attention}, which we give every moment.
It's important that the attention is more like {quietly watching} and observing without taking sides, without passing judgment or instantaneously commenting the inner experiences.
It's a {pure perception}  of the momentary experience, unclouded by judgments helps us to see what is happening inside our mind {without changing}, censoring or intellectualizing it
or getting lost in never--ending thinking process.


This {observing and investigating} approach to everything which gets created in this moment is the characteristic of Mindfulness and it's at the same time the biggest difference to other forms of meditation.
The goal of mindfulness is to {be more aware and connected} with life and what happens in your body and your mind in this very moment.
If a thought or feeling is disturbing or we feel real physical {pain} we try to resist the temptation to withdraw from unpleasant experience.
Instead we try to see and accept it as clearly as possible, just because it's {present in the moment}.

Acceptance\index{acceptance} obviously doesn’t mean being passive or not caring.
On the contrary: By {accepting the moment} totally the way it is we {open ourselves} up to the experiences of life   
and we get more capable to {react appropriately} to every situation that life presents us with.
Acceptance give us a tool to navigate through the highs and lows of life. This demands of us to direct our awareness even to our\index{feelings!negative} tensions, stress, physical pain, and our mental states, like fear,
anger, frustration, insecurities, and feeling of worthlessness.
When these states emerge, {we have to meet them}, accept them and even bid them welcome.
Why? The acceptance of reality, the way it presents itself, if positive or negative, is the the {first step towards changes} of this reality and our relation to it.
 Avoiding meeting things upfront often leads to getting stuck somewhere and having difficulties to change us and to grow.


 {Don't shoot the messenger!}
Pain is only the messenger, telling you very precisely and efficiently {where something is not okay} and {what not to do} because it would make the injury worse.
It's job is to {get your attention} no matter the situation, to prevent further injury to the body. It is seriously good at it. And let's be honest, at times it is really hard to get our attention.
Listen to the message it has to deliver and {acknowledge it}, thank the pain even for the job well done and reassure it that everything will be taken care of as good as possible. 
Often that helps with the {perception of the pain}. It is still there, but it's manageable and not that needle sharp focused thing as before.

The following image might help to clarify mindfulness: {our mind is like the surface of a lake or the ocean}.
There's always {waves on the ocean}, sometimes they are big, sometimes they are small. 

Many people think the goal of meditation is to avoid the waves, so that the surface gets calm and peaceful. That's a misleading assumption.
The spirit of mindfulness is better illustrated by the picture of a 70 years old yogi with big white beard and flowing robes on a surfboard riding the waves in Hawaii. The caption says:
\begin{center}
You can't stop the waves, \\
but you can learn how to ride them.
\end{center}

\epigraph{The stars reflect as well in dirty ponds, puddles and manure run--offs}{\textit{Friedrich Georg J\"unger}, transl. from German}

\subsection{How You Can Practice Mindfulness Yourself}
\index{mindfulness!practice}
Mindfulness gets trained and practiced in two ways, both of which are essential. The first is the \emph{formal} in which we apply specific methods to help to sustain the awake and mindful state longer. The other one is the \emph{informal} practice. There we try to remember to be present in daily life and to once in a while check in if we're still being mindful. Best to imagine mindfulness as a state of being and less as a technique. At the end of the day it's the question, if and how much we're willing to be awake and present at the unfolding of our lives.

\setlength\epigraphwidth{.6\textwidth}
\epigraph{Watch your thoughts, they become words;\\
Watch your words, they become actions;\\
Watch your actions, they become habits;\\
Watch your habits, they become character;\\
Watch your character, for it becomes your destiny.}{Origin disputed.}
\setlength\epigraphwidth{.4\textwidth}

\subsection{The Formal Practice}

\index{meditation!formal} There are different forms of formal meditation, the so called \emph{body scan} (perception of the whole body, lying), the \emph{sitting meditation}, but as well different postures from hatha yoga, which get executed gently, slowly and mindfully (ex: the sun salute).
These approaches offer in multiple ways different doors into the same room.
Mindfulness of the breathing is an integral part of all of them.
First, we're looking at the sitting meditation, given that it can be practiced always and anywhere.

Each of these forms of meditation has a a focus, or a sequence of focus points, on which the the attention is directed.
Whatever focus you choose, it normally doesn't take long for the the {mind to wander off}, no matter how motivated we were to not let it happen.
Every single time this happens, we {take notice} (as nonjudgmental as possible) to where the mind has wandered, before we direct the mind {back to the object of focus}.
We might notice that we were busy with a memory, a future fantasy or a bodily perception or that we were affected of feelings of boredom, impatience or fear. Afterwards, we bring back the attention to the original object of meditation.

The aspect of bringing the attention back to a certain object resembles the one of the concentration meditation, it differs by an important element: you observe where the mind wandered. You become conscious of the changing nature of each experience and that's the core aspect of the mindfulness training.

\subsection{The Informal Practice}

The time and effort dedicated to formal practice helps to {stay present} in very the moment in daily life. 
In theory, it is very easy to be mindful during the day, but it is hard to make it into a {daily routine}. We only have to remember every moment to be present. But even tough it sounds so easy, it's hard to put it in practice.
Instead we have the habit to pass a good amount of our lives mindlessly on {autopilot}, entangled with our own thoughts, and feelings, our moods and reaction to things. Only seldom we're able to see the bigger picture.
 We are dealing with {habits} here. It is hard to break habits. The best way to go about it is to practice and {empower a new, alternate habit}.

 \epigraph{Inside of you! Inside of you is a source of goodness which never stops to bubble, as long and you don't stop digging.}{\textit{Marc Aurel}}


Mindfulness is just a {moment--to--moment awareness}, therefore every ``mundane'' task can be an opportunity to practice: eating, brushing your teeth, cleaning the dishes, walking, driving, declaring your taxes, cleaning up after your dog in the park and many other situations we're facing daily.
The wonderful advantage is that we don't need to put additional time aside.


\textbf{All what's needed is a change in consciousness. A flick of the switch from a habit--driven blind  mode of being to {awake presence}.}

In this mode, daily {life becomes practice, our meditation master and coach}.





% %---------------------------------------------------------------
% \begin{description}
% \item[Mindfulness] \index{mindfulness}
% The {informal practice} is doing this in daily life. 
% {Occurring thoughts} and feelings are noted as an outside observer.  
% The advantages of the informal practice that it needs no extra time and can seamlessly be incorporated into daily life.
% \item[Meditation] is the {formal practice} of Mindfulness, it's method--based. 
%  The {mind's focus} is on an {object}, often the breath. 
%  {Occurring thoughts} and feelings are considered a distraction and are not to be pursued.
%  \end{description}
\end{document}