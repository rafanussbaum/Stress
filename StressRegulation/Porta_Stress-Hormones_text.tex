\documentclass[../main.tex]{subfiles}
\graphicspath{{\subfix{../images/}}}
\begin{document}
Twenty tips, which lead to less stress.
\begin{enumerate}
\item Begin the day well rested. A  lack of sleep\index{sleep} is one of the worst stress accelerators. 

If you didn't get enough sleep during the night, try to lessen that deficit during the day. Already a little nap can drastically reduce stress.
\item Try to act in every situation in a well--planned\index{plan} manner and plan every day up to a certain amount.
``Management by Chaos'' means stress without end, reasonable planning on the other hand is a ``stress killer''.

The one who exaggerates and tries to plan everyone and everything, who leaves his life no room to unfold, will reach the opposite: The stress only gets bigger. You better content yourself with the calming feeling that you have your life more or less under control. Planning to this rough extent is totally enough to diminish stress.
\item Don't take too much on you. Every day, your intend should  be to do a tiny bit less, than you can realistically get done. This small reserve\index{plan!reserves} takes away pressure and therefore stress. Plus, goals\index{plan!goals} which get reached and even surpassed give you nice experiences of success\index{plan!success}, which on their own term reduce stress.
\item Separate your pile of work in easy, medium and hard tasks\index{plan!prioritize}. Then file your work either as urgent, less urgent or ``can wait''. Have the courage to let less urgent work lie for a bit. 

Trust me: The best way to do certain tasks are to let them take care of themselves.
\item Among the urgent tasks, start with the easy ones. After that to the medium ones and then the difficult ones.

Easy tasks can be solved very efficiently and lead to success-- experiences (=Stress--reduction). The hard tasks are then easier to solve.
\item Always do one task at a time\index{plan!step by step}. Don't try to start as much as possible at once, else the `hurry--sickness''\index{hurry} threatens, a disease which means a lot of stress.
\item Try to bring in as much own initiative as possible into your work. That creates experiences of success\index{success} and reduces stress.
\item Politeness\index{politeness} is an ideal stress defense, because it creates respect\index{respect} and harmony. Already the little smile\index{smile} that you got as a thank you for a helping out\index{helping out} somebody acts as a stress brake.
\item With all due politeness, still have a healthy self--esteem\index{self--esteem} and know how to set your boundaries\index{boundaries} --- that strengthens the stress defense.
\item Always try to avoid tardiness\index{tardiness}. Notoriously tardy people put themselves under pressure (bad consciousness) and will be put under pressure by others (upset about tardiness). The result is stress. 

People who try their best at being on time\index{on time}, have more relaxed lives, meaning less stress.
\item Thinking positively\index{thinking positively} inhibits stress. Before a difficult test you best say to yourself: ``Thousands of others passed before me, why then should I not succeed?''

Always keep in mind: if you don't believe in yourself\index{believe in yourself} --- why should others do it?
\item Your body should always be sufficiently stocked with magnesium\index{magnesium}. Magnesium is a very important protection from stress. That's very important, if you nourish yourself mostly with fast food or TV dinners or if you're on a diet to lose weight.
\item When in magnesium depletion, chocolate\index{chocolate} is a well--known and efficient first aid measure against stress.
\item Laugh\index{laugh} and smile\index{smile} a lot. And why not once in a while about yourself? Each smile chases the stress hormones away and shows enough strength to not take yourself seriously\index{sense of humor}. Someone who's able to do that, has sufficient self--esteem (=experience of success!).
\item Undesired loneliness\index{loneliness} is a stress risk. Engaging in social clubs is one of many possibilities to  avoid this risk.
\item Extreme fluctuations in temperature\index{temperature fluctuations} and a strong impairment of your ability to move (example: airplane\index{airplane}) are among the strongest stress triggers. Try to avoid taxing situations like these or  anticipate them early on. Reduce the ``initial stress level'' (meaning other sources of stress) as much as possible. In that case you won't get stressed as quickly.
\item Use every opportunity to keep your body in shape\index{in shape}. It doesn't matter if you ``only'' go for a walk or really work out --- every form of movement\index{movement} regulates your stress levels down. It has to be fun, though. Overtaxing yourself builds up new stress.
\item Profit from the ``Arnold--Schwarzenegger-effect'': the more muscle mass you acquired, the faster you get rid of stress. Important: Even people over 60 years old  can increase their muscle strength\index{muscles!strength} drastically.

Muscles\index{muscles} have the advantage that they use a lot of energy in the form of  fat components. If you train\index{train} your muscles, you can eat more and still stay in shape.
\item Play through different scenarios of difficult situations, which you see coming your way.\index{anticipation} Habituation\index{habituation} helps to reduce stress.
\item A happy love\index{love} is the best mean against unhappy stress.
\end{enumerate}




\end{document}