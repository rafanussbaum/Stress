\documentclass[../main.tex]{subfiles}
\graphicspath{{\subfix{../images/}}}
\begin{document}

\subsubsection{Autonomous Defensive Reactions}
The term autonomous\footnote{separate, independent} defensive reactions means reactions, which get caused by personality structures or by normal physiological reflexes.

Example: When you get stung by an insect you first pull back your hand. Somebody might start swearing, while somebody else might be thinking about what he can do against it.

Reaction:

  \begin{itemize}
  \item Aggression and/or apathy
  \item Flight
    \item Resignation
    \end{itemize}

    \subsubsection{Pragmatic Defensive Reactions}

    Pragmatic\footnote{concerning the application of something, an action or a concrete thing} defense reaction are based on a concept which we trained. Examples are people who have to endure a big amount of stress due to their job, like special units, pilots and emergency medical personnel.

    \epigraph{There's no such thing like bad weather, only inappropriate clothing.}{\textit{Sir Ralph Fiennes}, \textit{Johannes M\"uller} and many others}

    Reactions:
    
    \begin{itemize}
    \item Self consolation
    \item Compensation
     \item Defense
    \end{itemize}

    These examples show well, that it's worth training for stress situations. The trained calmth allows us to do the necessary processing in a more relaxed way.

The deciding factor is the insight, that we can decide ourselves, how we want to live. We can't change the situation itself, but we can always decide if we can or want to adapt (adaptation\index{adaptation}). So the solution to our stress problems lie in our hands.
    
\end{document}