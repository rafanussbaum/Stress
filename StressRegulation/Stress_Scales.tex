\documentclass[../main.tex]{subfiles}
\graphicspath{{\subfix{../images/}}}
\begin{document}

American scientist created a scale which is supposed to show, which are the biggest stress burden for humans.
Such statistical values are very subjective and therefore, the scale is be used with care, if even.

\begin{center}
  \begin{tabular}{r l}
    100 & Death of a loved one \\
    73 & Divorce \\
    63 & Prison \\
    57 & Diagnosis of a severe disease \\
    47 & Loss of job \\
    23 & Quarrels with the boss
  \end{tabular}
\end{center}

This list isn't comprehensive.

Professor Sepp Porta\index{Porta, Sepp} isn't totally agreeing with this scale. He explains it in his literature (translated from \cite{PortaStress}, p. 154): ``No stress at all can be the worst stress.''. Totally sensory deprivation is a terrible type of stress.
He explains in his book that so called ``Couch Potatoes'' often don't realize, that they are double stressed, by living such a sedentary lifestyle in front of the TV. Firstly, they voluntarily restrict their freedom of movement. The second point is that with passive consumption there's a lack of experiences of success leads to big instances of lack of enjoyment.
This often gets compensated by the easiest form of consumption, by eating.

\epigraph{Don't gauge your size by your shadow.}{From Zaire}

\end{document}