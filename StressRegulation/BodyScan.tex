\documentclass[../main.tex]{subfiles}
\graphicspath{{\subfix{../images/}}}
\begin{document}
\label{Ex:BodyScan}\index{exercises!Body Scan}\index{Body Scan}

Your attitude towards health,\index{change in attitude!health}  your behaviors\index{change in attitude!behaviors}
and your vision of yourself\index{change in attitude!vision of self} and the world\index{change in attitude!vision of world} will change.

\begin{itemize}
\item You become more more {steady\index{effect!steady} and confident\index{effect!confident}}.
\item Your motivation to {take care of yourself} will increase\index{effect!motivation to take care of yourself}.
\item Your confidence in your capacity to {act in difficult situations appropriately}\index{effect!confidence, improve} will improve.
\item You will feel a stronger readiness to {see difficult situations as challenge} instead of a threat.\index{effect!difficult situations as challenges}
\item The most important: {life will gain in content and sense}.\index{effect!life, gain in content and sense}
\end{itemize} 

\textbf{Something about the {relationship to pain or negative emotions changes} by approaching them with the intention to {only perceive}, {breathe} with them and let yourself {sink into them};
  by the simple act of {turning towards them}, without evading right away.}

The practice of the body scan allows you to hold your {focus for a prolonged time} and target it precisely.
That improves your {capacity to focus} and to concentrate and helps you to develop mindfulness and a precursor to the {inner calm}.
The body scan becomes most effective by being practiced {daily over several weeks}.

During the body scan, you will {explore the different regions of your body}.
{Lie on your back} and focus your attention through your body: bit by bit, from one part to the next, every moment {conscious of your perceptions}.
{Lying on the back}, can be a wonderful form of meditation, as long as you are able to {stay awake}.
In order to not doze off, it's recommended to {not cross your legs} and to put your {arms next to your body, palms up}.
The guided body scan help help you to scan your body to the smallest detail, but you can do that as well yourself with {your own words and thoughts}. 
{From the toes of your left foot, through the foot and the leg to the hip, from the right toes again to the hip, up through the torso, the loin, the belly, the sacrum, the chest to the shoulders.
  From the fingers, through the arms to the shoulders, to the face, the back of your head and the crown}.

The body scan allows you to {focus your attention} onto a part of your body, to truly feel it and {stay present} in it while {inhaling into and exhaling out of it}. 
It sounds so mundane and yet it may induce {unforeseen outcomes}:
by consciously letting go of the habitual bodily sensations and the associated inner pictures and thoughts
the muscles are able to {relax} and built--up tensions can be released.
During the body scan, your {attention moves in a narrow focussed zone} along your body and {collects tensions and pains} and transports them up the spine to the crown,
where they {get exhaled} to leave a {purified body} behind.
Remember and picture this cleaning and detoxification process while performing the body scan.
That may support your efforts and restore in your body a feeling of oneness.

This {cleaning} is important, but you can't force it.
Let the cleaning process {take care of itself}.
Avoid getting in the situation to anticipate it or to judge it as ``good'' or ``bad'', just continue scanning your body.

People react in very different ways to the body scan.
Some people experience an {inner calm and well--being},\index{effect!calm and well--being}
others might experience {increased tensions, pain, anger, boredom, restlessness} or other irritating feelings. 
It doesn't matter if you have positive or negative feelings.
What counts, is {your readiness to accompany your feelings with your full attention} --- no matter if they are positive, neutral or negative.
Maybe it's hard in the first few hours to {really feel into your toes} or other parts of your body,
perhaps you are troubled by {chronic pain or bad feelings}, which distract you and make it {impossible to feel anything else} then the pain or the bad feeling.
Maybe you get so {relaxed} while doing the body scan and you {fall asleep}.
These and other experiences are {totally normal in the beginning}.
Don't get distracted by them.
Acknowledge these experiences and {continue practising the body scan}.

\epigraph{Allow yourself to experience the intensity of every feeling --- if pleasant or unpleasant ---  exactly at it's turn. }{\textit{Jon Kabat--Zinn}}

\end{document}