\documentclass[../main.tex]{subfiles}
\graphicspath{{\subfix{../images/}}}
\begin{document}

A stimulus\index{stimulus} (plural: stimuli) can be warmth, pressure, pain and so on.
They get registered at sensory cells\index{sensory cells} (receptors), for instance at the skin. A stimulus causes in the coupled nerve cells an electrical impulse, the so called excitation\index{nerve!excitation}.
For the causation of an excitation in the hearth and to transmit the excitation, it's not necessary to have a stimulus to cause it.

An stimulus has to have a certain strength in order to be able to even cause a reaction in the organism. If the stimulus is weaker than the stimulus threshold\index{stimulus!threshold} it doesn't get registered. There are optical, mechanical, thermal, electrical and other type of stimuli. The answer to a stimulus is called a reaction. 
\end{document}