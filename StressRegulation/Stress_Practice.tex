\documentclass[../main.tex]{subfiles}
\graphicspath{{\subfix{../images/}}}
\begin{document}

There's a part with practice materials to each unit of the book. %and doesn't have to be send in

It contains a list of questions. This serves as a help to study. These questions
\begin{itemize}
\item revisit the essential parts of the unit
\item are a starting point to a feasible learning strategy
\item they can easily be looked up in the corresponding chapters, due to the order
  % \item are good examples of questions of the test,
\item contain at the end at least one point which is oriented towards current events or connections
  (and hasn't been treated explicitly in the unit).
\end{itemize}

\noindent There's a list with tasks to be completed, which are meant to be completed before the next unit.
% Bring them along for the next day of seminar.
Those are normally
\begin{enumerate}[label = \Alph*]
\item (physical) exercises
\item preparing a presentation.
  % At the beginning the the seminar days, there will be the presentation of one or a group of participants.
  % It will randomly determined who has to present.
  % Every particiapnt submits an abstract of their presentation one week ahead of the seminar day by e--mail to
  % stress\_regulation@fastmail.com.
  % To have you practice presenting, there will every time be a different method assigned to present the material to the audience.
\item possibly other tasks
  \item possibly instructions for the next unit. %for the next day of seminar
\end{enumerate}

% \noindent The tasks which are to be completed in a written fashion are to be send at the latest one week before seminar day to stress\_regulation@fastmail.com.
% The deadline is always \underline{a week before the day of the seminar}.
% Please indicate \underlin{your full name and the seminar number}.

\noindent The \emph{work up of the current material including the practice section} is an ongoing task for the reader.
Get from the get go in the habit of working up the material and let the questions and tasks guide and help you in this endeavor.

If possible, link up with somebody who studies the same material.
% This is more than just a tip. We want that you have a study partner for this material and we will ask you about it

Make this book your personal \emph{study book}. It's yours and you can annotate it in any way that corresponds to you.
Highlight the essential points in every chapter, maybe with different colors so that you can grasp it in one glimpse and it's easier for your study partner to check your answers.

% Practice transparency towards the organisers of the sminar series. Tell where you're at, how you digest the material and what's expected of you.
% You're facilitating our work tremendiously if you give us this feed--back by e--mail at the above mentionned address.
% Address us depending in length and freqeuncy on your needs, at the latest after the second seminar day.
% We're glad if you can sum up your experience. Typically, there will be no reply to these e--mails excpet if we see it as necessary or you sk about feed back in your e--mail.

We're always interested in feed back, comments and ideas. Please contact us at stress\_regulation@fastmail.com.
% Feel free to incorporate it in your emails or let us know in a different channel.


% So, are you now getting a warranty that you are going to succeed this course and find an satisfying and well paid job in the vast field of stress regulation? ---
We are offering here a unique way, based on newest science and on many experiences and we will guide you to move optimally on this path.
What you do with it and where your path leads you, it's up to you to decide.
It's like with regulating stress: There's only somthing going to happen if I do it myself.

\section[Question]{Questions to Stressors, Sun Salute and Mindfulness}

\begin{enumerate}
\item What are our common goals?
\item How are you improving your authenticity?
\item We expect sufficiency, efficiency and consistency. What is meant by that?
\item What is the most effient method for you to sudy?
\item Describe a trainer who masters his tasks whole heartedly.
  \vspace{0.4cm}
\item What is stress?
\item  Is stress acting only negatively?
\item How does the body react to stress?
\item Name a few examples of coping strategies.
\item Explain the 3 phase model of stress.
\item Which role has the limbic system in the regulation of stress?
  \vspace{0.4cm}
\item Why can't we think anymore under the influence of stress?
\item  Our senses get their signals over certain stimuli. Talk about it.
\item Strain can lead to stress. Discern different cases.
\item Upon stress, we react with defense reactions. Which ones?
  \vspace{0.4cm}
\item Classify the stressors and name a few examples of each category.
\item Ongoing Strain can lead to physcal affects. Whcih ones?
  \vspace{0.4cm}
\item What does the term salutogenesis mean? Which signification does it hold for you?
  \vspace{0.4cm}
\item Encounter stress with movement. Justify!
  \vspace{0.4cm}
\item What does is mean to be mindful/attentive?
\item In which points does mindfulness differ from other forms of meditation?
\item Distinguish the two main forms of meditation.
\item Instruct a sitting meditation. Just focus on the start.
\item Whcih conditions do we need to look at in a sitting meditation?
\item Your thougths are wandering during a meditation. Do you have a recipee to stay present?
\item There's many possibilities in the body--scan. How can you approach these possibilities?
  \item What are those possibilities?
  \item What is a body--scan?
  \item Whcih obvious advantages do you gain by getting up early in the morning?
    What are the disadvatages and eventual conditions? How do you proceed?
  \item Are you able to conscously wake up in the morning? What does ``consciously wake up in the morning'' mean for you?
    \vspace{4mm}
  \item Which are your five most important levels of stress regulation? Order them in a way that makes sense for you.
  \item Can you name ten of the twenty tips from Porta's text?
    \vspace{4mm}
  \item Name the positions of the sun salute.
  \item Whcih kind of people would you recommend practising the sun salute?
    \vspace{4mm}
    \item Stress is also an economical factor and consideration. Explain!
  \end{enumerate}

\end{document}