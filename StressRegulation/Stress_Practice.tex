\documentclass[../main.tex]{subfiles}
\graphicspath{{\subfix{../images/}}}
\begin{document}

There's a part with practice materials to each unit of the book. %and doesn't have to be send in

It contains a list of questions. This serves as a help to study. These questions
\begin{itemize}
\item revisit the essential parts of the unit
\item are a starting point to a feasible learning strategy
\item they can easily be looked up in the corresponding chapters, due to the order
  % \item are good examples of questions of the test,
\item contain at the end at least one point which is oriented towards acurrent events or connections
  (and hasn't been treated explicitely in the unit).
\end{itemize}

\noindent There's a list with tasks to be completed, which are meant to be completed before the next unit.
% Bring them along for the next day of seminar.
Those are normally
\begin{enumerate}[label = \Alph*]
\item (physical) exercises
\item preparing a presentation.
  % At the beginning the the seminar days, there will be the presentation of one or a group of participants.
  % It will randomly determined who has to present.
  % Every particiapnt submits an abstract of their presentation one week ahead of the seminar day by e--mail to
  % stress_regulation@fastmail.com.
  % To have you practice presenting, there will every time be a different method assigned to present the material to the audience.
\item possibly other tasks
  \item possibly instructions for the next unit. %for the next day of seminar
\end{enumerate}

% \noindent The tasks which are to be completed in a written fashion are to be send at the latest one week before seminar day to stress_regulation@fastmail.com.
% The deadline is always \underline{a week before the day of the seminar}.
% Please indicate \underlin{your full name and the seminar number}.

\noindent The \emph{work up of the current material including the practice section} is an ongoing task for the reader.
Get from the get go in the habit of working up the material and let the questions and tasks guide and help you in this endevour.

If possible, link up with somebody who studies the same material.
%This is more than just a tip. We want that you have a study partner for this material and we will ask you about it

Make this book your personal \emph{study book}. It's yours and you 

\end{document}