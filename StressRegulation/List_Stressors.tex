\documentclass[../Book.Stress_regulation.tex]{subfiles}
\graphicspath{{\subfix{../images/}}}
\begin{document}

To counteract your own stress, it's important to know those overburdened stimuli, the so called stressors who impair our well-being.

\subsubsection{Physical Stressors}
\begin{itemize}
\item Heat
\item Cold
\item Loud noise
\item Changes in air pressure
\item Hunger
\item Thirst
\item Infection
\item Wounds
\item Illness
\item Surgical treatment
\item Sleep deprivation
  \item Too intense light
  \end{itemize}

  \subsubsection{Personal (Mental) Stressors}

  \begin{itemize}
  \item Insecurities
  \item Too much or not enough motivation
  \item Family conflicts
  \item Work
  \item Career
  \item Not having the permission for essential needs, i.e.e no permission to use the bathroom.
    \item Negative Thinking 
  \end{itemize}
\subsubsection{Psychological Stressors}


\begin{itemize}
\item Tendency to impatience
\item Anger
\item Engagement
\item Fear
\item Hostility
\item Dominant behavior
\item Competitiveness
\item Wrong assessment of the situation
\item To get yourself worked up in something
\item Your own time and performance pressure
\item Alarmist tendencies
\item Too high expectations for yourself
\item Disappointment
\item Imagined threats or helplessness
\item Loss of a loved one or animal 
\item Time pressure
\item Over challenged
\item Under challenged
\item Testing situation
\item Density (crowds)
\end{itemize}



\subsubsection{Social Stressors}

\begin{itemize}
\item Conflicts
\item Differences of opinions
\item Loss of relatives and close beings
\item Isolation
\item Group pressure (Mobbing)
\item Rejection by other people
\end{itemize}

\subsubsection{Chemical Stressors, Foreign Substances}
\begin{itemize}
\item Nicotine
\item Alcohol
\item Drugs
\item Chemicals, ex: Deodorants, tooth paste, body lotions
\end{itemize}

Ongoing excessive burden will cause states of exhaustion, sensitivity to be nervous, but as well in back pain, ulsters, high blood pressure and hearth attack. Humans can adapt to the straining situation up to a certain point. In such stress situations physical and mental reserves get mobilized to uphold the equilibrium. Permanent mismanagement of your own resources invariably lead into a dead end. When these reserves are depleted often an instantaneous collapse ensues.

\epigraph{A pessimist sees the difficulty in every opportunity; an optimist sees the opportunity in every difficulty}{\textit{G\"unther F. Gross}}
\end{document}