\documentclass[../main.tex]{subfiles}
\graphicspath{{\subfix{../images/}}}
\begin{document}

The advantages of getting up early\index{exercise!early morning}\label{Ex:EarlyMorning} in the morning have nothing to do with the fact that you have even more hours to be busy and work.
Pretty much the opposite, the advantages lie in the silence and loneliness of the early hours of the day
and the possibility to use those hours to expand the consciousness, to contemplate, to have time to just be, time for the conscious effort to do nothing.
The peaceful mood, the darkness, dawn, the silence -- all this makes the early morning into an especially good time for mindfulness exercises.

To wake up early has another advantage, that you encounter the day way more peacefully, due to your ``head start''.
When you start the day after rooted yourself into mindfulness and inner peace, you will be significantly better to let your actions come out of the current moment and your heart while working on your daily chores.
You will likely approach things with a solid foundation of mindfulness.
Inner peace and balance will accompany you thorough your day, no matter how urgent and important your activities might be.
It is a totally different way to start your day and routine than jumping out  of bed and finish your chores in haste.

There's such a strength in waking up daily, that it can profoundly change the life of a human,
if one makes it a habit, even if you don't formally spend your time in the early morning with mindfulness.
Just to experience how the day starts is on it's own a waking up call to the spirit.

When you root yourself in mindfulness early in the morning, you are becoming aware of the fact that things constantly change, that good and bad things come and go
and that it's possible to maintain in every situation an attitude of stability, wisdom and inner peace.
The daily decision to rise early and to meditate is an expression of this view.

When you are hesitant to get up an hour earlier than usual, you can try it with half an hour, a quarter o an hour or even just five minutes.
What's decisive is your intention and attitude.
Even five minutes which you dedicate to mindfulness first thing in the morning can be very valuable.
By sacrificing five minutes of your sleep, you will get aware of how much you value your sleep and how much decisiveness and discipline you have to muster to take yourself this little bit of time to be awake, without doing anything.

Our mind always has a very credible excuse ready. Given that there is no urgent necessity, why I should start his morning with these exercises?
Especially given that there's no tangible outcome of this exercise, why shouldn't I allow myself the additional sleep, which I can use so well.
In order to overcome such obstacles, you will have determine in the evening, at what time you are going to get up, no matter what your brain is going to tell you in the morning.
That requires true zest, eagerness, force and inner discipline. You're doing something out of the reason that you made a commitment to yourself to so it and at a given time, no matter how parts of your mind feel like doing it or not.

After a while, this discipline will be part or yourself.
It is just a different way of living which you acquired by training it.
There's no more forcing yourself, and it won't feel as if you're forcing yourself to do it.
Your values and your actions are changed.

If you're not ready yet to commit to get up early on a regular base (and even if you are) you can use the moment of waking up as a moment of mindfulness, no matter at which time.
Try to visualize how your breath is moving even before you move your body.
Feel how your body lies in bed.
Stretch your body.
Ask yourself: ``Am I now truly awake? Am I really aware that I just got a gift of a new day?
Will I face this new day all awake? What will happen today?
I don't know yet in this very instant.
Am I able to stay open towards this not--knowing, even while I ask myself what's coming up?
Am I able to see the here and now as a plenitude of new possibilities?''

\setlength\epigraphwidth{0.9\textwidth}

\epigraph{Morning is the time when I awake and the day rises inside of me.
  We humans have to relearn to get awake again and keep ourselves awake, not by mechanical tools and means,
  but by relentlessly expecting the rise of the sun. An expectation which mustn't even quit us entirely in our deepest sleep.
  I don't know any more noble and more uplifting capacity of humans, then the undoubtedly capacity to uplift your own life to a higher ground through conscious effort.
  There is no doubt a lot of value in painting an especially beautiful painting, to chisel a statue, to create something beautiful.
  But it is by far more honorable to paint or to chisel the atmosphere, the medium itself through which we look through\ldots
  To act on the very fabric of the day itself, that is the highest of all forms of art.
}{\textit{Henry D. Thoreau, Walden}}
\setlength\epigraphwidth{.4\textwidth}

\end{document}