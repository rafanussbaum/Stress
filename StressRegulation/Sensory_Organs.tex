\documentclass[../main.tex]{subfiles}
\graphicspath{{\subfix{../images/}}}
\begin{document}
 
A stimulus triggers the receptors of the sensory cells to give off a signal, which get transported as action potentials.
If these signals cause a sensation by becoming conscious as part of our senses they get called perceptions\index{perception}.
Humans know five senses and their corresponding stimuli:
\begin{description}
\item[Touch] skin; pressure/touch and temperature.
\item[Taste] tongue; salty, sour, sweet and bitter.
\item[Smell] nose; odorous molecules, participates on taste, too.
\item[Seeing] eyes; brightness and color.
  \item[Hearing] ears; sound waves = pressure waves.
  \end{description}

  There's a plethora of other types of stimuli, which a human can't perceive directly, like magnetism and ultra sonic sound.
\end{document}