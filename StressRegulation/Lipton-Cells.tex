\documentclass[../main.tex]{subfiles}
\graphicspath{{\subfix{../images/}}}
\begin{document}
Just as the cells, humans are also not able to maintain completely their growth\index{growth} impulse if they go over in a protective behavior\index{protective behavior}. 
For the survival\index{survival mode} they need at that moment all their energy for escape and fight\index{fight or flight}. 
The diversion of energy in favor of the protective reaction always goes at the expenses of the growth. 
Besides, necessary energy is drawn off not only to the preservation of the organs and tissues. 
Growth processes also require an open exchange\index{exchange with surroundings} between the organism and the surroundings, for example, food is taken up and waste-products are eliminated. 

 

Nevertheless, a protective reaction requires to close the system to wall off the organism against the expected danger. 
Nevertheless, the cut off of growth is also weakening because the growth process uses not only energy, but also it produces. 
If the protective position maintains for longer time, it restrains the production of the life-receiving energy. 
The longer you remain in the protective position, the stronger the growth suffers from it. 
They can prevent your growth processes even so far that you are able really to ``fear to death''\index{fear!to death}.

 

Differently than with protozoans the growth reaction or protective reaction with multi cellular organisms is not an "Either and Or" decision and not all our 50 billion cells must go at the same time to a growth reaction or protective reaction. 
The part of the cells which are involved in a protective reaction depend on the gravity of the perceived danger\index{perception!danger}.
Thus you can also survive under stress, but the chronic restriction of the growth mechanisms goes at the expenses of the vitality\index{vitality}. 
It is also important to note that it is more necessary for the full development of your vitality, than to reduce the stress factors in your life. 
In a growth and protective continuum the removal of the stress factors moves you only into a neutral position. 
To blossom and to prosper we must not only get rid of the stress factors, we also have to strive actively for a joy-filled\index{joy}, affectionate\index{affection}, fulfilling\index{fulfillment} life which provides growth stimuli\index{growth!stimuli} for us.
\end{document}