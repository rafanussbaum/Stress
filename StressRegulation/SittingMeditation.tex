\documentclass[../main.tex]{subfiles}
\graphicspath{{\subfix{../images/}}}
\begin{document}
\index{exercises!sitting meditation}
\label{Ex:SittingMed}
In the center of the formal practice is the sitting meditation\index{sitting meditation}.
In the {sitting meditation}, we {sit and breathe}.
Two activities we encounter everywhere in daily life. 
The sitting of the sitting meditation differs from the usual, daily sitting by the presence of our {inner mindful attitude}. 
In the sitting meditation it is important to sit in a {dignified posture},
with erect head, neck and back, without tensing up. 
You can use a chair with {straight back rest} or on a pillow on the {floor with your legs crossed}.
The position should radiate and {awake and dignified posture}.

Normally you start by choosing an {object to focus the attention} - for instance your {breath}. 
To be more precise, you focus on a single aspect of your breath,
like the sensation of the {air streaming in and out} at the back end of your nostrils
or the gentle {stretching and lowering of your abdominal wall} with each inhale and exhale.
Once a certain degree of focus developed, the {attention can get widened} beyond the changing sensations of your breath. 
Sounds, perceptions, thoughts or other objects will be perceived in a heedful manner as soon as they enter your awareness. We try as good as possible to maintain a {calm, non--reactive and steady awareness}, anchored by the breath.

\subsubsection{What do you need for a sitting meditation?}

 A {fixed point of time}. 
 A place where you can relatively {undisturbed}. 
 An {upright posture} which allows you to sit for a prolonged period of time. 
 Find a seat on a chair with straight back rest or on the floor.
 If you sit on a chair, both {feet} should rest with the whole {soles on the floor}.
 Most of the time it’s indicated to not use the back rest, but to keep your {back free, straight and upright}.
 If this is too exhausting, then it’s preferable to lean against the backrest then to be constantly distracted by the unfamiliar strain.


\end{document}