\documentclass[../main.tex]{subfiles}
\graphicspath{{\subfix{../images/}}}
\begin{document}


The human cerebrum is split in half, the \emph{hemispheres}\index{brain!hemisphere}.
Each half has four lobes: The \emph{frontal, parietal, temporal and the occipital lobe}.
They are separated by the longitudinal fissure.
A white substance, the {corpus callosum}, connects the two.
The corpus callosum consists of motor and sensory axons which serve as connectors between the two hemispheres.
The cerebrum works in a \emph{cross--over pattern}.
Half of the body is controlled by the opposite hemisphere.
Research showed, that each hemispheres process information in a unique way.

\begin{figure}[htb]
  \centering
  \includegraphics[width=5cm]{brain_hemispheres.eps}
  \caption{The hemispheres of the brain\cite{needpix}.}
\end{figure}


The ``logical hemisphere''\index{brain!hemisphere!logical} (typically left) organizes language, reading, writing, calculations, musical capacities and analytical thinking.

The ``Gestalt--hemisphere''\footnote{German ``Gestalt'': a figure. In this sense here: shape, form.}\index{brain!hemisphere!Gestalt} (typically the right one) is associated with feelings, connections, language, spatial perception, intuition, thinking in pictures and synthesis.

The corpus callosum\index{corpus callosum} serves as data freeway, which allows fast access to the linear details of the ``logical'' hemisphere and the ``Gestalt'' hemisphere and therefore make integrated thinking processes possible.

\begin{table}[htb]
  \centering
  \begin{tabular}{p{5.5cm}p{5.5cm}}
    \textbf{Logical} & \textbf{Gestalt} \\
    Perception of the components of the language & perception of the holistic view \\
    sense of and signification of language (semantics) & comprehension of language \\
    structure of sentences (syntax) & rhythm, flow of the language, dialects \\
    letters, sentences, numbers & picture, signification \\
    analysis & synthesis, immediate perception \\
    linear & simultaneously multiple thought processes \\
    noticing differences & spontaneous, free flowing \\
    controlling feelings & allowing feelings to flow \\
    structuring, planning & oriented towards feelings and experiences \\
    logical, structured thinking & intuitive, emotionally present \\
    oriented towards the future & oriented at the present moment \\
    sports: usage of hand, eye and feet & sports: flowing and rhythm \\
    arts: media, usage of tools & arts: pictures, emotions, flow \\
  \end{tabular}
  \caption{Processes of the two hemispheres.}
\end{table}

The better the access to the two hemispheres succeeds, the more intelligently we can act.
Therefore it's our goal, to \emph{engage both hemispheres of the brain} in as many activities as possible, to be able to perform well.

\vspace{1cm}

\mytextbox[0]{Translated from~\cite{SpitzerLernen}
  \vspace{5mm}
   
  \noindent In February 2002 the case of a 7 year old girl got published in the international medical journal Lancet.
  She had the left hemisphere of the brain surgically removed at the age of three to treat an otherwise fatal chronic encephalitis with uncontrollable epileptic seizures.

  The girl was missing a hemisphere of her brain. On top of that, the left, language dominating hemisphere.
  One could reasonably expect heavy impairment of one half of the body and the lack of verbal communication.

  This case was special: the child was at the age of seven normally developed and didn't only fluently master one, but two languages.
}

For instance creativity\index{creativity} is not a function of exclusively the Gestalt hemisphere.
Creativity is a holistic process which needs the techniques and the details from the logical hemisphere and the flow and the emotions from the Gestalt hemisphere.
In order to be able to fluently speak we also need the words and sentence structures from the left and the pictures, emotions and dialects from the right hemisphere.

\emph{Cross--over movement} are suited to support balance\index{balance} and to activate both hemispheres, similar to crawling of a baby or the Brain--Gym movements\index{exercises!Brain--Gym}.
These exercises activate both sides of the body evenly involve the coordinated movement of both eyes, ears, hands, feet and also the balance of the core muscles.
When both eyes, ears, hands and feet are equally getting used, the corpus callosum\index{corpus callosum!training} who supports and allows the coordination between both hemispheres develops optimally.
As an effect, the cognitive functions improve and learning becomes easier.


\vspace{1cm}

\mytextbox[0]{Translated from~\cite{LiptonCell}
  \vspace{5mm}
   
  \noindent British neurologist Dr. John Lorberr highlighted in a 1980 article in \emph{Science} called into questions the notion that the size of the brain is the most important consideration for human intelligence (Lewin 1980~\cite{LewinWater}).  Lorber studied many cases of hydrocephalus (“water on the brain”) and concluded that even when most of the brain’s cerebral cortex (the brain’s outer layer) is missing, patients can live normal lives. Science writer Roger Lewin quotes Lorber in his article:

“There’s a young student at this university [Sheffield University] who has an IQ of 126, has gained a first-class honors degree in mathematics, and is socially completely normal. And yet the boy has virtually no brain…When we did a brain scan on him, we saw that instead of the normal 4.5 centimeter thickness of brain tissue between the ventricles and the cortical surface, there was just a thin layer of mantle measuring a millimeters or so. His cranium is filled mainly with cerebrospinal fluid.”
}


\end{document}