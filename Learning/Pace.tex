\documentclass[../main.tex]{subfiles}
\graphicspath{{\subfix{../images/}}}
\begin{document}
\label{Ex:PACE}
\index{exercises!PACE}

(Translated parts from the book of Carla Hannaford ``Bewegung, das Tor zum Lernen''~\cite{BraingymD}
(English Version ``Smart Moves: Why Learning Is Not All In Your Head''~\cite{BraingymE}))\index{brain!whole}

\vspace{1cm}
PACE is a series of exercises from Brain-Gym  to increase the learning capacity and
readiness\index{increase!learning capacity and readiness}\index{symptom!learning, capacity}\index{effect!learning, readiness}.
Normally, done in the morning, after breaks and meals to get the students effectively ready for learning.\index{symptom!learning, ready for it}\index{effect!learning, ready for it}
It's also very useful to practice ahead of activities when you need to be completely integrated.
It consists in \emph{drinking water} to provide sufficient energy for the learning process, then \emph{Brain Buttons}, the \emph{Cross Crawl} and \emph{Hook Ups}.


\vspace{1cm}
\begin{tabular}{ll}
\textbf{P}ositive & Hook Ups\\
\textbf{A}ctive & Cross Crawl\\
\textbf{C}lear & Brain Buttons\\
\textbf{E}nergy & Drink Water\\
\end{tabular}

\subsection{Brain Buttons}

In  the exercise brain buttons, hold one {hand over the belly button}, the other hand stimulates two points between the rips.
The hand on the {belly button} brings the attention to the {center of gravity}.\index{effect!center of gravity, focus}
  Here are the {core muscles}\index{muscle!core}, which are very important for the {physical equilibrium and balance}.
  The vestibular system gets activated\index{effect!vestibular system, activated}, which {readies the brain for incoming sensory input}.
  If somebody is staring in the space in front of them
  ({ocular blockage})\index{ocular blockage},\index{symptom!ocular blockage}
  the {vestibular activation} helps to get the {eyes back to moving} to allow the brain to start processing external visual information again.



  The other hand rubs the dents immediately left and right of the sternum between the first and second rib directly underneath the collarbone.
  It is assumed that this {increases the blood} flow through the main arteries into the brain.\index{effect!brain, blood flow}
  The main arteries branch out from the hearth and bring {fresh oxygen to the brain}.
  The buttons lie close to point where the main arteries fork.
  It is probably the {baroreceptors} (pressure receptors)\index{receptors!baroreceptors} which are activated while rubbing these {points}.\index{effect!baroreceptors, activate}
  Baroreceptors react to a change in blood pressure and regulate a normal flow of blood to the brain.  
%These are the end points of the kidney meridian (energy pathway in our body).


  \subsection{Cross Crawl}
  
  The Cross Crawl is a simple crossover movement in place. Having the right elbow touch the left knee and the other way around,
  big sections of our brain get simultaneously activated.\index{effect!brain!activate}
  This crossover movement is a conscious walking motion,
  which promotes balanced neuronal activity across the corpus callosum\index{training!corpus callosum}\index{effect!corpus callosum, train}.
  If this happens on a regular base, more neuronal network gets build in the corpus callosum with thicker myelin layers,
  which allows a faster and more integrated communication between the two hemispheres\index{effect!hemispheres, integrate}
  and therefore thinking on a higher level.\index{hemispheres!communication training}\index{effect!thinking, higher level}
  
  {The cross over movement should be should be {executed very slowly}.
    A slow execution uses more {fine motor skills} and {balance}  and this leads to a conscious activation of the {vestibular system} in the {frontal lobe}.
    The more fine motor skills are involved, the more the frontal lobes connects with the basal ganglion of the limbic system and the cerebellum in the brain stem.}

  This simple movement elegantly activates the functions of the whole brain and radiates  into the frontal lobes.
  Whenever I'm blocked in my thinking, I either do the cross crawl or a walk, and the thoughts are able to flow again.\index{symptom!thinking, blocked}

  Robert Dustman is leader of the neuro-psychological research at the Veterans Affairs Medical Center in Salt Lake City, Utah.
  His research showed that walking increases the mental productivity of men and women between fifty and sixty years of age.
  First he had relatively healthy people between fifty and sixty years a series of mental and physical tests.
  Then the participants went through a four month program, where they had to walk briskly on a regular basis.
  After the four months, the participants scored ten percent higher at  the same tests.

  This movement is ideal to {activate the body--mind system}\index{body--mind!activate}\index{effect!body--mind, activate} before physical activities like sport or dancing.

  \subsection{Hook Ups}
  
  Stand with the {legs crossed}, left over right.
  {Cross the arms} over in front of your body, interdigitate (interlock) your hands and rotate them {to the front of your sternum}.
  To do so, reach out with your hands, put the {back of the hands together}, thumbs are pointing down.
  Keeping the orientation of the hands, put one over the other, palm to palm and {interlock the hands} (interdigitate).
  Then {turn} the interlocked {hands down, towards the body}. They will eventually be {in front of the sternum}, with the elbows facing down. 

  Similar to the cross crawl movement, this complex cross over movement {integrates the hemispheres} and the whole brain.
  On top of that, it activates the {sensor and motor areas}
  of the cortex in both hemispheres.\index{effect!brain!sensor area, activate}\index{effect!brain!motor area, activate}
  
  Additionally, a possibility to expand the exercise is to push the tongue flat against the roof of your mouth.
  It focuses the attention to the {mid brain}, which lies above the roof of your mouth.
  It lowers an {increased tongue pressure}\index{symptom!tongue pressure, increased}, which has been   caused by an unbalanced posture.
  In this position,  the emotions in the limbic system connect with the reason in the neocortex.\index{effect!emotion!connect to reason}
  This gives a more balanced perspective\index{effect!perspective, balance} which allows to learn and act
  more \index{effect!learning, more efficient}\index{effect!action, more efficient}
  efficiently.\footnote{According
    to traditional Chinese medicine and for instance the Taoist tradition, this position connects the {two main meridians}.
  The conception vessel ends in the tip of the tongue and connects with the end of the governor vessel and that allows a flow of chi through these meridians.~\cite{ChiaMCO}}
 
My trainer established the two minute rule as consultant in school classes.
When kids between five and fifteen years old got send to her because of a fight or disturbing the
class,\index{symptom!children!unruly}\index{symptom!children!class, disturbing}\index{symptom!children!fights}
they had stand in the hook up position for two minutes, before they would talk.
Their conscious attention goes to the motor cortex of the frontal hemispheres, away from the survival center of the reptile brain.\index{symptom!survival mode}
This significantly reduced adrenaline production.
After two minutes they were could see their and the other view point more clearly.
None of these students wanted to get in trouble. They were grateful to have a tool to control and steer their own behavior.\index{effect!behavior, control}

This is the BrainGym Exercise, which I apply most.
Teacher often use it for themselves, when the stress level rises, or then to provide some peace and focus to the students after recess.\index{symptom!stress level rise}
I want to invite you to an experiment.
Focus on a very stressful or taxing situation in your life.
Pay attention to where you start to tense up, where your muscles tense and what your breath is doing\ldots -- to all your reactions.
Then do the hook up exercise while sitting, standing or lying down, two to five minutes.
Pay attention to the differences in muscle tension, breath and in your attitudes.
It's the same situation, but the whole body--mind system gets used to better deal with the situation at hand.

On the following page are the practical instructions for the PACE exercise. These exercises increase the readiness to learn.
It's an ideal exercise for students and people who want to be intensely integrated in their activities.\index{effect!integrated in activity}
\newpage

\end{document}