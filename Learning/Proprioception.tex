\documentclass[../main.tex]{subfiles}
\graphicspath{{\subfix{../images/}}}
\begin{document}

The word  proprioception\index{proprioception} (proprius = ``own'', ``itself'') means as much as ``{perception} or reception of the {body's own stimuli}''.
To perceive these stimuli, we need corresponding {peripheral receptors}.
Those are present in the posture and movement system of the body in joint capsules, muscles, tendons, fascia and in the skin.
They are called {mechano--receptors}\index{mechano--receptors} and exist in different forms. 
As the receptors of the equilibrium system, they transmit information about the {position and movement of the body} to the central nervous system, over sensitive afferent (incoming) nerves\index{nerves!afferent}.
The central nervous system {reacts} to these afferent stimuli with efferent (outgoing to the periphery) impulses\index{nerves!efferent}.

Let's have a look at an example: \newline
If we're walking through a dark room, we are still informed about the position of our body, without the visual control, like a blind person.
This is due to sensitive receptors at the bottom of our feet and in the joints of our legs, which maintain constant ``radio contact'' with the central nervous system over the afferent nerves.
If we would suddenly step on an inclined plane, these receptors sends the news of this event immediately to the central nervous system, which in turn initiated to counter movements, so that a fall can be avoided.
In this case, the proprioception works out.

A person suffering from diabetes with severe peripheral neuropathy\footnote{Damage to the afferent nerves which transmit signals to the brain or spinal cord},
the risk of falling is considerably higher.
The overall picture of an affected person walking is almost like on ``walking on cotton'', they are dependent on visual control by the eyes.
In this case, the proprioception can't work.

\begin{description}
\item[What happens now under stress influence? ---]
  When the feedback system between the proprio--receptors and the muscles is well developed through constant use, the balance system is able to cope the physical balance is maintained.
  When stress disturbs the even activation of this system, we're ``not centered anymore'', and the sense of balance and the conscious sense of the room around the person gets lost.
  Accidents are prone to happen, leading to scratches, bruises or even broken bones.
  Our sense of proprioception isn't focused on the room around the person anymore, but on escaping a danger.
\item[Proprioception and learning:] The more we look at the intricate collaboration of brain and body, the clearer and more obvious it becomes that
  \emph{movement is absolutely necessary for learning}\index{learning!movement}.
  Modern neurological research\cite{Connectome} shows that there exists dedicated discrete nerve networks (connectomes)\index{connectome} which indeed link the motor cortex of the brain to the adrenal glands, but also to the affective ares, which deals with feelings and emotions.

  Movement awakes and activates many of our mental capacities. Movement integrates and anchors new information and experiences into our neuronal network.
  And on top of that, movement is the prerequisite to express what we learned, our understanding of it and ourselves through actions.
  And we can research our environment with our senses over our muscles.
  
\end{description}

Proprioception also means the perception of the body of itself. Even with closed eyes, we still have a pretty good impression about our position in the room, of the position of our arms and hands.
This is possible due to sensory cells in the joints and muscles which keep the brain (especially the cerebellum) constantly updated about the placement in space.

Proprioceptive perception is also decisive in order to be able to take and maintain a stable posture. Only with the help of ``well trained'' sensors are we even able to stand upright and to maintain that upright stance over linger periods of time.
That signifies, that training of the muscle force and stretching the stabilizing core muscles alone is not enough, to automatically take a good posture.
The interplay of the muscles and the awareness of the own body  also has to be trained for a good posture.

Good proprioceptive training\index{proprioception!training}\label{Ex:Proprioceptive} consists for instance in executing the ``active stance'' (always without shoes), at best in front of a mirror.
After slowly step for step get in the correct position, close the eyes, and try to ``feel''  the position of your body and the degree of tension in the different musles.

A possible progression is the same exercise standing on one leg only, and eventually while closing your eyes.

Balance boards (stable wooden board, mounted on a cylinder or blockso that it can tilt) are good tools for proprioceptive training.
They require the user to constantly stay in slight movement while standing to keep the balance.
To increase the difficulty level of standing on one leg, stand on a soft mat   or folded up woll cover.


\end{document}