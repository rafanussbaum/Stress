\documentclass[../main.tex]{subfiles}
\graphicspath{{\subfix{../images/}}}
\begin{document}

The word  proprioception\index{proprioception} (proprius = ``own'', ``itself'') means as much as ``{perception} or reception of the {body's own stimuli}''.
To perceive these stimuli, we need corresponding {peripheral receptors}.
Those are present in the posture and movement system of the body in joint capsules, muscles, tendons, fascia and in the skin.
They are called {mechano--receptors}\index{mechano--receptors} and exist in different forms. 
As the receptors of the equilibrium system, they transmit information about the {position and movement of the body} to the central nervous system, over sensitive afferent (incoming) nerves\index{nerves!afferent}.
The central nervous system {reacts} to these afferent stimuli with efferent (outgoing to the periphery) impulses\index{nerves!efferent}.

Let's have a look at an example: \newline
If we're walking through a dark room, we are still informed about the position of our body, wihtout the visual control, like a blind person.
This is due to sensitive receptors at the bottom of our feet and in the joints of our legs, which maintain constant ``radio contact'' with the central nervous system over the afferent nerves.
If we would suddenly step on an inclined plane, these receptors sends the news of this event immediately to the central nervous system, which in turn initiated to counter movements, so that a fall can be avoided.
In this case, the proproiception works out.

A person suffering from diabetes with severe peripheral neuropathy\footnote{Damage to the afferent nerves which transmit signals to the brain or spinal cord},
the risk of falling is considerably higher.
The overall picture of an affected person walking is almost like on ``walking on cotton'', they are dependant on visual control by the eyes.
In this case, the prorprioception can't work.

\begin{description}
\item[What happens now under stress influence? ---]
  When the feedback system between the proprio--receptors and the muscles is well develloped through constant use, the balance system is able to cope the physical blance is maintained.
  When stress disturbs the even activation of this system, we're ``not centered anymore'', and the sense of balance and the conscious sense of the room around the person gets lost.
  Accidents are prone to happen, leading to scratches, bruises or even broken bones.
  Our sense of proprioception isnt focussed on the room around the person anymore, but on escaping a danger.
\item[Proprioception and learning:] The more we look at the intricate collaboration of brain and body, the clearer and more obvious it becomes that \emph{movement is absolutely necessary for learning}\\index{learning!movement}.
  Modern neurological research\cite{Connectome} shows that there exists dedicated discrete nerve networs (connectomes)\index{connectome} which indeed link the motor cortex of the brain to the adrenal glands, but also to the afective ares, which deals with feelings and emotions.

  Movement awakes and activates many of our mental capcities. Movement integrates and ankors new information and experiences into our neuronal network.
  And movement 
  
\end{description}

\end{document}