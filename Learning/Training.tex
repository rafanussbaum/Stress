\documentclass[../main.tex]{subfiles}
\graphicspath{{\subfix{../images/}}}
\begin{document}


        From the mentioned theories of learning and the definition of the problem, we see that training (practicing)\index{training}\index{practicing} is central to achieve certain given or self defined goals.


        Based on the example of training of social skills we can get leads to where the inhibitions, disturbing factors or promoters are to be found in us.

        There are many difficulties arising for people with a psychotic or neurotic disorders or just people with daily problems just from the fact, that they are inhibited, maladjusted or insecure in their \emph{social behavior}.

\mytextbox[0]{Translated from \cite{Rueckle}
  \vspace{5mm}
  
  \noindent Social skills are reactions or patterns of reactions which allow it a person to succeed while approaching others or in the interaction with others. Covered by it is what to say or to do in a given situation (content), how (style) and when (timing) it is to be said or done. And how to assure that in the conversation partner the desired reactions gets triggered (consequences).
}

\epigraph{Our wishes are the premonitions of the capacities, which are inside of us.}{\textit{Johann Wolfgang von Goethe}}

One of the most prevalent problems with social skills is lacking confidence and the inability to put the thoughts and wishes into words in a clear, direct and not aggressive manner. Even though problems that adults have in that domain often got caused in the childhood, it's nevertheless important to diligently work on them.

\textbf{People with insufficient social skills, which is an important, often unconscious stressor, can be lead with training to a more successful behavior.}

\vspace{1cm}

\mytextbox[0]{Translated from Bert Hellinger
  \vspace{5mm}
   
  \noindent Bedazzlement

  A circus acquired a polar bear. Given that they needed him only for exhibitions, they put him in a cage.
  It was so narrow that he couldn't even turn around in there. So he continued to go two steps forward in his cage, then two steps backward.

  
  After many years they had pity with the polar bear and sold him to a zoo. There he had a big cage and a big compound. But he continued to go two steps forward, then two steps backward. Another polar bear asked him: ``Why are you doing this?'' he answered: ``That's because I was trapped for so long in a little cage.''
  }


\end{document}