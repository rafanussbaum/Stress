\documentclass[../main.tex]{subfiles}
\graphicspath{{\subfix{../images/}}}
\begin{document}

Which experiences you make while learning with your study partner?
Did the learning success, which you hoped for take place?

\section{Questions to Learning and Training}
\begin{enumerate}
\item I can put obstacles in the path of my learning process. Which ones? Which ones am I more prone to?
\item Taking notes is not just about noting things down.
\item Instruct somebody to study with with passive auditory listening with a sound recording device (cell phone).
\item It's possible to come up with different ways of making yourself pictures about the material to study.
\item You learn by teaching.
\item Examples help to make the material more tangible.
\item To apply new knowledge in everyday situations brings advantages with it.
\item Is repetition funny?
\item Learning can awake joy.
\item Which is your most valuable food, the one you feed your brain with?
\item Which five things do you like most to doodle?
\item Make yourself an own picture to each of the ten mentioned learning aids.
  Can you do it without looking at the last ten questions?
\item Explain ``Mens sana in corpore sano''. In which situations is this especially true?
\item The limbic system plays a special role in this context. Which one?
\item Information gets to our brain over different senses.
\item In terms of efficiency of storage, we can distinguish between three types of memory.
\item Fissures
\item Name the parts of the limbic system.
\item The limbic system is involved in different behaviors. Which ones do you know?
\item Which time frame of storage is associated with the three types of memory?
\item Justify: Learning always happens in interaction with the surrounding.
\item Three researcher significantly contributed to learning psychology. Talk about them.
\item Is a problem the same as a task?
\item By which means can people with insufficient social skills be lead to more successful behavior?
\item Instruct metacognitive exercises.
\item What should be the effects of them?
\item Attribute every one of the seven named colors one to two characteristics.
\item To which chakras do these colors correspond?
\item How could I use this knowledge about colors and their effects in a practical situation?
  \item How is our brain build up?
\item Name the essential processes which get attributed to the hemispheres of the brain.
\item Can you explain the general mechanism of our balance organ?
\item Name the three exercises to strengthen the sense of balance.
\item Without referring to it by name, the script contains the cogwheel model,
  which Dr~Alois Br\"ugger developed as a means to remember an optimal seated posture. Explain it.
\item Bad posture doesn't only strain externally.
\item Demonstrate the normal posture of the human body. Demonstrate three other, less normal postures.
\item There was an exercise about the central axis, no?
\item What is proprioception?
\item What does the acronym PACE signify? What are the advantages of this technique?
\item Instruct one of the exercises and explain it's effects.
  \item What are insufficient social skills? Name a few examples and comment on their frequency. 
  \end{enumerate}

  \section{Tasks to Learning and Training}

  \begin{enumerate}[label = \Alph*]
  \item Make sufficient amount of photo copies of the three metacognitive exercises and practice them every day at the same time. Are you making progress?
    % Take them along for the next day of course.
  \item Physical exercises are as important in the regulation of stress as are things which support the psyche or the soul.
    Often they are the most efficient bridge which are able to reach people which are under stress influence.
    Therefore: train the sun salute, if possible daily.
  \item Practice once a day the PACE or the small morning program.
    Try it as warm up before the sun salute.

    Comment the individual movements as you go along, always say what you're about to do. That makes you more attentive to the individual movements and it helps you while teaching.

  \item Try to incorporate everywhere balance exercises and proprioception training.
    (For instance putting one foot on the other while talking to somebody, 
    execute different tasks while standing on one foot only,
    walking over uneven terrain.)
    \end{enumerate}
\end{document}