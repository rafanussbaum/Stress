\documentclass[../main.tex]{subfiles}
\graphicspath{{\subfix{../images/}}}
\begin{document}
We humans are learning our whole life. If by reading this sentence, you feel like putting this material aside, I want to urge you to read on in spite of your inner resistance\index{learning!resistance}.
There are people who associate learning with pleasure, as they could already shine with a good performance at school.
Those people associated learning with positive feelings\index{learning!positive feelings} and applied this strategy over and over again, even if most of the time unconsciously.

There are people who associate learning with negative emotions\index{learning!negative feelings} and put up an inner resistance against learning.
This might be because already in school that person wasn't able to live up the the expectation of the school and parents.
The outside pressure creates an inner counter pressure and we have a perfect stress situation.

A lot of the people in this situation believe themselves to be incapable of learning, of not being intelligent enough to master the material. For this type of people can look forward to get rid of these inner voices and replace them by your own learning successes. We are born to learn.

In general it's true, that we can learn anything that we want to learn. We really have to want to learn it and don't put up any obstacles to the learning process.
There are different learning strategies, to acquire the study material. The key to success of the learning process are the repetitions\index{learning!repetition}. Given that learning is a process, the cycle of practicing is needed to allow the material to settle in the brain.

Somebody reading a nonfiction text and is being disillusioned about the amount of data in it, should ask themselves first, if they really want to take up all this new knowledge. If you really want to learn it, then it is obvious, that all this material isn't available yet in your memory as knowledge.
We can spare the negative feelings, we already know before starting reading that we are going to encounter new territory.

\mytextbox[0]{translated from ~\cite{SpitzerLernen} 

  \vspace{5mm}
  The day of September 11, 2001 will be in most of us very vividly in memory. Shortly before and after 3 pm, Central European Time, two passenger planes crashed into the two towers of the World Trade Center. Who has seen the pictures won't be able to get them out of their head: two burning skyscrapers which collapse within an hour and bury thousands of innocent people underneath them.\\
Where were you, when you heard the first time about it? Who else was with you? With whom did you first talk about it?\\
Most people can answer these questions with ease, where as the afternoon of 9/10 or 9/12 disappeared forever in the mist of the not accessible memories of the past.\\
These messages show two qualities, which automatically make our brain register and store events exactly as they were happening and were perceived by us. These two qualities are novelty and significance. Important news we hear once and we memorized them immediately.\\
It's not just political events that have these character of importance and novelty. Most people clearly remember their wedding day, the first kiss, the first hug, the first declaration of love or the first night with the partner. In other words meaningful events in their personal life.}

\setlength\epigraphwidth{.4\textwidth}
\end{document}