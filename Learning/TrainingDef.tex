\documentclass[../main.tex]{subfiles}
\graphicspath{{\subfix{../images/}}}
\begin{document}


      The term training\index{training} (practice) generally means a measure, which aims at improving the physical, psychological, mental and/or motor capacity.
      In sports it means a regular/repeating physical strengthening with the goal to enhance the physical fitness and the skill levels in the specific sport.

      In the domain of medicine and psychology, there are little distinct terms. Professionals maybe talk of treatment programs, which in my opinion doesn't have too much to do with training.
      Nevertheless we can say, that there are many possibilities to optimize our body in the psychological domain and therefore in the medical domain, too --- through training, for instance Autogenic Training, focused muscle relaxation, posture improvement by practicing gymnastics or yoga and muscle formation by training weights.


      Each form of {learning changes} us. Learning is about  acquiring knowledge and skills and  changing thinking patterns, world views and behaviors. For instance learning French will over time lead to adapting the French way of thinking (the way that French people think).
    
      When I learned something, I have now a skill I didn't have before. I'm able to demonstrate this new skill in with my behaviors.
      Learning also means changing my behaviors and my views respective a person or an object.

      {Inhibitions might hinder a person to produce this new skill or view. This leads to the dilemma: Being able to do versus really doing it. To conclude: {learning always happens in interaction with the surrounding.}}

      Learning doesn't only mean acquiring positive behaviors, meaning behaviors which get interpreted positively by the current surrounding, but as well acquiring negative behaviors.
      This lead to the questions, if me as the teacher can even evaluate, how my teaching activities will influence my students? This is not really possible to know ahead of time.
Due to this reason, I have to foster the student in his/her own context. I promote the sides in him, which already are positive.
      
Learning doesn't always happen in a planned and focused way, often it happens more or less {randomly or under pressure} --- maybe through random encounters and activities, through situations which put up pressure (I have to earn money) or through suggestion.
\emph{Learning is the integration of problems into process} ---  \emph{A problem becomes a task through a learning process}. The problem gets recognized and analyzed and therefore resolvable. The same type of situations will not be seen as problematic anymore after the process of learning: ``No Problems''. Therefore we are again at a form of stress regulation (more in the section~\ref{Le:problem} on page ~\pageref{Le:problem}: ``The definition of the problem'').

\mytextbox[0]{Translated from \cite{SpitzerLernen}
  \vspace{5mm}
  
  \noindent Two groups of two subjects heard one of the following stories, which differentiate in their emotional content.
  
  Story 1: A boy drives with his mum through the city to visit dad who works in a hospital. Once there, he gets demonstrated a series of medical procedures.

  Story 2: A boy drives with his mum through the city and gets hurt very badly in a car accident. They quickly bring him to a hospital, where a series of medical procedures get executed. 

  Both groups had a presentation of a list of medical procedures of the clinic and then got send home. A week later, they were summoned to the lab again and were being asked about the medical procedures of the clinic.

  Even though both stories had the same length and started and ended the same way, the study of the retention rate a week later showed that the people who have heard the emotional story number two were way better at retaining the details of the medical procedures.
}

In this context, let's look at a few theories from psychology, which explain human learning as an acquiring of knowledge or behaviors: the learning theories.
\emph{Pawlow}\index{Pawlow} published the \emph{theory of classical conditioning}\index{classical conditioning} or \emph{signal learning}.
Who hasn't heard of the famous experiment, where Pawlow rang a bell while feeding his dog. Later on, these ``conditional reflexes'' (due to experiences) got triggered by only hearing the signal.


\mytextbox[0]{Translated from \cite{ZimbardoPsych}
  \vspace{5mm}
  
  \noindent The Russian physiologist Ivan Pavlov (1849--1936) actually didn't intend to research neither classical conditioning nor another psychological phenomena. He discovered classical conditioning while doing research on digestion, for which he got the Nobel price in 1904.
  
  Pavlov developed a method, which allowed to examine digestion conditions on dogs. He implanted tubes into their glands and digestive organs to extract the body fluids out of the body and therefore make it possible to measure and analyze these samples. In order to produce these secretes, Pavlov applied meat powder to the mouth of the dogs.

  After applying these treatments several times, Pavlov started to discover an unexpected behavior of his dogs. They started to produce saliva even before they were given the meat powder. In fact, they started to produce saliva only seeing the food, later upon seeing the assistant who brought the food, and even just the sounds of the assistant's steps.
  Each stimulus, which precedes on a regular level the food, could increase the flow of saliva. Pavlov discovered so to say by chance, the learning can be caused by the association of two stimuli.

  Pavlov had the capacity and the thirst for knowledge to research this phenomena. He ignored the advice of the leading physiologist of his time, Sir Charles Sherrington, to abandon the useless research of ``mental'' secretions.
  Pavlov quit his work on digestion and by doing so changed the face of psychology forever.
  
}

\emph{Skinner}\index{Skinner} phrased the theory of \emph{learning through success}\index{learning!through success}. It says, that if a behavior is rewarded, it will more likely be repeated that if it gets punished.

\mytextbox[0]{Translated from \cite{ZimbardoPsych}
  \vspace{5mm}
  
  \noindent
  While being associated with reinforcement or punishment certain stimuli which precede a specific reaction gain the ability to determine the context of the behavior. These stimuli are called discriminatory clue stimuli.

  Organisms learn that their behavior and the absence of other behavior very likely has a certain effect on the surroundings. For instance in the presence of a green signal the behavior of crossing the intersection gets reinforced. Is the signal red, this behavior gets punished --- it can end in a fine or an accident. Skinner named in 1953 the effects of discriminatory stimulus, behavior and consequence as triple contingency and believed, that those can explain most human behaviors.
  }

  In the view of \emph{Bandura}\index{Bandura} we learn by \emph{observing}. We observe other people at a certain behavior and whether it gets rewarded or punished we apply this behavior ourselves or not. This ``learning by model''\index{learning!by model} requires that we are capable and ready to execute the observed behavior.

\mytextbox[0]{Translated from \cite{ZimbardoPsych}
  \vspace{5mm}
  
  \noindent A classic demonstration of human learning by observation happened 1963 in the lab of Albert Bandura. After kids could observe adults pushing, beating and stomping on a plastic clown puppet called Bobo, they showed higher number of similar incidences themselves than kids of a control group who didn't see the aggressive models.

  There's no questions, that we learn a lot --- prosocial (helpful) as well as antisocial (hurtful) behaviors --- by model; but there are many models in the world. Which variables determine, which models most likely will influence you?
  Research delivers the following general consequences:

  The behavior of a model will be especially influential, when
  \begin{itemize}
  \item there's a perception that the behavior brings reinforcing consequences;
  \item the model gets perceived as positive, popular and respected;
  \item there's a perception that there are similarities with respect to attributes and traits of the model with the observer;
  \item the observer gets rewarded for putting his attention to the behavior of the model;
  \item the behavior of the model is well visible and salient --- when it contrasts as a clear image against the background of competition between models;
    \item it is possible for the observer to imitate the behavior.
  \end{itemize}


  % Systemic approaches and Jean Piaget: https://en.wikipedia.org/wiki/Jean_Piaget
}



\end{document}