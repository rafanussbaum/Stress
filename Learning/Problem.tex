\documentclass[../main.tex]{subfiles}
\graphicspath{{\subfix{../images/}}}
\begin{document}


A \emph{problem}\index{problem}\footnote{Greek = what's presented, brought in front of you} gets created, when the knowledge or the skills is missing to achieve a specific goal.
It shows how important it is in our lives to train \emph{creativity}, in order to find the way to \emph{several possible solutions}.

\epigraph{Learning is like rowing against the current. As soon as you stop, you drift back.}{\textit{Benjamin Britten}}

The problem is characterized by the tension between the goal and the stating situation, which is different from the goal.
The process of eliminating a problem by overcoming the described tension is called \emph{solving a problem}.\index{problem!solving}
It is evident, that in this tension there's a certain pressure (stress) created, which gets called the inhibition to solve the problem.

\epigraph{The problem is serious, the solution cheerful.}{\textit{Bert Hellinger}}

A problem is not the same as a task. We have rules for a task, to come to the solution.
In order to recognize the rules, it's important to relieve the pressure (stress) as soon as possible. Otherwise an overburdening of the pressure doesn't allow for a solution anymore.
As we all have a different amount of knowledge, problems as well as tasks are \emph{specific to the person}. This proves that something that can be a huge stress can be solved elegantly by another person.
We distinguished between problems which concern only one individual from the ones which are a problem for many people.

Changing the inner views through targeted and skilled transforming \emph{problem based thinking} to \emph{task based thinking} and eventually to \emph{solutions based thinking} can be assisted by training metacognitive elements (see section~\ref{metacognitive} on page~\pageref{metacognitive}).

If you can split up a problem in multiple parts, it's called a \emph{hierarchical problem}. if this isn't possible, then it's a \emph{elementary problem}.

        \setlength\epigraphwidth{.8\textwidth}
        \epigraph{The vengeance
          
          \vspace {5mm} \noindent A small trader writes a letter to a big supplier:
          
  Dear Sir,

  Four weeks ago I ordered a box of curd soap. It still didn't arrive. My stock is depleted and I have significant losses through your negligence. We, the small ones, have to be contend to be treated like dirt by the big traders.
  
  The letter went on for several pages like this. After the signature, there was a little note:
  
  P.S. In the meanwhile I found the delivered box with the soap.}{\textit{Bert Hellinger}}
        \setlength\epigraphwidth{.4\textwidth}



\end{document}